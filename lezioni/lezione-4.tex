\documentclass[../appunti-analisi.tex]{subfiles}

\begin{document}

\section{Lezione 4}

\subsection{Edo a variabili separabili}

Una EDO si dice a variabili separabili se è della forma:

\[
    y'(x) = f(x) g(y(x))
\]

Parte che dipende da y viene moltiplicata a quella che dipende da x.

Dove le funzioni $f$ e $g$ sono continue nei loro domini di definizione

Il procedimento per risolverle è il seguente:

\begin{enumerate}
    \item Si cercano le soluzioni costanti $g(y)=0$ (cioè gli zeri)

        Si determinano gli eventuali $\bar y$ reali t.c. $g(\bar y)$

        $y(x)= \bar y$ sono soluzioni singolari del problema 
        
    \item Se $y \neq \bar y$ si procede separando le variabili, ovvero dividiamo per $g(y)$
\end{enumerate}

E quindi alla fine abbiamo:

\[
    \frac{y'(x)}{g(y(x))} = f(x) \overset{\text{integro rispetto ad x}}{=} \int_{{}}^{{}} {\frac{y'(x)}{g(y(x))} } \: d{x} {} = \int_{{}}^{{}} {f(x)} \: d{x} {}
\]

Uso la sostituzione $y = y(x)$ e $dy = y'(x) dx$:

\[
    \int_{{}}^{{}} {\frac{1}{g(y)} } \: d{y} = {\int_{{}}^{{}} {f(x)} \: d{x} {}}
\]

Chiamate $G$ e $F$ una primitiva di $\frac{1}{g} $ e di $f$ rispettivamente:

\[
    G(y(x)) = F(x) + c
\]

Applico la funzione inversa di $G$ a entrambi i membri per scrivere esplicitamente la soluzione:

\[
    y(x) = G ^{-1} (F(x) + c)
\]

\textbf{Esempio}

Determinare tutte le soluzioni dell'equazione differenziale:

È non lineare

\[
    y'(x) = (1-y)(2-y)x
\]

Le prime due parentesi sono $g(y)$ il resto $f(x)$

\begin{enumerate}
    \item Trovare le soluzioni costanti

        Pongo $g(y(x)) = 0 $:

        \[
            (1-y)(2-y) = 0
        \]

        quindi $y=1$ e $y=2$

    \item Cerchiamo le altre soluzioni dividendo per $g(y)$

        \[
            \frac{y'(x)}{(1-y(x))(2-y(x))} = x
        \]

        Quindi integro:

        \[
            \int_{{}}^{{}} {\frac{1}{(1-y)(2-y)} } \: d{y} {}= \int_{{}}^{{}} {x} \: d{x} {}
        \]

        Uso i fratti semplici per risolvere il primo membro:

        \[
            \frac{A}{1-y} + \frac{B}{2-y} = \frac{1}{(1-y)(2-y)} 
        \]

        \[
            A(2-y)+B(1-y) = 1
        \]

        \[
            (-A -B) y +2A + B = 1
        \]

        \begin{equation}
            \begin{cases}
              -A -B = 0\\
              2A+B= 1
            \end{cases}\,.
        \end{equation}

        $A=1$ e $B=1$
        
        Quindi:

        \[
            \int_{{}}^{{}} {\frac{1}{1-y}} \: d{y} {}- \int_{{}}^{{}} {\frac{1}{2-y} } \: d{y} {} = \int_{{}}^{{}} {x} \: d{x} {}
        \]

        \[
            -log|1-y| + log|2-y| = \frac{x ^{2}}{2} +c
        \]

        \[
            log|\frac{2-y}{1-y} | = \frac{x ^{2}}{2} +c
        \]

        Adesso devo esplicitare per $y$ quindi passo agli esponenziali:

        \[
            |\frac{2-y}{1-y} | = e^{(\frac{x ^{2}}{2} +c)}
        \]

        \[
            |\frac{2-y}{1-y} | = e^{(\frac{x ^{2}}{2})} e ^{c} = c_1 e ^{\frac{x ^{2}}{2} } >0
        \]

        Tolgo il valore assoluto:

        \[
            \frac{2-y}{1-y} = \pm c_1 e ^{\frac{x ^{2}}{2} }\overset{\text{usando un'altra costante}}{=}c_2 e ^{\frac{x ^{2}}{2} }
        \]

        \[
            \frac{2-y}{1-y} = c_2 e ^{\frac{x ^{2}}{2} }
        \]

        Con $c_2 \in \mathbb{R}$

        Noi vogliamo trovare la $y(x)$ (per semplicità pongo $c_2 = c$):

        \[
            \frac{2-y}{1-y} = c e ^{\frac{x ^{2}}{2} }
        \]

        Porto di la il denominatore:

        \[
            2-y = c ^{\frac{x ^{2}}{2} } (1-y)
        \]

        Porto di la le cose:

        \[
            (c e ^{\frac{x ^{2}}{2} })y = c ^{\frac{x ^{2}}{2} }-2
        \]

        E quindi le due soluzioni (quella costante e quella non) sono:

        \begin{equation}
            \begin{cases}
            y(x) = \frac{c e ^{\frac{x ^{2}}{2} }-2}{c e ^{\frac{x ^{2}}{2} }-1}   \\
            y=1
            \end{cases}\,.
        \end{equation}

\end{enumerate}

\textbf{Esercizio Problema di Cauchy}

Risolviamo ora il problema:

\begin{equation}
    \begin{cases}
      y'=(1-y)(2-y)x\\
      y(0)=3
    \end{cases}\,.
\end{equation}

e decidiamo qual è il più ampio intervallo su cui è definita la soluzione

Avendo già risolto la EDO imponiamo la condizione $y(0) = 3$:

\[
    y(0) = \frac{c-2}{c-1} = 3
\]

\[
    c-2 = 3c -3
\]

\[
    c = \frac{1}{2} 
\]

La soluzione del problema è quindi (sostituisco la c trovata all'equazione):

\[
    y(x) = \frac{\frac{1}{2} e ^{\frac{x ^{2}}{2} }-2}{\frac{1}{2} e ^{\frac{x ^{2}}{2} }-1} 
\]

\[
    y(x) = \frac{ e ^{\frac{x ^{2}}{2} }-4}{ e ^{\frac{x ^{2}}{2} }-2} 
\]

La soluzione è definita nel più ampio intervallo contenente $x_0 = 0 $ (per cui l'espressione ha senso) nel nostro caso il denominatore $\neq 0$

\[
    e ^{\frac{x ^{2}}{2} } - 2 \neq 0
\]

\[
    e ^{\frac{x ^{2}}{2} }  \neq 2
\]

\[
    x ^{2} \neq 2 log 2
\]

\[
    x \neq \pm \sqrt{2 log2}
\]

Quindi l'intervallo più ampio è quello che contiene zero ed è compreso tra le regole che abbiamo appena trovato:

\[
    0 \in (-\sqrt{2log2},+\sqrt{2log2}) 
\]

Osserviamo che la soluzione:

\[
    y(x) = \frac{ e ^{\frac{x ^{2}}{2} }-4}{ e ^{\frac{x ^{2}}{2} }-2} 
\]

è definita $\forall x \in \mathbb{R}$ con $x \neq \pm \sqrt{2log2}$


Il motivo per cui la soluzione del problema di Cauchy è definita su un intervallo si capisce bene se si pensa al significato fisico del nostro problema:

\begin{equation}
    \begin{cases}
        \text{x tempo}\\
        \text{y(x) evoluzione del sistema}\\
        \text{condizione iniziale}
    \end{cases}\,.
\end{equation}

Se partendo dall'istante iniziale ($x_0$) e procedendo in avanti o a ritroso nel tempo troviamo un istante per cui il sistema non esiste (nel caso di prima $\pm \sqrt{2log2}$) la $y(x)$ non esiste più, non ha senso domandarsi che cosa succede oltre quell'istante

Se lo vedo dal punto di vista matematico se accettassimo soluzioni definite su intervalli disgiunti non avremmo più l'unicità della soluzione (ce ne sarebbero 3 nel nostro caso e non una come volevo) perché avremmo rami distinti della funzione $y(x)$ definiti su intervalli disgiunti che non si raccordano tra di loro, dunque la condizione iniziale $y(x_0) = y_0$ non determina i valore della funzione $y(x)$ negli intervalli che non contengono l'istante iniziale $x_0$

\begin{itemize}
    \item \textbf{ Soluzione in piccolo (locale) } (è definita in un intorno di $x_0$)
    \item \textbf{ Soluzione in grande (globale) } (è definita in tutto l'intervallo)
\end{itemize}

\textbf{Esercizio per casa}

\begin{equation}
    \begin{cases}
      y'(x) = xy(x)+2x\\
      y(0) = 1
    \end{cases}\,.
\end{equation}

\textbf{Soluzione} 

Raccolgo: 

\[
    y'(x)=x(y+2)
\]

Trovo le soluzioni stazionarie:

\[
    y+2=0
\]

\[
    y=-2
\]

Trovo le altre:

\[
    \int_{}^{} {\frac{1}{y+2}} \: dy = \int_{}^{} {x} \: dx 
\]

\[
    y+2 = c e ^{ \frac{x^{2}}{2}}+c
\]

Impongo le condizioni di Cauchy e trovo c sostituendo:

\[
    y=3e ^{ \frac{x^{2}}{2}}-2
\]

Quindi la soluzione completa è:

    \begin{equation}
        \begin{cases}
            y=-2\\
            y=3e ^{ \frac{x^{2}}{2}}-2
        \end{cases}\,.
    \end{equation}



\subsection{EDO lineari del II ordine}

\[
    a_2(x)y''(x) + a_1(x) y'(x) + a_0(x) y(x) = f(x)
\]

con $a_0(),a_1(),a_2(),f()$ continue in I

In forma normale:

\[
    y''(x) + a(x) y'(x) + b(x)y(x) = f(x)
\]

se pongo $f(x) = 0$ ho la omogenea associata (2)

le sue soluzioni sono linearmente indipendenti

Se abbiamo due soluzioni $y_1$ e $y_2$ di:

\[
    a_2(x)y''(x) + a_1(x) y'(x) + a_0(x) y(x) = 0
\]

Poniamo:

\[
    y(x) = c_1 y_1(x) + c_2 y_2(x)
\]

io so che le soluzioni soddisfano l'equazione (per definizione):

\[
    a_2(x)y_1''(x) + a_1(x) y_1'(x) + a_0(x) y_1(x) = 0
\]

\[
    a_2(x)y_2''(x) + a_1(x) y_2'(x) + a_0(x) y_2(x) = 0
\]

adesso:

\[
    y(x) = c_1 y_1(x) + c_2 y_2(x)
\]

Derivo due volte:

\[
    y'(x) = c_1 y_1'(x) + c_2 y_2'(x)
\]

\[
    y''(x) = c_1 y_1''(x) + c_2 y_2''(x)
\]

\[
    a_2(x) [ c_1y_1''(x) + c_2 y_2 ''(x) ] + a_1(x)[ c_1 y_1'(x) + c_2 y_2'(x) ] + a_0(x) [ c_1 y_1(x) + c_2 y_2(x)]= 
\]

\[
    = c_1[a_2(x) y_1''(x) + a_1(x) y_1'(x) + a_0(x) y_1(x)] + c_2 [a_2(x) y_2''(x) + a_1(x) y_2'(x) + a_0(x) y_2(x)] \overset{\text{dato che è soluzione}}{=} 0
\]

\subsection{Lineare indipendenza}


$y_1(x)$ e $y_2(x)$ sono linearmente indipendenti su I se:

\[
    c_1y_1(x) +c_2y_2(x) = 0 \Leftrightarrow c_1=c_2=0
\]

\textbf{Esercizi per Casa} 

\textbf{Esercizio 1}

\[
    y(x) = ce ^{x^{2}-x}+e ^{x^{2}-x}\int_{}^{} {xe ^{x}} \: dx = c e ^{x^{2}-x}+xe ^{x^{2}}-e ^{x^{2}}
\]

Ponendo le condizioni di Cauchy:

\[
    y(0)=2
\]

La soluzione è:

\[
    2 e ^{x^{2}-x}+xe ^{x^{2}}-e ^{x^{2}}
\]

\textbf{Esercizio 2} 

\[
    y'=\sqrt[3]{x}y^{2}
\]

Una soluzione è:

\[
    y=0
\]

Le altre le trovo facendo l'integrale di:

\[
    \int_{}^{} { \frac{1}{y^{2}}} \: dy = \int_{}^{} {\sqrt[3]{x}} \: dx  
\]

quindi $y(x)$:

\[
    y(x)  = \frac{4}{3 \sqrt[3]{x^{4}}+c}
\]

impongo le condizioni e trovo c:

\[
    y(0) = \frac{-4}{0+c} = 2
\]

quindi:

\[
    c = -2
\]

ergo il la soluzione è:

\[
    y(x) = -\frac{4}{3 \sqrt[3]{x^{4}} - 2}
\]

il denominatore deve essere $\neq 0$:

\[
3 \sqrt[3]{x^{4}} - 2 \neq 0
\]

quindi:

\[
    x \neq (\frac{2}{3}^{ \frac{3}{4}})
\]

Il più ampio intervallo è:

\[
    0 \in (-\infty, \frac{2}{3}^{ \frac{3}{4}}) 
\]

\end{document}
