\documentclass[11pt]{article}


\usepackage[italian]{babel}
\usepackage[utf8]{inputenc}	% Para caracteres en español
\usepackage{amsmath,amsthm,amsfonts,amssymb,amscd}
\usepackage{multirow,booktabs}
\usepackage[table]{xcolor}
\usepackage{fullpage}
\usepackage{lastpage}
\usepackage{enumitem}
\usepackage{fancyhdr}
\usepackage{mathrsfs}
\usepackage{wrapfig}
\usepackage{graphicx}
\graphicspath{ {./images/} }
\usepackage{setspace}
\usepackage{calc}
\usepackage{multicol}
\usepackage{bookmark}
\usepackage[normalem]{ulem}
\usepackage{cancel}
\usepackage[margin=3cm]{geometry}
\usepackage{amsmath}
\newlength{\tabcont}
\setlength{\parindent}{0.0in}
\setlength{\parskip}{0.05in}
\usepackage{empheq}
\usepackage{framed}
\usepackage[most]{tcolorbox}
\usepackage{xcolor}
\usepackage{FiraSans}
\usepackage{esint}
\usepackage{listings}
\usetikzlibrary{shadings,intersections,calc,trees,positioning,arrows,fit,shapes}
\usepackage{centernot}

\usepackage{import}
\usepackage{pdfpages}
\usepackage{transparent}
\usepackage{mathrsfs}
\usetikzlibrary{arrows}


\usepackage{pgfplots}
\pgfplotsset{compat=1.18}
\usepackage[framemethod=TikZ]{mdframed}
\usepackage{hyperref}
\hypersetup{
    colorlinks,
    citecolor=black,
    filecolor=black,
    linkcolor=black,
    urlcolor=black
}

\usepackage{subfiles}

\geometry{margin=1in, headsep=0.25in}
\setcounter{secnumdepth}{0}

\newlength{\totbarheight}

\newlength{\bardepth}

%%%%%%%%%%%%%%%%%%%%%%%%%%%%%%
% Theorem, Corollari ecc.
%%%%%%%%%%%%%%%%%%%%%%%%%%%%%%

\mdfsetup{skipabove=\topskip,skipbelow=\topskip}
%%% upto here
\newcounter{theo}[section]
\newenvironment{theo}[1][]{%
\stepcounter{theo}%
\ifstrempty{#1}%
 {\mdfsetup{%
   frametitle={%
    \tikz[baseline=(current bounding box.east),outer sep=0pt]
    \node[anchor=east,rectangle,fill=blue!20]
         {\strut Teorema~\thetheo};}}
 }%
{\mdfsetup{%
  frametitle={%
   \tikz[baseline=(current bounding box.east),outer sep=0pt]
   \node[anchor=east,rectangle,fill=blue!20]
        {\strut Teorema~\thetheo:~#1};}}%
 }%
\mdfsetup{innertopmargin=10pt,linecolor=blue!20,%
       linewidth=2pt,topline=true,
       frametitleaboveskip=\dimexpr-\ht\strutbox\relax,}
   \begin{mdframed}[]\relax%
}
{\end{mdframed}}

\newcounter{lemm}[section]
\newenvironment{lemm}[1][]{%
\stepcounter{lemm}%
\ifstrempty{#1}%
 {\mdfsetup{%
   frametitle={%
    \tikz[baseline=(current bounding box.east),outer sep=0pt]
    \node[anchor=east,rectangle,fill=blue!20]
         {\strut Lemma~\thelemm};}}
 }%
{\mdfsetup{%
  frametitle={%
   \tikz[baseline=(current bounding box.east),outer sep=0pt]
   \node[anchor=east,rectangle,fill=blue!20]
        {\strut Lemma~\thelemm:~#1};}}%
 }%
\mdfsetup{innertopmargin=10pt,linecolor=blue!20,%
       linewidth=2pt,topline=true,
       frametitleaboveskip=\dimexpr-\ht\strutbox\relax,}
   \begin{mdframed}[]\relax%
}
{\end{mdframed}}


\newcounter{corollario}[section]
\newenvironment{corollario}[1][]{%
\stepcounter{corollario}%
\ifstrempty{#1}%
 {\mdfsetup{%
   frametitle={%
    \tikz[baseline=(current bounding box.east),outer sep=0pt]
    \node[anchor=east,rectangle,fill=blue!20]
         {\strut Corollario~\thecorollario};}}
 }%
{\mdfsetup{%
  frametitle={%
   \tikz[baseline=(current bounding box.east),outer sep=0pt]
   \node[anchor=east,rectangle,fill=blue!20]
        {\strut Corollario~\thecorollario:~#1};}}%
 }%
\mdfsetup{innertopmargin=10pt,linecolor=blue!20,%
       linewidth=2pt,topline=true,
       frametitleaboveskip=\dimexpr-\ht\strutbox\relax,}
   \begin{mdframed}[]\relax%
}
{\end{mdframed}}

\newcounter{definizione}[section]
\newenvironment{definizione}[1][]{%
\stepcounter{definizione}%
\ifstrempty{#1}%
 {\mdfsetup{%
   frametitle={%
    \tikz[baseline=(current bounding box.east),outer sep=0pt]
    \node[anchor=east,rectangle,fill=blue!20]
         {\strut Definizione~\thedefinizione};}}
 }%
{\mdfsetup{%
  frametitle={%
   \tikz[baseline=(current bounding box.east),outer sep=0pt]
   \node[anchor=east,rectangle,fill=blue!20]
        {\strut Definizione:~#1};}}%
 }%
\mdfsetup{innertopmargin=10pt,linecolor=blue!20,%
       linewidth=2pt,topline=true,
       frametitleaboveskip=\dimexpr-\ht\strutbox\relax,}
   \begin{mdframed}[]\relax%
}
{\end{mdframed}}

\newcounter{propo}[section]
\newenvironment{propo}[1][]{%
\stepcounter{propo}%
\ifstrempty{#1}%
 {\mdfsetup{%
   frametitle={%
    \tikz[baseline=(current bounding box.east),outer sep=0pt]
    \node[anchor=east,rectangle,fill=blue!20]
         {\strut Proposizione~\thepropo};}}
 }%
{\mdfsetup{%
  frametitle={%
   \tikz[baseline=(current bounding box.east),outer sep=0pt]
   \node[anchor=east,rectangle,fill=blue!20]
        {\strut Proposizione~\thepropo~ - #1};}}%
 }%
\mdfsetup{innertopmargin=10pt,linecolor=blue!20,%
       linewidth=2pt,topline=true,
       frametitleaboveskip=\dimexpr-\ht\strutbox\relax,}
   \begin{mdframed}[]\relax%
}
{\end{mdframed}}

\newcounter{ese}[section]
\newenvironment{ese}[1][]{%
\stepcounter{ese}%
\ifstrempty{#1}%
 {\mdfsetup{%
   frametitle={%
    \tikz[baseline=(current bounding box.east),outer sep=0pt]
    \node[anchor=east,rectangle,fill=blue!20]
         {\strut Esempio~\theese};}}
 }%
{\mdfsetup{%
  frametitle={%
   \tikz[baseline=(current bounding box.east),outer sep=0pt]
   \node[anchor=east,rectangle,fill=blue!20]
        {\strut Esempio~\theese~#1};}}%
 }%
\mdfsetup{innertopmargin=10pt,linecolor=blue!20,%
       linewidth=2pt,topline=true,
       frametitleaboveskip=\dimexpr-\ht\strutbox\relax,}
   \begin{mdframed}[]\relax%
}
{\end{mdframed}}

%%%%%%%%%%%%%%%%%%%%%%%%%%%%%%
% SELF MADE COMMANDS
%%%%%%%%%%%%%%%%%%%%%%%%%%%%%%


\newcommand*\Eval[3]{\left.#1\right\rvert_{#2}^{#3}}
\newcommand{\teorema}[2]{\begin{theo}[#1]{}{}#2\end{theo}}
\newcommand{\cor}[2]{\begin{corollario}[#1]{}{}#2\end{corollario}}
\newcommand{\lemma}[2]{\begin{lem}[#1]{}{}#2\end{lem}}
\newcommand{\proposizione}[2]{\begin{propo}[#1]{}{}#2\end{propo}}
% \newcommand{\mprop}[2]{\begin{Prop}{#1}{}#2\end{Prop}}
% \newcommand{\clm}[3]{\begin{claim}{#1}{#2}#3\end{claim}}
% \newcommand{\wc}[2]{\begin{wconc}{#1}{}\setlength{\parindent}{1cm}#2\end{wconc}}
% \newcommand{\thmcon}[1]{\begin{Theoremcon}{#1}\end{Theoremcon}}
% \newcommand{\ex}[2]{\begin{Example}{#1}{}#2\end{Example}}
\newcommand{\defn}[2]{\begin{definizione}[#1]{}{}#2\end{definizione}}
\newcommand{\esempio}[2]{\begin{ese}[#1]{}{}#2\end{ese}}
% \newcommand{\dfnc}[2]{\begin{definition}[colbacktitle=red!75!black]{#1}{}#2\end{definition}}
% \newcommand{\qs}[2]{\begin{question}{#1}{}#2\end{question}}
% \newcommand{\pf}[2]{\begin{myproof}[#1]#2\end{myproof}}
% \newcommand{\nt}[1]{\begin{note}#1\end{note}}


\newcommand{\incfig}[2][1]{%
    \def\svgwidth{#1\columnwidth}
    \import{\subfix{./figures/}}{#2.pdf_tex}
}


\newcommand\DrawBlock[3]{
\ifx#1b\relax
  \path[draw]
    (lm\the\numexpr#2-1\relax) -- ++(0,0,#3) coordinate (blocklf)
    (bm\the\numexpr#2-1\relax) -- ++(0,0,#3) coordinate (blocklb)
    (lm#2) -- ++(0,0,#3) coordinate (blockrf)
    (bm#2) -- ++(0,0,#3) coordinate (blockrb);
  \filldraw[fill=white,draw=black]
    (lm\the\numexpr#2-1\relax) -- (blocklf) -- (blocklb) -- (blockrb) -- (blockrf) -- (lm#2);
\else  
  \ifx#1f\relax
    \path[draw]
      (fm\the\numexpr#2-1\relax) -- ++(0,0,#3) coordinate (blocklf)
      (lm\the\numexpr#2-1\relax) -- ++(0,0,#3) coordinate (blocklb)
      (fm#2) -- ++(0,0,#3) coordinate (blockrf)
      (lm#2) -- ++(0,0,#3) coordinate (blockrb);
    \filldraw[fill=white,draw=black]
      (fm\the\numexpr#2-1\relax) -- (blocklf) -- (blocklb) -- (blockrb) -- (blockrf) -- (fm#2);
  \fi
\fi
\draw (blocklf) -- (blockrf);
}

\pdfsuppresswarningpagegroup=1


\begin{document}

\title{Lezione 2}
\author{Guglielmo Bartelloni}

\maketitle
\tableofcontents
\newpage
\thispagestyle{empty}

\section{Facciamo vedere che il teorema precedente valeva anche per $n>1$}

Supponiamo che $u$ e $v$ siamo due soluzioni di (1), cioè che:

$Lu=f$ e $Lv=f$ su $I$

La differenza di queste diventano soluzione su $I=[a,b]$ dell'omogenea associata

Usando la propietà della linearità:

\[
    L(\lambda u+\mu v) = \lambda L u + \mu L v
    
\]

\[
    L(u-v) = Lu-Lv = f- f=0
\]

Se indichiamo con $V_0$ l'insieme di tutte le soluzioni dell'equazione omogenea associata ($Lw=0$ su $I=[a,b]$ e $V_0$ è l'insieme delle $w \in \mathbb{C}^n(I)$) e con $\bar u(t)$ una soluzione nota di (1)

\[
    u(x) = \bar u(x) +w(x)
\]

L'uguaglianza sopra, al variare di $w(x)$ in $V_0$ ci da tutte le soluzioni del problema di partenza. 

(Il problema quindi, diventa solo di studiare il problema omogeneo)

\section{Torniamo al I ordine}

Adesso ritorniamo al problema di I ordine (in forma normale):

\[
    (1)\ y'(x)+a(x)y(x)=f(x)
\]

dove $a()$ e $f()$ sono continue su $[a,b]$

\[
    (2)\ y'(x)+a(x)y(x)=0
\]

Secondo il teorema della prima lezione:

\[
    y(x)=z(x)+\bar y(x)
\]

Come si determina l'insieme di tutte le soluzioni (integrale generale) di (2), cioè:

\[
    (2)\ y'(x)+a(x)y(x)=0,x \in [a,b]
\]

Sia $A(x)$ una \textbf{primitiva} di $a(x)$:

\[
    A(x) = \int_{{}}^{{}} {a(x)} \: d{x} {}
\]

Moltiplichiamo i due membri della (2) per $e^{A(x)}$:

\[
    e^{A(x)} + e ^{A(x)}a(x) y(x)=0, x \in [a,b]
\]

La posso scrivere anche (la derivata di $e ^{A(x)}y(x)$):

\[
    (e ^{A(x)} y(x))' = e ^{A(x)}a(x)y(x) + e ^{A(x)}y'(x)
\]

quindi (sempre chiaramente nell'intervallo $[a,b]$):

\[
    (e ^{A(x)}y(x))'=0
\]

Questo mi dice che:

\[
    e ^{A(x)}y(x) = costante=c \in \mathbb{R}
\]

porto dall'altra parte:

\[
    y(x) = c e ^{-A(x)}
\]

espandendo $A(x)$:

\[
    y(x) = c e ^{\int_{{}}^{{}} {a(x)} \: d{x} {}}
\]

posso considerare le soluzioni come:

\[
    y(x) = c z_0
\]

dove $z_0$ è una soluzione particolare di (2).

Infatti $e ^{-A(x)}$ è soluzione di (2)

\begin{proof}
    \[
    e ^{-A(x)} = -a(x) e ^{-A(x)}
\]

ovvero

\[
    (e ^{-A(x)})'+a(x) e ^{-A(x)}=0
\]
    
\end{proof}


\subsection{Determinazione dell'integrale particolare}

Sappiamo:

\[
    (1)\ y'(x)+a(x)y(x)=f(x)
\]

\[
    (2)\ y'(x)+a(x)y(x)=0
\]

Cerco l'integrale particolare ad occhio oppure uso il \textbf{metodo della variazione della costante}

\subsubsection{Metodo della variazione della costante}

Cerco questa $c(x)$ in questa forma:

\[
    \bar y(x) = c(x) e ^{-A(x)}
\]

Ovviamente la cerco dopo che so che $\bar y(x)$ è soluzione del problema.

\begin{proof}
    Poichè $\bar y(x)$ è soluzione di (1) si ha che $\bar y'(x)+a(x) \bar y(x)=f(x)$ da cui sostituendo $\bar y(x) = c(x) e ^{-A(x)}$:

    \[
        (c(x) e ^{-A(x)})'+ a(x) c(x) e ^{-A(x)} = f(x)
    \]
    
    Deriviamo:

    \[
        c'(x) e ^{-A(x)} - c(x) a(x) e ^{-A(x)} + a(x) c(x) e ^{-A(x)}= f(x)
    \]
    
    semplifico

    \[
        c'(x) e ^{-A(x)} = f(x)
    \]

    \[
        c'(x) e ^{-A(x)} = f(x)
    \]

    \[
        c'(x) = f(x) e ^{A(x)} \rightarrow e(x) = \int_{{}}^{{}} {f(x) e ^{A(x)}} \: d{} {}
    \]

    e dunque:

    \[
        \bar y(x) = e ^{-A(x)} \int_{{}}^{{}} {f(x) e ^{A(x)}} \: d{x} {}
    \]

    Cioè l'integrale particolare
\end{proof}

Se metto tutto insieme l'integrale generale diventa:

\[
    y(x) = c e ^{-A(x)} + e ^{-A(x)} \int_{{}}^{{}} {f(x) e ^{A(x)}} \: d{x} {}
\]

\subsection{Osservazioni sulla formula}

$A(x)$ è \textbf{una} primitiva di $a(x)$ scelta una volta per tutte.

Non occorre mettere una costante arbitraria (ovvero considerare come $A(x) + K,K \in \mathbb{R}$ ) poiche l'integrale generale non cambia

Non serve neanche nell'integrale perchè verrebbe buttato dentro $c$ dell'integrale generale


\subsection{Esempi}

\[
    y'(x) = 5y(x) + e ^{x}
\]

in questo caso $a(x) = -5$

\[
    A(x) = - \int_{{}}^{{}} {5} \: d{x} {}=-5x
\]

Quindi: 

\[
    e ^{-A(x)}=e ^{5x}
\]

\[
    y(x) = c e ^{5x} + e ^{5x} \int_{{}}^{{}} {e^x e ^{-5x}} \: d{x} {} = c e ^{5x} + e ^{5x} \int_{{}}^{{}} {e ^{-4x}} \: d{} {} = c e ^{5x} + e ^{5x} (\frac{1}{4} e ^{-4x}) = c e ^{5x} - \frac{1}{4} e ^{x}
\]

Esercizio per casa:

\[
    u' + \frac{u}{t} = e ^{t}
\]

\end{document}
