\documentclass[../appunti-analisi.tex]{subfiles}

\begin{document}

\section{Lezione 1}

\subsection{Equazioni differenziali}

Le equazioni differenziali sono equazioni in cui l'incognita è un'equazione insieme a qualche sua derivata.

\subsubsection{Equazioni differenziali ordinarie}

Noi vedremo quelle del primo ordine lineari e di secondo ordine con coefficienti costanti

Problema di Cauchy: problema con condizioni iniziali.


\defn{}{Una equazione di ordine n è una equazione del tipo:
\[
    F(x,y(x),y'(x),\ldots,y^{(n-1)}(x),y^{(n)}(x))=0
\]
\[
    x \in I \subseteq \mathbb{R}
\]
dove l'incognita è la qualunque y(x). F è funzione di (n+2) variabili $x,y(x),y'(x)\ldots.$
}



L'\textbf{ordine} è dato dal massimo ordine di derivazione che compare.


Per esempio:
\[
    y'''+2y''+5y = e^x
\]
è di ordine 3

\defn{Soluzione (curva) integrale}{La soluzione di una EDO di ordine n sull'intervallo I 
    \begin{equation}\label{eq:soluzione}
         F(x,y(x),y'(x),\ldots) = 0
     \end{equation}
\[
    x \in I \subseteq \mathbb{R}
\]

$\varphi(x)$ che sia definita (almeno) in I e ivi derivabile fino all'ordine n per cui valga \ref{eq:soluzione}, ovvero:
\[
    F(x,\varphi(x),\varphi ' (x), \ldots ) = 0 
\]

$\forall x \in I$

Chiaramente cambia a seconda dell'intervallo
}

\defn{Integrale Generale}{Si chiama integrale \textbf{generale} di \ref{eq:soluzione} in I l'insieme di tutte le soluzioni di \ref{eq:soluzione} in I}


È possibile definire un'espressione più esplicita

\defn{Forma normale}{Una Equazione Differenziale Ordinaria (EDO) di ordine n si dice in forma normale se è in forma

    \[
        y^{(n)} = f(x,y(x),y'(x), \ldots ,y^{(n-1)}), x \in I
    \]
    
    Esempio:
    \[
        y'''=-5y'+sinx
    \]
    Quella sopra è un EDO di III ordine normale.
}

\defn{EDO di ordine n lineare}{Una EDO di ordine n si dice lineare se è nella forma
    \[
        a_n(x)y^{(n)}(x)+a_{n-1}(x)y^{(n-1)}+ \ldots + a_2(x)y''(x)+a_1(x)y'(x)+a_0y(x)=f(x),x \in I
    \]

    Dove le funzioni \[
        a_0(x),a_1(x),a_2(x), \ldots,a_n(x),f(x)
    \]

    sono assegnate (continue) in I

    Esempio:
    \[
        xy''+5y = sin x
    \]

}


Quando $f(x)=0$ allora l'equazione si dice l'\textbf{omogenea associata} 


Nel nostro caso le equazioni di secondo ordine lineari saranno a \textbf{coefficienti costanti}


Vediamo come si risolve il problema della determinazione delle soluzioni di EDO lineari


\subsubsection{I ordine (n=1)}

\[
    F(x,y(x),y'(x))=0
\]

La considero in forma normale:
\[
    (1)\ y'(x)+a(x)y(x)=f(x), x \in [a,b]
\]

dove le funzioni $a(x)$ e $f(x)$ sono continue in $[a,b]$


Se $f(x)=0$ si ottiene omogenea associata:
\[
    (2)\ y'(x)+a(x)y(x)=0
\]

Come si determina l'integrale generale di (1)?

Il teorema che enunciamo vale per tutte le EDO lineari di ordine n

\teorema{}{L'integrale generale di (1) in $[a,b]$ è dato dalla somma dell'integrale generale dell'omogenea associata (2)
con un integrale particolare noto di (1)
\[
\int_{{}}^{{}} {gen} (1) = \int_{{}}^{{}} {gen(2)}  + \int_{{}}^{{}} {particolare} (1)
\]
}

\begin{proof}
    



Sia $y(x)$ una soluzione qualsiasi di (1) ($y(x)$ appartiene all'integrale generale di (1))
e sia $\bar y(x)$ una soluzione particolare (nota) di (1). Voglio far vedere è che la loro differenza è una soluzione qualsiasi di (2)

Dunque per ipotesi n ha che:
\[
    y'(x)+a(x)y(x) = f(x), \forall x \in [a,b]
\]

\[
    \bar y'(x) + a(x) \bar y(x) = f(x)
\]

Entrambe soddisfano la (1)

Sottraggo membro a membro le due:

\[
    y'(x)-\bar y'(x) + a(x)y(x) - a(x) \bar y(x) = f(x) - f(x)
\]

\[
    y'(x)-\bar y'(x) + a(x)[y(x) - \bar y(x)]=0
\]

Si può scrivere anche (le derivate raccolte):
\[
    [y(x)-\bar y(x)]' + a(x)[y(x) - \bar y(x)]=0
\]

E dunque  la funzione $y(x) - \bar y(x) = z(x)$ è soluzione di (2)
Quindi:
\[
    y(x) = \bar y(x) + z(x)
\]

Viceversa se $z(x)$ è una qualsiasi soluzione di (2) e $\bar y(x)$ è una soluzione particolare di (1) 
voglio mostrare che la loro somma è soluzione di (1)

Pongo:
\[
    y(x) = z(x) + \bar y(x)
\]

Devo mostrare che $y(x)$ verifica (1)

sapendo che:
\[
    z'(x) + a(x)z(x) = 0
\]

\[
    \bar y'(x) + a(x) \bar y(x) = f(x)
\]

\[
    y'(x) = (z(x) + \bar y(x) )' = z'(x) + \bar y'(x) =
    -a(x)z(x)-a(x)\bar y(x) + f(x) = -a(x) [z(x) + \bar y(x)] +f(x)
\]

E quindi ho dimostrato che:
\[
    y'(x) = -a(x)y(x) + f(x)
\]

\[
    y'(x) +  a(x)y(x) = f(x)
\]

\[
    y(x) = z(x) + \bar y(x)
\]

\end{proof}

\end{document}
