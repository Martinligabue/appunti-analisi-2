\documentclass[../appunti-analisi.tex]{subfiles}

\begin{document}

\section{Lezione 19}

\teorema{Derivazione della funzione composta}{
    Supponiamo $\gamma(t)$ derivabile $\forall t \in I$ ovvero $\gamma'(t)$ è definito $\forall t \in I$ con $\gamma'(t) = (\gamma_1'(t),\ldots,\gamma_n'(t))$ e supponiamo che $f$ sia differenziabile in $\gamma(t) \in A$ (data $f: A \subseteq \mathbb{R}^{n}\rightarrow \mathbb{R}$) allora la funzione composta $F= f \circ \gamma: I \rightarrow \mathbb{R}$ è derivabile in t.


    Inoltre:

    \[
        F'(t) = \langle \nabla f(\gamma(t)), \gamma'(t) \rangle = \sum^{n}_{i=1} \frac{\partial f}{\partial x_i}(\gamma(t))\gamma_i'(t)
    \]

}

\begin{proof}
       Inizio dimostrando che $f$ è derivabile, ovvero che esiste finito il limite del rapporto incrementale:

       \[
           \frac{F(t+h) - F(t)}{h} = \frac{F(\gamma(t+h)) - F(\gamma(t))}{h} = \langle \nabla f(\gamma(t+h)), \underbrace{\frac{\gamma(t+h) - \gamma(t)}{h}}_\text{$1$} \rangle + \underbrace{\frac{o(\gamma(t+h) - \gamma(t)|)}{h}}_\text{$2$}
       \]

       quando $h \rightarrow 0$:

       \begin{enumerate}
        \item \[
            \lim_{ h \to 0 } \frac{\gamma(t+h) - \gamma(t)}{h} = \gamma'(t)
        \]

    \item \[
        \lim_{ h \to 0 } \frac{o(|\gamma(t+h) - \gamma(t)|)}{h} \overset{\text{moltiplico e divido per il numeratore}}{=} \lim_{ h \to 0 } \underbrace{\frac{|o(|\gamma(t+h) - \gamma(t)|)|}{|\gamma(t+h) - \gamma(t)|}}_\text{$0$} \cdot \underbrace{\frac{| \gamma(t+h) - \gamma(t)|}{|h|}}_\text{quantita' finita} = 0 
    \]

    Quantità finita perché:

    \[
        \lim_{ h \to 0 } \frac{|\gamma(t+h) - \gamma(t)|}{|h|} = \lim_{ h \to 0 } ( \sum^{n}_{i=1} (\frac{\gamma_i(t+h) - \gamma_i(t)}{h})^{2})^{ \frac{1}{2}} =
    \]

    \[
        = \sqrt{\sum^{n}_{i=1} ( \gamma_i'(t))^{2}} = \underbrace{| \gamma'(t)|}_\text{lunghezza di un vettore} >0
    \]

       \end{enumerate}

\end{proof}

\textbf{Esercizio} 

\[
    f(x,y)= x e ^{y}
\]

\[
    \gamma(t) = (x(t), y(t)) = (\cos (t) , \sin (t))
\]

\[
    F(t) = (f \circ \gamma) (t) = f(\gamma(t)) = \cos (t) e ^{\sin (t)}
\]

\[
    F'(t) = -\sin (t) e ^{\sin (t)} \cos (t)
\]

Le derivate parziali sono:

\[
    \frac{\partial f}{\partial x} = e ^{y} 
\]

\[
    \frac{\partial f}{\partial y} = x e ^{y}
\]

\[
    F'(t) = \langle \nabla f(\gamma(t)), \gamma'(t) \rangle  
\]

dove:

\[
    \nabla f(\gamma(t)) = (e ^{\sin (t)}, \cos (t) e ^{\sin (t)})
\]

e:

\[
    \gamma'(t) = (-\sin (t), \cos (t))
\]

quindi:

\[
    F'(t) = e ^{\sin(t)} (- \sin (t)) + \cos (t) e ^{\sin (t)}\cos (t)
\]

\teorema{Formula del gradiente}{Se $f(x,y)$ è differenziabile in $P=(x,y)$ allora $f$ ammette derivate direzionali in $(x,y)$ per ogni direzione. Inoltre per ogni versore $\bar{v}= (a,b)$, vale:

    \[
        D_{\overrightarrow{v} }f(x,y) = \langle \nabla f(x,y), \overrightarrow{v}  \rangle = \frac{\partial f}{\partial x}(x,y)\cdot a + \frac{\partial f}{\partial y}(x,y) \cdot b
    \]

}

\newpage

\subsection{Derivate successive}

Data $f: A \subseteq \mathbb{R}^{n} \rightarrow \mathbb{R}$, $A$ aperto ed $f$ derivabile su $A$, risultano ben definite le funzioni $\frac{\partial f}{\partial x_i} \forall i=1,\ldots,n$ $(\frac{\partial f}{\partial x_i}: A \rightarrow \mathbb{R}$, vogliamo vedere se è però ulteriormente derivabile.

\defn{Derivate seconde}{ Se $\exists$ sono della forma:

    \[
        \frac{\partial^{2} f}{\partial x_i \partial x_j} = f_{x_i x_j}
    \]

    e si ottengono al variare di $i,j$ da $1$ ad $n$.
}

\defn{Matrice Hessiana}{Attraverso le derivate seconde si ottiene una matrice $n\times n$ che ha come elementi tutte le derivate seconde. Questa è chiamata matrice Hessiana, e si indica come:

\[
    Hf= \begin{bmatrix}
        \frac{\partial^{2} f}{\partial x_1 \partial x_1} & \frac{\partial^{2} f}{\partial x_1 \partial x_2} & \ldots & \frac{\partial^{2} f}{\partial x_1 \partial x_n}\\
        \frac{\partial^{2} f}{\partial x_2 \partial x_1}& \ldots & \ldots & \ldots \\
        \ldots & \ldots & \ldots & \ldots \\
        \frac{\partial^{2} f}{\partial x_n \partial x_1}& \ldots & \ldots & \frac{\partial^{2} f}{\partial x_n \partial x_n} 
    \end{bmatrix} = \begin{bmatrix}
        f_{x_1 x_1}& f_{x_1 x_2}& \ldots & f_{x_1 x_n}\\
        f_{x_2 x_1}& \ldots & \ldots & \ldots \\
        \ldots & \ldots & \ldots & \ldots \\
        f_{x_n x_1}& \ldots & \ldots & f_{x_n x_n} 
    \end{bmatrix}
\]

}

\textbf{Nota:} se la matrice Hessiana è ben definita, e quindi esistono tutte le derivate seconde allora $f$ è derivabile 2 volte in $x_0$.

\defn{Derivate Pure}{
    Sono quelle che derivano per la stessa variabile:

    \[
        \frac{\partial^{2} f}{\partial x_i \partial x_i} = \frac{\partial^{2} f}{(\partial x_i)^{2}}
    \]

}

\defn{Derivate Miste}{ 

    \[
        \frac{\partial^{2} f}{\partial x_i \partial x_j}
    \]
}

\newpage

\textbf{Esempio matrice Hessiana} 
Caso $n=2$:

\[
    Hf(x,y) = \begin{bmatrix}
    \frac{\partial^{2} f}{\partial x^{2}}(x,y) & \frac{\partial^{2} f}{\partial x \partial y}(x,y)\\
    \frac{\partial^{2} f}{\partial y \partial x}(x,y) & \frac{\partial^{2} f}{\partial y^{2}}(x,y)  
    \end{bmatrix}  
\]


\end{document}
