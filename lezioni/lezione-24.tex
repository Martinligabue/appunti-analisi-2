\documentclass[../appunti-analisi.tex]{subfiles}


\begin{document}

\section{Lezione 24}

\subsection{Integrali Multipli}

\subsubsection{Integrali doppi su rettangoli}

% Quello che in sostanza si cerca di fare è arrivare alla definizione di integrale cercando l'area di un rettangolo (come ad analisi I)

$R \in \mathbb{R}^{2}$ rettangolo chiuso $R = [a,b]\times [c,d]$:

\[
    R = \{(x,y) \in \mathbb{R}^{2}: a \le x \le b, c \le y \le d \}
\]

$D_1$ suddivisione di $[a,b]$, $D_2$ suddivisione di $[c,d]$.
$D_1 \{x_0,x_1,\ldots,x_n\}$ con $\alpha = x_0 < x_1 < \ldots < x_n = b$ ($n+1$ punti ordinati)

$D_2 \{y_0,y_1,\ldots,y_n\}$ con $\alpha = y_0 < y_1 < \ldots < y_n = b$ ($n+1$ punti ordinati)


$D = D_1 \times D_2$ suddivisione di $R$ (tanti rettagnolini)

$[a,b]$ è suddiviso in $n$ intervalli $[x_{i-1}, x_i]$ con $i = 1 \ldots n$

$[c,d]$ è suddiviso in $m$ intervalli $[y_{j-1}, y_j]$ con $j = 1 \ldots m$

$R$ è suddiviso in $n\cdot m$ rettangolini $R_{ij} = [x_{i-1}, x_i] \times [y_{j-1}, y_j]$:

\[
    R_{ij} = \{(x,y) \in \mathbb{R}^{2}: x_{i-1} \le x \le x_i, y_{j-1} \le  y \le y_i\}
\]

\textbf{Area}: $A_{ij} = A(R_{ij}) = (x_i - x_{i-1}) \cdot (y_j - y_{j-1})$


\newpage 

Supponiamo $f: R \rightarrow \mathbb{R}$ limitata (ovvero $\exists m,M$ t.c. $m \le f(x,y) \le M$ $\forall (x,y) \in R$) e per $i=1 \ldots n, j= 1 \ldots m$ consideriamo:

\[
    m_{ij} = \underbrace{inf}_\text{$(x,y) \in R_{ij}$} f(x,y)
\]

\[
    M_{ij} = \underbrace{sup}_\text{$(x,y) \in R_{ij}$} f(x,y)
\]

\defn{Somma inferiore}{ Definiamo somma inferiore di $f$ rispetto a $D$:

    \[
        s(f,D) = \sum^{n}_{i=1} \sum^{m}_{j=1} m_{ij} \cdot A_{ij}
    \]

    cioe' la somma dei parallelepipedi piccoli (vedi figura)

}

\defn{Somma superiore}{ Definiamo somma superiore di $f$ rispetto a $D$:

    \[
        S(f,D) = \sum^{n}_{i=1} \sum^{m}_{j=1} M_{ij} \cdot A_{ij}
    \]

    cioe' la somma dei parallelepipedi grandi (vedi figura)

}

Vale sempre per ogni suddivisione $D$ di $R$:

\[
    m \underbrace{(b-a)(d-c)}_\text{area di $R$} \le s(f,D) \le S(f,D) \le M(b-a) (d-c)
\]

dunque risultano ben definite le quantità:

\[
    \begin{pmatrix}
     \underbrace{inf }_\text{$D$}S(f,D)& \underbrace{sup}_\text{$D$} s(f,D) 
    \end{pmatrix}
\]

quindi si può avere:

\begin{itemize}
    \item $sup s(f,D) < inf S(f,D)$ 
    \item $sup s(f,D) = inf S(f,D)$ in questo caso $f$ si dice integrabile
\end{itemize}

\newpage

\defn{Funzione integrabile secondo Riemann}{ Sia $f: R = [a,b] \times [c,d] \rightarrow \mathbb{R}$ limitata, si dice integrabile secondo Riemann se:

\[
    sup s(f,D) = inf S(f,D)
\]

In tal caso il valore comune si dice \textbf{integrale}  di $f$ su $R$ si indica in vari modi:

\[
    \int_{R}^{} {f}
\]

\[
    \iint_R {f}
\]

\[
    I(f,R)
\]

\[
    \iint_{R} f(x,y) dx dy 
\]

\[
    \int_{b}^{a} {\int_{d}^{c} {f(x,y) dx dy}} 
\]

}

La classe delle funzioni integrabili secondo Riemann su $R$ si indica $\mathbb{R}(R)$.


\proposizione{}{
Ogni funzione costante $f(x,y) = K$ è integrabile su $\mathbb{R}$:

\[
   \iint_R {K} \: dx dy   = K(b-a) (d-c)
\]

}

\textbf{Esempio di funzione non integrabile} 

$R = [0,1] \times [0,1]$:

\[
    f(x,y) = \begin{cases}
        1 & \text{se $(x,y) \in Q$ razionale} \\
         0& \text{altrimenti}
    \end{cases}
\]

\[
    s = \sum^{n}_{i=1} \sum^{m}_{j=1} \underbrace{0}_\text{inf $f$} (x_i - x_{i-1})(y_j - y_{j-1}) = 0
\]


\[
    S = \sum^{n}_{i=1} \sum^{m}_{j=1} \underbrace{1}_\text{sup $f$} (x_i - x_{i-1})(y_j - y_{j-1}) = 1 (b-a) (d-c) = 1
\]

Essendo due valori diversi ($s \neq S$) allora non è integrabile

\subsubsection{Interpretazione Geometrica}

Nel caso unidimensionale l'integrale era l'area del sottografico (trapezioide). In due variabili è il valore di un solido detto \textbf{cilindroide}.

Sia $f \in \mathbb{R}(R)$ e $f \ge 0$.

Il suo integrale $\iint_R {f}$ si può interpretare come volume di $T$ solido in $\mathbb{R}^{3}$ detto cilindroide, delimitato dal basso da $R$ e dall'alto dal grafico di $f$.

\begin{tikzpicture}[y={(0:1cm)},x={(225:0.86cm)}, z={(90:1cm)}]

% coordinates for the lower grid
\path
  (1,3,0) coordinate (bm0) -- 
  (4,3,0) coordinate (fm0) coordinate[midway] (lm0) --
  (4,8,0) coordinate[pos=0.25] (fm1) coordinate[midway] (fm2) coordinate[pos=0.75] (fm3) coordinate (fm4) --
  (1,8,0) coordinate (bm4) coordinate[midway] (lm4)--
  (bm0) coordinate[pos=0.25] (bm3) coordinate[midway] (bm2) coordinate[pos=0.75] (bm1);
\draw[dashed]
  (lm0) -- 
  (lm4) coordinate[pos=0.25] (lm1) coordinate[midway] (lm2) coordinate[pos=0.75] (lm3);

% the blocks
\DrawBlock{b}{1}{4}
\DrawBlock{b}{2}{3.7}
\DrawBlock{b}{3}{4.3}
\DrawBlock{b}{4}{5}
\DrawBlock{f}{1}{3.3}
\DrawBlock{f}{2}{3.5}
\DrawBlock{f}{3}{4}
\DrawBlock{f}{4}{4.7}

\foreach \Point/\Height in {lm1/3.7,lm2/4.3,lm3/5}
  \draw[ultra thin,dashed,opacity=0.2] (\Point) -- ++(0,0,\Height);

% the lower grid
\foreach \x in {1,2,3}
  \draw[dashed] (fm\x) -- (bm\x);
\draw[dashed] (fm0) -- (bm0) -- (bm4);
\draw (fm0) -- (fm4) -- (bm4);
\draw[dashed] (lm0) -- (lm4);

% coordinates for the surface
\coordinate (curvefm0) at ( $ (fm0) + (0,0,4) $ );
\coordinate (curvebm0) at ( $ (bm0) + (0,0,4) $ );
\coordinate (curvebm4) at ( $ (bm4) + (0,0,6) $ );
\coordinate (curvefm4) at ( $ (fm4) + (0,0,5.7) $ );

% the surface
\filldraw[ultra thick,fill=gray!25,fill opacity=0.2]
  (curvefm0) to[out=-30,in=210] 
  (curvefm4) to[out=-4,in=260]
  (curvebm4) to[out=215,in=330]
  (curvebm0) to[out=240,in=-20]
  (curvefm0);

% lines from grid to surface
\draw[very thick,name path=leftline] (curvefm0) -- (fm0);
\draw[very thick] (curvefm4) -- (fm4);
\draw[very thick,name path=rightline] (curvebm4) -- (bm4);
\draw[very thick,dashed] (curvebm0) -- (bm0);

% coordinate system
\coordinate (O) at (0,0,0);
\draw[-latex] (O) -- +(5,0,0) node[above left] {$x$};
\path[name path=yaxis] (O) -- +(0,10,0) coordinate (yaxisfinal) node[above] {$y$};
\draw[-latex] (O) -- +(0,0,5) node[left] {$z$};
\path[name intersections={of=yaxis and leftline,by={yaxis1}}];
\path[name intersections={of=yaxis and rightline,by={yaxis2}}];
\draw (O) -- (yaxis1);
\draw[densely dashed,opacity=0.1] (yaxis1) -- (yaxis2);
\draw[-latex] (yaxis2) -- (yaxisfinal);

% for debugging
%\foreach \Name in {bm0,fm0,lm0,fm1,fm2,fm3,fm4,bm4,lm4,bm1,bm2,bm3,lm1,lm2,lm3,%
%curvefm0,curvebm0,curvebm4,curvefm4}
%  \node at (\Name) {\Name};  
\end{tikzpicture}

Ogni addendo di $s$ e $S$ è un parallelepipedo (alto $m_{ij}$ per $s$ e basso $M_{ij}$ per $S$).

Quindi $s$ è il volume del solido \textbf{contenuto} in $T$ e $S$ il volume del solido \textbf{che contiene} $T$.


Siccome l'integrale è $sup s = inf S$ abbiamo che esso è proprio il volume di T.

\teorema{}{ Sia $f: R \rightarrow \mathbb{R}$ limitata, allora $f$ è integrabile secondo Riemann su $R ( f \in \mathbb{R}(R)) \Leftrightarrow  \forall \varepsilon$ esiste una suddivisione $D_\varepsilon$ di $R$ per cui:

    \[
        S(f,D_{\varepsilon}) - s(f,D_{\varepsilon}) < \varepsilon
    \]

}


\proposizione{}{Se $f$ è continua su $R$ (quindi limitata) allora è integrabile.}


\newpage 

\subsection{Calcolo degli integrali doppi (rettangoli)}

Si fanno variare $x$ e $y$ separatamente, per ottenere due integrali semplici.

Integrare parzialmente rispetto ad $x$: considero le tracce di $f$ con $y$ fissato e integro rispetto a $x$. 

Integrare parzialmente rispetto ad $y$: considero le tracce di $f$ con $x$ fissato e integro rispetto a $y$. 

\teorema{di riduzione}{ Sia $f \in \mathbb{R}(R)$ dove $R=[a,b]\times [c,d]$

    \begin{enumerate}
        \item Se, per ogni $y \in [c,d]$, esiste l'integrale:

            \[
                G(y)=\int_{a}^{b} {f(x,y)} \: dx 
            \]

            allora la funzione $y \rightarrow G(y)$ è integrabile in $[c,d]$ e vale la formula:

            \[
                \iint_R {f} = \int_{c}^{d} {G(y)} \: d y = \int_{c}^{d} {\left(\int_{a}^{b} {f(x,y)} \: dx \right)} \: d y 
            \]

        \item Se, per ogni $x \in [a,b]$ esiste l'integrale 

            \[
                H(x) = \int_{c}^{d} {f(x,y)} \: d x d y
            \]

            allora la funzione $x \rightarrow H(x)$ è integrabile in $[a,b]$ e vale la formula:
            
            \[
                \iint_R {f} = \int_{a}^{b} {H(x)} \: dx = \int_{a}^{b} {\left(\int_{c}^{d} {f(x,y)} \: d y \right)} \: dx 
            \]

    \end{enumerate}
}

Chiaramente se la $f$ è continua in R allora va bene (perché la funzione è integrabile).

\teorema{Formule di riduzione}{Sia $f: R=[a,b]\times [c,d] \rightarrow \mathbb{R}$ continua, allora $f \in R(\mathbb{R}) $ e si ha:

\[
    \iint_R {f(x,y)} \: d x d y = \int_{a}^{b} {\left(\int_{c}^{d} {f(x,y)} \: d y \right)} \: dx   = \int_{c}^{d} {\left(\int_{a}^{b} {f(x,y)} \: dx \right)} \: d y 
\]

}

\newpage 

\textbf{Esempio} 

Sia $f(x,y)= x^{-3} e ^{ \frac{y}{x}}$ dove $R = [1,3] \times [0,1]$

$f$ è continua in $R$ e dunque possiamo usare la formula:

\[
    \iint_R {f(x,y)} \: dx d y = \iint_R {x ^{-3} e ^{ \frac{y}{x}}} \: d x d y 
\]

\begin{equation}\label{eq:riduzione_prima}
    \int_{a}^{b} {\left(\int_{c}^{d} {f(x,y)} \: d y \right)} \: dx   = \int_{1}^{3} {\left(\int_{0}^{1} {x ^{-3} e ^{ \frac{y}{x}}} \: d y \right)} \: dx 
\end{equation}

\begin{equation}\label{eq:riduzione_seconda}
    \int_{c}^{d} {\left(\int_{a}^{b} {f(x,y)} \: dx \right)} \: d y  = \int_{0}^{1} {\left(\int_{1}^{3} {x ^{-3} e ^{ \frac{y}{x}}} \: d y \right)} \: dx 
\end{equation}

usiamo la \ref{eq:riduzione_prima}:

\[
\int_{1}^{3} {\left(\int_{0}^{1} {x ^{-3} e ^{ \frac{y}{x}}} \: d y \right)} \: dx  = \int_{1}^{3} {d x \left(\int_{0}^{1} {x ^{-3} e ^{ \frac{y}{x}}} \: dy \right)}
\]

prima facciamo:

\[
    \int_{0}^{1} {x ^{-3} e ^{ \frac{y}{x}}} \: d y = x ^{-3} \int_{0}^{1} { e ^{ \frac{y}{x}}} \: d y = x ^{-3} \Eval{[x e ^{ \frac{y}{x}}]}{0}{1} = x ^{-3} [ x e ^{ \frac{1}{x}}-x ] = x ^{-2} \left(e ^{ \frac{1}{x}} -1\right)
\]

e dunque:

\[
    \int_{1}^{3} {x ^{-2} \left(e ^{ \frac{1}{x}}-1\right)} \: dx  = \int_{1}^{3} {x ^{-2} e ^{ \frac{1}{x}}} \: dx  = \int_{1}^{3} {x ^{-2}} \: dx  = \Eval{\left[ -e ^{ \frac{1}{x}} + \frac{1}{x}\right]}{1}{3} = -e ^{ \frac{1}{3}} + \frac{1}{3} + e -1
\] 




\end{document}
