\documentclass[../appunti-analisi.tex]{subfiles}

\begin{document}

\section{Lezione 22}

\textbf{Esercizio}

\[
    f(x,y) = 2x^{2} - 2xy +y^{2}
\]

Cerco i punti critici quindi quelli dove $\nabla f(x,y) = 0$

\[
    \begin{cases}
           4x -2y = 0\\
           -2x +2y = 0
    \end{cases}\, =
    \begin{cases}
           x= 0\\
           y = 0
    \end{cases}\,.
\]

Unico punto critico $(0,0)$ $f(0,0)=0$

Osserviamo che:

\[
    f(x,y) = x^{2} + (x-y)^{2}
\]

è una somma di due quadrati quindi:

\[
    f(x,y) > f(0,0),\forall (x,y) \neq (0,0)
\]

Quindi si ha che $(0,0)$ è  un punto di \textbf{minimo globale} (assoluto) stretto.


\textbf{Esercizio 1} 

\[
    f(x,y) = x^{2}-y^{2}
\]

\[
    \nabla f(x,y) = (2x, -2y)
\]

anche qui $(0,0)$ unico punto critico:

\[
    f(0,0) = 0
\]

$(0,0)$ è una sella, cioè né minimo né massimo, infatti:

\[
    f(x,y) - f(0,0) = \underbrace{x^{2}-y^{2}-0}_\text{a seconda del valore di $(x,y)$ si ha un cambiamento di segno}
\]

$\forall (x,y) \in B(0,r)$ con $r>0$. Vediamolo meglio:

\begin{itemize}
    \item Per $f(x,0)$ (considero l'incremento muovendomi lungo l'asse):

        rimane $x^{2} \rightarrow  f(x,0)=x^{2}\ge 0$ $\forall x$
    \item Per $f(0,y) = -y^{2} <0$ $\forall y$:

        abbiamo quindi un cambiamento di segno nell'intorno sferico (quindi ne' minimo ne' massimo)
\end{itemize}

\newpage


\subsection{Classificazione dei punti critici}

\subsubsection{Test dell'hessiana}

Sia $f: A \subseteq \mathbb{R}^{2} \rightarrow \mathbb{R}$, $A$ aperto con $f \in \mathbb{C}^{2}(A)$

Sia $(x_0,y_0) \in A$ un punto critico per $f$ (ovvero $\nabla f(x_0,y_0) = 0$) per cui il determinante della matrice Hessiana sia $\neq 0$ cioè:

\[
    det (Hf(x_0,y_0)) = det \begin{bmatrix}
    f_{xx}(x_0,y_0) & f_{xy}(x_0,y_0)\\
    f_{xy}(x_0,y_0) & f_{yy}(x_0,y_0)  
    \end{bmatrix} = f_{xx}(x_0,y_0) f_{yy}(x_0,y_0) - f_{xy}^{2}(x_0,y_0) \neq 0
\]

se la condizione sopra vale allora posso usare il test.

In particolare:

\begin{itemize}
    \item Se $det(Hf(x_0,y_0))>0$ e $f_{xx}(x,y)>0$  $\rightarrow (x_0,y_0)$ è punto di minimo locale
    \item Se $det(Hf(x_0,y_0))>0$ e $f_{xx}(x,y)<0$  $\rightarrow (x_0,y_0)$ è punto di massimo locale
    \item Se $det(Hf(x_0,y_0))<0$ $\rightarrow (x_0,y_0)$ è una sella
\end{itemize}

\textbf{Osservazione importante} 

Se $det(Hf(x_0,y_0)) =0$ il test dell'hessiana \textbf{non} fornisce informazioni. Devo quindi fare un'analisi diretta dell'incremento subito dalla funzione cioè:

\[
  \Delta f(x,y) = f(x,y)-f(x_0,y_0)
\]

$\forall (x,y) \in A \cap B(x_0,y_0)$


\begin{proof}
       È basata sull'approssimazione al secondo ordine della nostra funzione attraverso la formula di Taylor (al II ordine) col resto di Peano.

       \[
           \underbrace{f(\bar{x_0} + \bar{h} )}_\text{$f(x,y)$} = \underbrace{f(\bar{x_0} )}_\text{$f(x_0,y_0)$} + \langle \underbrace{\nabla f(\bar{x_0} )}_\text{$=0$},\bar{h}  \rangle + \frac{1}{2} \langle Hf(\bar{x_0} )\bar{h} ,\bar{h}   \rangle + o(|\bar{h} |^{2})
       \]

       osserviamo che:

       \begin{equation} \label{quadratica hessiana}
           \frac{1}{2} \langle Hf(\bar{x_0} )\bar{h} ,\bar{h}   \rangle
       \end{equation}

       è un polinomio di II grado in $h,k$ i cui coefficienti sono le derivate seconde quindi ci fornisce il \textbf{segno} 


       per $h \rightarrow 0$ abbiamo:

       \begin{itemize}
        \item $\bar{x_0} +\bar{h} =x$
        \item $\bar{h} = \bar{x} - \bar{x_0} $
        \item $\bar{h} =(h,k)$
       \end{itemize}

       Vediamo cosa succede:

       \[
           f(x,y) - f(x_0,y_0) = \underbrace{\frac{\partial f}{\partial x}(x_0,y_0)}_\text{$=0$}(x-x_0) + \underbrace{\frac{\partial f}{\partial y}(x_0,y_0)}_\text{$=0$} (y-y_0) + \frac{1}{2}[f_{xx}(x_0,y_0)(x-x_0)^{2} + 2 f_{xy}(x_0,y_0) (x-x_0) (y-y_0) 
       \]

       \[
           + f_{yy}(x_0,y_0) (y-y_0)^{2}] + o((x-x_0)^{2}+(y-y_0)^{2})
       \]


       Osserviamo adesso \ref{quadratica hessiana} forma quadratica dell'hessiana in $\bar{h} \in \mathbb{R}^{n}$ è un polinomio di grado 2 omogeneo nelle variabili $h_1,\ldots,h_n$

       Ad ogni forma quadratica è associata una matrice:

       \[
           q(\bar{h} ) = \sum^{n}_{i,j=1} a_{ij} h_i h_j \leftrightarrow \langle A \bar{h} , \bar{h}  \rangle
       \]

       dove $A = (a_{ij})$. Notiamo che tutti i $h_i^{2}$ hanno coefficienti $a_{ii}$ (stanno sulla diagonale).
       
       Nel nostro caso la matrice associata è la matrice Hessiana, che è \textbf{simmetrica} ($a_{ij} = a_{ji}$ per il teorema di Schwarz).

       Quindi per avere il coefficiente di posto $ij$, siccome $a_{ij} = a_{ji} \rightarrow a_{ij} + a_{ji} = 2a_{ij}$, devo dividere il coefficiente per 2.


\textbf{Esempio} 

$n=2$ e $\bar{h} =(h_1,h_2)$

Sappiamo in generale che:

\[
    q(h_1,h_2) = a_{11} h_1^{2} + \underbrace{2a_{12} h_1 h_2}_\text{A simmetrica} + a_{22}h_2^{2}
\]

per una matrice $2\times 2$:

\[
    A= \begin{pmatrix}
        \overbrace{1}^\text{$a_{11}$} & \overbrace{5}^\text{$2a_{12}$}\\
        \underbrace{5}_\text{$2a_{21}$} & \underbrace{4}_\text{$a_{22}$}
    \end{pmatrix} \leftrightarrow 
    q(h_1,h_2) = h_1^{2}+4 h_2^{2}+ 10 h_1 h_2
\]

per una matrice $3\times 3$:

\[
    A = \begin{pmatrix}
    2 & -2 & 5\\
    -2 & 3 & 0\\
    5 & 0 & 4
    \end{pmatrix} \leftrightarrow
    q(\bar{h} ) = 2 h_1^{2}+3 h_2^{2}+ 4 h_3^{2} - 4 h_1 h_2 + 10 h_1 h_3
\]

\newpage

\textbf{Studiamo il segno} 

Adesso studiamo il segno della quadratica hessiana

\defn{}{ $q(\bar{h} )$ si dice \textbf{definita positiva} se $\forall h \neq 0$ si ha $q(\bar{h} ) >0$}

\defn{}{ $q(\bar{h} )$ si dice \textbf{definita negativa} se $\forall h \neq 0$ si ha $q(\bar{h} ) <0$}

\defn{}{ $q(\bar{h} )$ si dice \textbf{indefinita} se $\exists \bar{h_1}, \bar{h_2} \in \mathbb{R}^{2}$ t.c. $q(\bar{h_1} ) <0<q(\bar{h_2} )$ cioè cambia segno}

\textbf{Nota} 

Per studiare il segno della matrice Hessiana possiamo usare il segno degli autovalori (Regola di Cartesio). Quello che sto facendo è definire il segno di $\Delta f$

\textbf{Conclusione} 

Vediamo quindi cosa succede:

\begin{itemize}
    \item $det(Hf(x_0,y_0)>0$ e $f_{xx}(x_0,y_0)>0 \rightarrow $ la forma quadratica corrispondente è definita positiva
    \item $det(Hf(x_0,y_0)>0$ e $f_{xx}(x_0,y_0)<0 \rightarrow $ la forma quadratica corrispondente è definita negativa
    \item $det(Hf(x_0,y_0)<0$ $\rightarrow $ la forma quadratica corrispondente è indefinita
\end{itemize}

\end{proof}
 
\textbf{Esempio 1} 

Classificare i punti critici delle seguenti funzioni:

\[
    f(x,y) = 2x^{3}-6xy + 3y^{2}
\]

\[
    \nabla f(x,y)=0 = \begin{cases}
               6x^{2}-6y=0\\
               -6x + 6y =0
        \end{cases}\,. = \begin{cases}
        (0,0)\\
        (1,1)
        \end{cases}
\]

Test dell'hessiana:

\[
    f_{xx}(x,y) = 12x
\]

\[
    f_{xy}(x,y)= -6
\]

\[
    f_{yy}(x,y) = 6
\]

quindi:

\[
    H f(x,y) = \begin{pmatrix}
    12x & -6\\
    -6 & 6 
    \end{pmatrix}
\]

gli sostiuiamo i punti trovati prima:

\[
    Hf(0,0) = \begin{bmatrix}
    0 & -6\\
    -6 & 6
    \end{bmatrix}
\]

quindi:

\[
    det (Hf(0,0)) = -36 <0 \rightarrow \text{ sella}
\]

\[
    Hf(1,1) = \begin{bmatrix}
    12 & -6\\
    -6 & 6
    \end{bmatrix}
\]

quindi:

\[
    det (Hf(1,1)) = 36 >0 \text{ e } f_{xx}(1,1) = 12 \rightarrow \text{ minimo locale}
\]

\textbf{Altro esercizio} 

\[
    f(x,y) = x^{3} + (x-y)^{2}
\]

\[
    \nabla f(x,y)=0
\]

\[
        \begin{cases}
    3x^{2}+2(x-y) = 0\\
    -2(x-y) = 0
        \end{cases}\, = \begin{cases}
    3x^{2}= 0\\
    x-y= 0
        \end{cases}\, = x = y =0
\]

\[
    f_{xx}(x,y) = 6x+2
\]

\[
    f_{xy}(x,y) = -2
\]

\[
    f_{yy}(x,y) = 2
\]

\[
    H f(x,y) = \begin{pmatrix}
        6x+2 & -2\\
        -2 & 2
    \end{pmatrix}
\]

\[
    det(H f(x,y)) = 2(6x+2) -4
\]

\[
    det(H f(0,0)) = 4-4 = 0
\]

Quindi non posso applicare il test dell'hessiana.

Allora considero l'incremento in un intorno sferico dell'origine:

\[
    \Delta f(x,y) = f(x,y) - f(0,0) = x^{3}+(x-y)^{2} -0 = x^{3}+(x-y)^{2}
\]

e dunque, si può escludere che in $(0,0)$ ci sia minimo o massimo relativo (perché il segno cambia).

Restringiamo adesso la funzione lungo un cammino $y=x$:

\[
    \Delta f(x,x) = x^{3}
\]

questa cambia segno dipendentemente dal valore di $x$ dunque $(0,0)$ è un punto di sella.


\subsection{Massimi e minimi su domini limitati e chiusi di $\mathbb{R}^{2}$}

In questo caso (domini limitati e chiusi) vale:

\teorema{Weistrass}{ Sia $f: K \rightarrow \mathbb{R}$ con $K$ \textbf{limitato e chiuso} di $\mathbb{R}^{2}$ continua allora $f$ è \textbf{limitata} ed assume minimo e massimo su $K$, ovvero:

    \[
        \exists (x_m,y_m),(x_n,y_n) \in K
    \]

    t.c.

    \[
        f(x_m,y_m) \le f(x,y) \le f(x_n, y_n) \forall x,y \in K
    \]

    cioè
    \begin{itemize}
        \item $f(x_m, y_m)$ valore minimo assoluto di $f$ su $K$ 
        \item $f(x_n, y_n)$ valore massimo assoluto di $f$ su $K$
    \end{itemize}


}



\end{document}
