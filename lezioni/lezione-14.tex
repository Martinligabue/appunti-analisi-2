\documentclass[../appunti-analisi.tex]{subfiles}

\begin{document}

\section{Lezione 14}

\subsection{Calcolo differenziale di funzioni di più variabili}

\textbf{Derivate Parziali}

$f'(x_0)$ misura il tasso di variazione istantanea di $f$ sul punto $x_0$:

\[
    \lim_{ h \to 0 } \frac{f(x_0+h)-f(x_0)}{h}
\]

se esiste ed è finito allora $f$ è derivabile in $x_0$ ed $f'(x_0) = limite$.

Quindi $f'(x_0)$ rappresenta la pendenza della retta tangente al grafico $f$ nel punto $p_0=(x_0,f(x_0))$

\textbf{Esempio} 

sia $f: \mathbb{R}^{2}\rightarrow \mathbb{R}$ e $P_0=(x_0,y_0)$ 

Vogliamo calcolare la derivata parziale di $f$ fatta rispetto a $x$ calcolata in $(x_0,y_0)$:

\[
    \lim_{ h \to 0 } \frac{f(x_0+h,y_0) -f(x_0,y_0)}{h}
\]

purché esista, finito.

Anche la derivata parziale di $f$ rispetto ad $y$ nel punto $(x_0,y_0)$:

\[
    \lim_{ h \to 0 } \frac{f(x_0,y_0+h) - f(x_0,y_0)}{h}
\]

purché esista, finito.

Queste due derivate si indicano:

\[
    \frac{\partial f}{\partial x}(x_0,y_0) = f_x(x_0,y_0) = D_xf(x_0,y_0)
\]

\[
    \frac{\partial f}{\partial y}(x_0,y_0) = f_y(x_0,y_0)= D_yf(x_0,y_0)
\]

\textbf{In generale}:

\[
    f: A \subseteq \mathbb{R}^{n} \rightarrow \mathbb{R}
\]

$A$ definito

\[
    \bar{n} = (x_1, \ldots ,x_n) \rightarrow f(\bar{x} ) = f \ldots 1,...,x_n) \in \mathbb{R}
\]

Sia $P_0= (x_1, \ldots  \ldots i,...,x_n) \in A$

Si definisce \textbf{derivata parziale di $f$ fatta rispetto a $x_i$ nel punto $P_0=(x_1, \ldots ,x_n)$}:

\[
    \lim_{ h \to 0 } \frac{f(x_1,\ldots,x_{i-1},x_{i+h},x_{i+1},\ldots,x_n) - f(x_1,\ldots,x_{i-1},x_{i+h},x_{i+1},\ldots,x_n)}{h}
\]

se tale limite esiste ed è finito.

Si indica:

\[
    \frac{\partial f}{\partial x_i}; D_{x_i}f;f_{x_i}
\]

Vediamo quindi avendo:

\[
    f_x(x_0,y_0),f_y(x_0,y_0)
\]

Derivata parziale rispetto ad $x$ nel punto $P_0=(x_0,y_0)$ tengo fermo $y=y_0$ rispetto a $x$:

\[
    g_1(x)= f(x,y_0) 
\]

questa è una funzione che dipende solo da $x$ (quindi diventa di una sola variabile), se questa è derivabile in $x_0$ cioè:

\[
    \lim_{ h \to 0 } \frac{g_1(x_0+h)-g_1(x_0)}{h} := g_1'(x_0) = \frac{\partial f}{\partial x}(x_0,y_0)
\]

se esiste finito.

Analogamente per $x=x_0$:

\[
    f(x_0,y) := g_2(y)
\]

se $g_2(y)$ è derivabile in $y_0$:

\[
    g_2'(y_0)  = \lim_{ h \to 0 } \frac{g_2(y_0+h)-g_2(y_0)}{h} = \frac{\partial f}{\partial y}(x_0,y_0)
\]


\textbf{Esempio di calcolo delle derivate parziali} 

Sia 

\[
f(x,y) = x \sin(xy^{3})+x^{5}y+2y^{2} 
\]

$(f: \mathbb{R}^{2}\rightarrow \mathbb{R})$


\[
    \frac{\partial f}{\partial x}(x,y) = \sin(xy^{3}) + x \cos(xy^{3}) y^{3}+5x^{4}y = \sin(xy^{3}) +xy^{3} \cos(xy^{3}) + 5x^{4}y
\]

\[
    \frac{\partial f}{\partial y}(x,y)  = x \cos(xy^{3}) 3xy^{2}+x^{5}+4y = 3 x^{2}y^{2} \cos(xy^{3}) + x^{5}+4y
\]

Se $P_0 = (2,1) = (x_0,y_0)$:

\[
    \frac{\partial f}{\partial x}(2,1) = \sin(2) + 2 \cos(2) + 5\cdot 2^{4} 
\]

\[
    \frac{\partial f}{\partial y}(2,1)  = 3\cdot 4 \cos(2) + 2^{5}+4 = 12 \cos(2) +32 + 4
\]

Usiamo il metodo delle tracce (quello di prima) con $y=1$:

\[
    f(x,1) = g_1(x)  = x \sin(x) + x^{5}+2 \cdot 1^{2} = c \sin x + x^{5}+2
\]

Sto semplicemente sostituendo e poi derivo:

\[
    g_1'(x) = \sin x+ x \cos x + 5x^{4}
\]

\[
    g_1'(2) = \sin 2 + 2 \cos 2 + 5 \cdot 2^{4} = \frac{\partial f}{\partial x}(2,1)
\]

infatti torna uguale all'altro metodo.

Adesso facciamo per $x=2$:

\[
    f(2,y) := g_2(y) = 2 \sin(2y^{3}) + 2^{5} \cdot y + 2y^{2} = 2 \sin(2y^{3}) + 32y + 2y^{2}
\]

\[
    g_2'(y) = 2 \cos(2y^{3}) 6y^{2} + 32 + 4y
\]

\[
    g_2'(1) = 12 \cos(2) + 32 + 4 = \frac{\partial f}{\partial x}(2,1)
\]


Per le funzioni in due variabili la derivabilità non implica la continuità:

\[
    derivabile \centernot\implies continua
\]

\textbf{Esempio} 

Calcolare, se esistono, le derivate parziali in $(0,0)$:


    \[
        f(x,y)=
     \begin{cases}
        \frac{xy^{2}}{x^{2}+y^{4}} & \text{se $(x,y)\neq (0,0)$} \\
        0 & \text{se $(x,y)=(0,0)$}
    \end{cases}
    \]

Consideriamo le tracce:

\[
    f(x,0) = 0 ; f(0,y) =0
\]

Dunque ottengo due funzioni identicamente nulle che sono derivabili con derivata nulla:

\[
    f_x(0,0) = f_y(0,0) = 0
\]

Osserviamo pero che in $(0,0)$ la funzione non è continua perché:

\[
    \lim_{ (x,y) \to (0,0) } \frac{xy^{2}}{x^{2}+y^{4}} \text{ non esiste}
\]

quello che ci servirà sarà il concetto di \textbf{differenziabilità}.

\subsection{Gradiente}

Se $f: A \subset \mathbb{R}^{2}\rightarrow \mathbb{R}$ è derivabile in $P_0=(x_0,y_0)$:

\[
    \nabla f(x_0,y_0)=(\frac{\partial f}{\partial x}(x_0,y_0), \frac{\partial f}{\partial y}(x_0,y_0)) \in \mathbb{R}^{2}
\]

gradiente di $f$ in $P_0$ (È un vettore).

In generale in $n$ variabili:

\[
    \nabla f(\bar{x} ) = (\frac{\partial f}{\partial x_1}(\bar{x} ), \frac{\partial f}{\partial x_2}(\bar{x} ), \ldots , \frac{\partial f}{\partial x_n}(\bar{x} )) \in \mathbb{R}^{n}
\]

\subsection{Significato geometrico delle derivate parziali}

Nelle funzioni di una variabile:

$f'(x_0)$: rappresenta retta tangente al grafico di $f$ nel punto $P_0=(x_0,f(x_0))$

Nelle due:

\[
    \frac{\partial f}{\partial x}(x_0,y_0), \frac{\partial f}{\partial y}(x_0,y_0)
\]

\[
    graf(f) = \{(x,y,z) \in \mathbb{R}^{3}, z = f(x,y), \forall (x,y) \in A \subset \mathbb{R}^{3}\}
\]

\begin{tikzpicture}[bullet/.style={circle,fill,inner sep=1pt},
 declare function={f(\x,\y)=2-0.5*pow(\x-1.25,2)-0.5*pow(\y-1,2);}]
 \begin{axis}[view={150}{45},colormap/blackwhite,axis lines=middle,%
    zmax=2.2,zmin=0,xmin=-0.2,xmax=2.4,ymin=-0.2,ymax=2,%
    xlabel=$x$,ylabel=$y$,zlabel=$z$,
    xtick=\empty,ytick=\empty,ztick=\empty]
  \addplot3[surf,shader=interp,domain=0.6:2,domain y=0.5:1.2,opacity=0.7] 
   {f(x,y)};
  \addplot3[thick,domain=0.6:2,samples y=1]  ({x},1.2,{f(x,1.2)}); 
  \draw[dashed] (1.75,0,0) node[above left]{$x_0$} -- (1.75,1.2,0)
  node[bullet] (b1) {}  -- (0,1.2,0) node[above right]{$y_0$}
  (1.75,1.2,0) -- (1.75,1.2,{f(1.75,1.2)})node[bullet] {};
  \draw (1.75,1.2,{f(1.75,1.2)}) -- (0.75,1.2,{f(1.75,1.2)+0.5})
  coordinate[pos=0.5] (aux1);
  \draw[opacity=0.5,upper left=gray!80!black,upper right=gray!60,
lower left=gray!60,lower right=gray!80!black] (2,1.2,0) -- (0.6,1.2,0)
   -- (0.6,1.2,2.2) -- (2,1.2,2.2) -- cycle;
  \addplot3[surf,shader=interp,domain=0.6:2,domain y=1.2:1.9,opacity=0.7] 
   {f(x,y)};
 \end{axis}
 \draw (aux1) -- ++ (-1,1) node[above,align=center]{slope in $x$ direction\\
  $\partial_xf(x,y)|_{x=x_0,y=y_0}$};
 \node[anchor=north west] at (b1) {$(x_0,y_0)$}; 
 %
 \begin{axis}[xshift=6.5cm,view={150}{45},colormap/blackwhite,axis lines=middle,%
    zmax=2.2,zmin=0,xmin=-0.2,xmax=2.4,ymin=-0.2,ymax=2,%
    xlabel=$x$,ylabel=$y$,zlabel=$z$,
    xtick=\empty,ytick=\empty,ztick=\empty]
  \addplot3[surf,shader=interp,domain=0.6:1.75,domain y=0.5:1.9,opacity=0.7] 
   {f(x,y)};
   \addplot3[thick,domain=0.5:1.9,samples y=1]  (1.75,{x},{f(1.75,x)}); 
  \draw[dashed] (1.75,0,0) node[above left]{$x_0$} -- (1.75,1.2,0)
  node[bullet] (b2){}
  -- (0,1.2,0) node[above right]{$y_0$}
  (1.75,1.2,0) -- (1.75,1.2,{f(1.75,1.2)})node[bullet] {};
  \draw (1.75,1.2,{f(1.75,1.2)}) -- (1.75,0.2,{f(1.75,1.2)+0.2})
   coordinate[pos=0.5] (aux2);
  \draw[opacity=0.5,upper left=gray!80!black,upper right=gray!60,
lower left=gray!60,lower right=gray!80!black] (1.75,0.5,0) -- (1.75,1.9,0)
   -- (1.75,1.9,2.2) -- (1.75,0.5,2.2) -- cycle;
  \addplot3[surf,shader=interp,domain=1.75:2,domain y=0.5:1.9,opacity=0.7] 
   {f(x,y)};
 \end{axis}
 \draw (aux2) -- ++ (0.3,1) node[above,align=center]{slope in $y$ direction\\
  $\partial_yf(x,y)|_{x=x_0,y=y_0}$};
 \node[anchor=north east] at (b2) {$(x_0,y_0)$};
\end{tikzpicture}

L'equazione del piano tangente alla funzione nel punto $P_0$ è:

\[
    z = f(x_0,y_0) + f_x(x_0,y_0) (x-x_0) + f_y(x_0,y_0) (y-y_0) 
\]


\textbf{Esempio} 

$z= x^{2}+y^{2}= f(x,y)$

Scrivere le equazioni del piano:

dove $S$ è il grafico della funzione
\[
    P_0=(0,0) \Leftrightarrow (0,0,0) \in S
\]

Usando la formula dell'equazione del piano tangente:

\[
    z=0
\]


\[
    P_0=(1,2) \Leftrightarrow (1,2,5) \in S
\]

\[
    z= 5+ 2(x-1) + 4(y-2) = 2x+4y -5
\]

quindi:

\[
    z = 2x+4y-5
\]


\end{document}
