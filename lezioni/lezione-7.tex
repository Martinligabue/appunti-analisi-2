\documentclass[../appunti-analisi.tex]{subfiles}

\begin{document}

\section{Lezione 7}

\subsection{Funzioni in più variabili}

In particolare:

\begin{itemize}
    \item funzioni reali di più variabili $f: A \in R^{n} \rightarrow \mathbb{R}$
    \item funzioni a valori vettoriali $g: A \in \mathbb{R}^{n}\rightarrow \mathbb{R}^{m}$
\end{itemize}

\defn{Vettore}{ Il vettore $n \in \mathbb{R}^{n}$ è una n-pla $x=(x_1,x_2, \ldots,x_n)$ }



Operazioni in $\mathbb{R}^{n}$:

\begin{itemize}
    \item moltiplicazione per scalare

        \[
            \forall c \in \mathbb{R}, \forall \textbf{x} \in \mathbb{R}^{n}
        \]

        \[
            c \textbf{x} = (cx_1, \ldots ,cx_n)
        \]

    \item somma

        \[
            \forall \textbf{x}, \textbf{y}  \in \mathbb{R}^{n}
        \]

        \[
       \textbf{x}  =(x_1, \ldots ,x_n), \textbf{y}  = (y_1, \ldots ,y_n)
        \]

        \[
            \textbf{x}+ \textbf{y} = (x_1+y_1, \ldots dots ,x_n+y_n)
        \]

    \item prodotto scalare

        \[
            \forall \textbf{x} ,\textbf{y} \in \mathbb{R}^{n}
        \]

        \[
            \langle x,y \rangle  = x \bullet y := \sum^{n}_{i=1} x_i y_i
        \]

        L'operazione quindi va $\langle , \rangle : \mathbb{R}^{n} \times \mathbb{R}^{n} \rightarrow \mathbb{R}$

        \textbf{Esempio} 

        \[
            x=(1,2,0,3,5) \in \mathbb{R}^{5}
        \]

        \[
            y=(2,5,1,7,3) \in \mathbb{R}^{5}
        \]

        \[
            \langle x,y \rangle  = x \bullet y = 48
        \]

        Il prodotto scalare verifica le seguenti proprietà

        \begin{enumerate}
            \item \textbf{Bilinearità} (lineare su ogni fattore):

                \[
                    (\alpha x_1 + \beta x_2) \bullet y = \langle (\alpha x_1 + \beta x_2), y \rangle  = \alpha \langle x_1,y \rangle+ \beta \langle x_2,y \rangle = \alpha x_1 \bullet y + \beta x_2 \bullet y
                \]

                \[
                    \forall x_1,x_2,y \in \mathbb{R}, \forall \alpha, \beta \in \mathbb{R}
                \]

            \item \textbf{Simmetria} (l'ordine non conta)

                \[
                    \forall x,y \in \mathbb{R}^{n}
                \]

                \[
                    x \bullet y = y \bullet x
                \]

            \item \textbf{Positività} 
                
                \[
                    \forall x \in  \mathbb{R}^{n}
                \]

                \[
                    x=(x_1, \ldots ,x_n)
                \]

                \[
                    x \bullet x = \langle x,x \rangle = \sum^{n}_{i=1} x_i^{2}\ge 0
                \]

                \[
                    x \bullet x = \langle x,x \rangle = 0 \Leftrightarrow x = 0 = (0, \ldots ,0)\ \text{vettore nullo}
                \]

        \end{enumerate}


        \defn{Norma}{
        Il numero reale (non negativo)

        \[
          |x|:=  \sqrt{x \bullet x} = \sqrt{\langle x,x \rangle}
        \]


        si chiama \textbf{lunghezza} o \textbf{norma} del vettore

        }

           
\end{itemize}

\newpage

  % proposizione 
\proposizione{Formula di Carnot}{

    \[
        \forall x,y \in \mathbb{R}^{n}
    \]

    si ha:

    \[
    |x+y|^{2} = |x|^{2} + |y|^{2} + 2x\bullet y
    \]
}

\begin{proof}
       \[
           |x+ y| ^{2} = \langle x+y , x+y \rangle = (x+y) \bullet (x+y) \overset{\text{bilinearità}}{=} \langle x,x+y \rangle + \langle y,x+y \rangle \overset{\text{sempre bilinearità}}{=} 
       \]    

       \[
           =\langle x,x \rangle + \langle x,y \rangle + \langle y,x \rangle + \langle y,y \rangle = \langle x,x \rangle + 2\langle x,y \rangle + \langle y,y \rangle= 
       \]

       \[
            = |x|^{2} + |y|^{2} + 2x \bullet y
       \]
\end{proof}


Altra cosa interessante:

 \[
    |x+y|^{2}= |x|^{2} + |y|^{2} \Leftrightarrow x\bullet y  =0
 \]



 \proposizione{Disuguaglianza di Cauchy-Schwarz}{

     \[
         \forall x,y \in \mathbb{R}^{n}
     \]

     \[
         |\langle x,y \rangle| \le ||x||\cdot ||y||
     \]
     
     (il primo è un valore assoluto, i secondi due sono norme)

     si ha:

     \[
         x\bullet y = |x| |y|  \Leftrightarrow y = 0 \lor x=\lambda y \text{ con } \lambda \in \mathbb{R}\ge 0
     \]

 }

 \begin{proof}
     Se
     \[
        y=0
     \]

     \[
         y=0=(0, \ldots ,0) 
     \]
     
     questo caso va bene.

     Sia dunque $\mathbb{R}^{n} \rightarrow y \neq 0$ e consideriamo la funzione reale di una variabile reale $t \rightarrow |x+ty|^{2}\ge 0$ polinomio di secondo grado in t

     \[
         |x + ty| ^{2} \overset{\text{Carnot}}{=} |x|^{2} + |ty|^{2} + 2\langle x,ty \rangle  = |x|^{2} + |y|^{2}t^{2} + 2 \langle x,y \rangle t
     \]

     è un polinomio di II grado in t dove $|y|^{2}> 0 $ essendo $y \neq 0$

     Il nostro $\frac{\Delta}{4}$ deve essere non positivo:

     \[
         (x\bullet y ) ^{2} - |x| ^{2}|y|^{2} \le 0
     \]

     \[
         (x\bullet y ) ^{2}\le  |x| ^{2}|y|^{2} 
     \]

     da cui si ha la tesi.

     Si verifica, se si ha che

     \[
         \langle x,y \rangle = |x| |y|
     \]

     si ha che il $\Delta$ del trinomio di II grado è nullo e dunque $t \in \mathbb{R}$ per cui $|x+ty|^{2}=0$ ovvero $x +ty=0$ $\rightarrow x=-ty$

     devo mostrare che $-t \ge 0$


     \[
         t = - \frac{\langle x,y \rangle}{|y|^{2}}
     \]

     si ricorda che $|y|>0$ essendo y non nullo

     \[
         -t = \frac{|x| |y|}{|y|^{2}}\ge 0
     \]

 \end{proof}


 Definiamo ora la funzione \textbf{lunghezza} che è una norma $\mathbb{R}^{n}\rightarrow \mathbb{R}_0^{+}$

 C'è una proprietà che è quella di omogeneità:

 \[
     |\lambda x| = \underbrace{|\lambda|}_\text{valore assoluto} |x| 
 \]

 \[
 \forall \lambda \in \mathbb{R},x \in \mathbb{R}^{n}    
 \]

 e anche 

 \defn{Disuguaglianza triangolare}{
 La disuguaglianza triangolare si definisce come:

 \[
     |x+y| \le |x|+|y|
 \]
   
se $|x+y| = |x| + |y|$ $\rightarrow $ $y =0$ $\lor$ $x= \lambda y $ con $\lambda \ge 0$:

 }

 

\begin{proof}
       Dimostriamo la disuguaglianza triangolare, considero:

       \[
           |x+y|^{2} = \langle x+y , x+y \rangle = |x|^{2}+|y|^{2} + 2 \langle x,y \rangle \le 
       \]

       \[
           \le |x|^{2} + |y| ^{2} + 2|\langle x,y \rangle| \le \underbrace{|x| ^{2} + |y|^{2} + 2|x|\bullet |y|}_{(|x|+|y|)^{2}}
       \]

       estraendo e passando alle radici si ha

       \[
           |x+y| \le |x|+|y|
       \]
\end{proof}



\defn{Distanza Euclidea}{
Distanza Euclidea si definisce come $d(x,y)$:

\[
    d(x,y) := |x-y| = \sqrt{\langle x-y,x-y \rangle} = \sqrt{\sum^{n}_{i=1} (x_i - y_i)^{2}} \ge 0
\]

questa è la norma

}


\end{document}
