\documentclass[../appunti-analisi.tex]{subfiles}

\begin{document}

\section{Lezione 12}

\subsection{Rappresentazioni di funzioni}

Per poter rappresentare le funzioni ci serviamo delle \textbf{sezioni} che ci permettono di usare dei sottoinsiemi per le rappresentazioni e quindi di semplificare.

\textbf{Esempio} 

\[
    f(x,y)=x^{2}-y
\]

il suo grafico sarà:

\[
    \{(x,y,z) \in \mathbb{R}^{3},z=x^{2}-y\}
\]

Provo a sezionarlo con $x=k$.

Nel piano $(y,z)$ la sezione è data dal grafico di:

\[
    z= k^{2}-y
\]

questo sarà rappresentato da un fascio di rette a $z=-y$:

\begin{figure}[ht]
    \centering
    \incfig{grafico-sezione-esempio}
    \caption{grafico-sezione-esempio}
    \label{fig:grafico-sezione-esempio}
\end{figure}


\subsection{Insiemi di livello (curva di livello)}

Nel caso in cui $z=k$.

\textbf{Esempio} 

Consideriamo i punti $(x,y)$ che stanno nell'insieme di livello $k=1$ della funzione $f(x,y)=x^{2}+y^{2}$:

\[
    E_k = \{(x,y) \in \mathbb{R}^{2},x^{2}+y^{2}=k\}
\]

sono i punti che stanno sulla circonferenza di centro $(0,0)$ di raggio 1:

\[
    \{(x,y) \in \mathbb{R}^{2},x^{2}+y^{2}=1\} = E_1
\]


\begin{figure}[ht]
    \centering
    \incfig{curva-di-livello-esempio}
    \caption{curva di livello esempio}
    \label{fig:curva-di-livello-esempio}
\end{figure}

Vediamo qualche altro esempio.

\textbf{Esempio} 

Determinare l'insieme di definizione delle seguenti funzioni:

\textbf{1} 

\[
    z=\sqrt{1-x^{2}-y^{2}}
\]


\begin{figure}[ht]
    \centering
    \incfig{disegno-curva-1}
    \caption{curva di livello 1}
    \label{fig:disegno-curva-1}
\end{figure}

\textbf{2} 

\[
    z=\sqrt{1-x^{2}}+\sqrt{1-y^{2}}
\]

devo imporre il dominio:

\[
    \{(x,y) \in \mathbb{R}^{2},x^{2}\le 1,y^{2}\le 1\}
\]


\begin{figure}[ht]
    \centering
    \incfig{curva-di-livello-2}
    \caption{curva di livello 2}
    \label{fig:curva-di-livello-2}
\end{figure}


\textbf{3} 

\[
    z= \frac{1}{\sqrt{y-\sqrt{x}}}
\]

\[
    D=\{(x,y) \in \mathbb{R}^{2} y-\sqrt{x}>0,x \ge 0\}
\]


\begin{figure}[ht]
    \centering
    \incfig{curva-di-livello-3}
    \caption{curva di livello 3}
    \label{fig:curva-di-livello-3}
\end{figure}

\end{document}
