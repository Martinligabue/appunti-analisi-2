\documentclass[11pt]{article}


\usepackage[italian]{babel}
\usepackage[utf8]{inputenc}	% Para caracteres en español
\usepackage{amsmath,amsthm,amsfonts,amssymb,amscd}
\usepackage{multirow,booktabs}
\usepackage[table]{xcolor}
\usepackage{fullpage}
\usepackage{lastpage}
\usepackage{enumitem}
\usepackage{fancyhdr}
\usepackage{mathrsfs}
\usepackage{wrapfig}
\usepackage{graphicx}
\graphicspath{ {./images/} }
\usepackage{setspace}
\usepackage{calc}
\usepackage{multicol}
\usepackage{bookmark}
\usepackage[normalem]{ulem}
\usepackage{cancel}
\usepackage[margin=3cm]{geometry}
\usepackage{amsmath}
\newlength{\tabcont}
\setlength{\parindent}{0.0in}
\setlength{\parskip}{0.05in}
\usepackage{empheq}
\usepackage{framed}
\usepackage[most]{tcolorbox}
\usepackage{xcolor}
\usepackage{FiraSans}
\usepackage{esint}
\usepackage{listings}
\usetikzlibrary{shadings,intersections,calc,trees,positioning,arrows,fit,shapes}
\usepackage{centernot}

\usepackage{import}
\usepackage{pdfpages}
\usepackage{transparent}
\usepackage{mathrsfs}
\usetikzlibrary{arrows}


\usepackage{pgfplots}
\pgfplotsset{compat=1.18}
\usepackage[framemethod=TikZ]{mdframed}
\usepackage{hyperref}
\hypersetup{
    colorlinks,
    citecolor=black,
    filecolor=black,
    linkcolor=black,
    urlcolor=black
}

\usepackage{subfiles}

\geometry{margin=1in, headsep=0.25in}
\setcounter{secnumdepth}{0}

\newlength{\totbarheight}

\newlength{\bardepth}

%%%%%%%%%%%%%%%%%%%%%%%%%%%%%%
% Theorem, Corollari ecc.
%%%%%%%%%%%%%%%%%%%%%%%%%%%%%%

\mdfsetup{skipabove=\topskip,skipbelow=\topskip}
%%% upto here
\newcounter{theo}[section]
\newenvironment{theo}[1][]{%
\stepcounter{theo}%
\ifstrempty{#1}%
 {\mdfsetup{%
   frametitle={%
    \tikz[baseline=(current bounding box.east),outer sep=0pt]
    \node[anchor=east,rectangle,fill=blue!20]
         {\strut Teorema~\thetheo};}}
 }%
{\mdfsetup{%
  frametitle={%
   \tikz[baseline=(current bounding box.east),outer sep=0pt]
   \node[anchor=east,rectangle,fill=blue!20]
        {\strut Teorema~\thetheo:~#1};}}%
 }%
\mdfsetup{innertopmargin=10pt,linecolor=blue!20,%
       linewidth=2pt,topline=true,
       frametitleaboveskip=\dimexpr-\ht\strutbox\relax,}
   \begin{mdframed}[]\relax%
}
{\end{mdframed}}

\newcounter{lemm}[section]
\newenvironment{lemm}[1][]{%
\stepcounter{lemm}%
\ifstrempty{#1}%
 {\mdfsetup{%
   frametitle={%
    \tikz[baseline=(current bounding box.east),outer sep=0pt]
    \node[anchor=east,rectangle,fill=blue!20]
         {\strut Lemma~\thelemm};}}
 }%
{\mdfsetup{%
  frametitle={%
   \tikz[baseline=(current bounding box.east),outer sep=0pt]
   \node[anchor=east,rectangle,fill=blue!20]
        {\strut Lemma~\thelemm:~#1};}}%
 }%
\mdfsetup{innertopmargin=10pt,linecolor=blue!20,%
       linewidth=2pt,topline=true,
       frametitleaboveskip=\dimexpr-\ht\strutbox\relax,}
   \begin{mdframed}[]\relax%
}
{\end{mdframed}}


\newcounter{corollario}[section]
\newenvironment{corollario}[1][]{%
\stepcounter{corollario}%
\ifstrempty{#1}%
 {\mdfsetup{%
   frametitle={%
    \tikz[baseline=(current bounding box.east),outer sep=0pt]
    \node[anchor=east,rectangle,fill=blue!20]
         {\strut Corollario~\thecorollario};}}
 }%
{\mdfsetup{%
  frametitle={%
   \tikz[baseline=(current bounding box.east),outer sep=0pt]
   \node[anchor=east,rectangle,fill=blue!20]
        {\strut Corollario~\thecorollario:~#1};}}%
 }%
\mdfsetup{innertopmargin=10pt,linecolor=blue!20,%
       linewidth=2pt,topline=true,
       frametitleaboveskip=\dimexpr-\ht\strutbox\relax,}
   \begin{mdframed}[]\relax%
}
{\end{mdframed}}

\newcounter{definizione}[section]
\newenvironment{definizione}[1][]{%
\stepcounter{definizione}%
\ifstrempty{#1}%
 {\mdfsetup{%
   frametitle={%
    \tikz[baseline=(current bounding box.east),outer sep=0pt]
    \node[anchor=east,rectangle,fill=blue!20]
         {\strut Definizione~\thedefinizione};}}
 }%
{\mdfsetup{%
  frametitle={%
   \tikz[baseline=(current bounding box.east),outer sep=0pt]
   \node[anchor=east,rectangle,fill=blue!20]
        {\strut Definizione:~#1};}}%
 }%
\mdfsetup{innertopmargin=10pt,linecolor=blue!20,%
       linewidth=2pt,topline=true,
       frametitleaboveskip=\dimexpr-\ht\strutbox\relax,}
   \begin{mdframed}[]\relax%
}
{\end{mdframed}}

\newcounter{propo}[section]
\newenvironment{propo}[1][]{%
\stepcounter{propo}%
\ifstrempty{#1}%
 {\mdfsetup{%
   frametitle={%
    \tikz[baseline=(current bounding box.east),outer sep=0pt]
    \node[anchor=east,rectangle,fill=blue!20]
         {\strut Proposizione~\thepropo};}}
 }%
{\mdfsetup{%
  frametitle={%
   \tikz[baseline=(current bounding box.east),outer sep=0pt]
   \node[anchor=east,rectangle,fill=blue!20]
        {\strut Proposizione~\thepropo~ - #1};}}%
 }%
\mdfsetup{innertopmargin=10pt,linecolor=blue!20,%
       linewidth=2pt,topline=true,
       frametitleaboveskip=\dimexpr-\ht\strutbox\relax,}
   \begin{mdframed}[]\relax%
}
{\end{mdframed}}

\newcounter{ese}[section]
\newenvironment{ese}[1][]{%
\stepcounter{ese}%
\ifstrempty{#1}%
 {\mdfsetup{%
   frametitle={%
    \tikz[baseline=(current bounding box.east),outer sep=0pt]
    \node[anchor=east,rectangle,fill=blue!20]
         {\strut Esempio~\theese};}}
 }%
{\mdfsetup{%
  frametitle={%
   \tikz[baseline=(current bounding box.east),outer sep=0pt]
   \node[anchor=east,rectangle,fill=blue!20]
        {\strut Esempio~\theese~#1};}}%
 }%
\mdfsetup{innertopmargin=10pt,linecolor=blue!20,%
       linewidth=2pt,topline=true,
       frametitleaboveskip=\dimexpr-\ht\strutbox\relax,}
   \begin{mdframed}[]\relax%
}
{\end{mdframed}}

%%%%%%%%%%%%%%%%%%%%%%%%%%%%%%
% SELF MADE COMMANDS
%%%%%%%%%%%%%%%%%%%%%%%%%%%%%%


\newcommand*\Eval[3]{\left.#1\right\rvert_{#2}^{#3}}
\newcommand{\teorema}[2]{\begin{theo}[#1]{}{}#2\end{theo}}
\newcommand{\cor}[2]{\begin{corollario}[#1]{}{}#2\end{corollario}}
\newcommand{\lemma}[2]{\begin{lem}[#1]{}{}#2\end{lem}}
\newcommand{\proposizione}[2]{\begin{propo}[#1]{}{}#2\end{propo}}
% \newcommand{\mprop}[2]{\begin{Prop}{#1}{}#2\end{Prop}}
% \newcommand{\clm}[3]{\begin{claim}{#1}{#2}#3\end{claim}}
% \newcommand{\wc}[2]{\begin{wconc}{#1}{}\setlength{\parindent}{1cm}#2\end{wconc}}
% \newcommand{\thmcon}[1]{\begin{Theoremcon}{#1}\end{Theoremcon}}
% \newcommand{\ex}[2]{\begin{Example}{#1}{}#2\end{Example}}
\newcommand{\defn}[2]{\begin{definizione}[#1]{}{}#2\end{definizione}}
\newcommand{\esempio}[2]{\begin{ese}[#1]{}{}#2\end{ese}}
% \newcommand{\dfnc}[2]{\begin{definition}[colbacktitle=red!75!black]{#1}{}#2\end{definition}}
% \newcommand{\qs}[2]{\begin{question}{#1}{}#2\end{question}}
% \newcommand{\pf}[2]{\begin{myproof}[#1]#2\end{myproof}}
% \newcommand{\nt}[1]{\begin{note}#1\end{note}}


\newcommand{\incfig}[2][1]{%
    \def\svgwidth{#1\columnwidth}
    \import{\subfix{./figures/}}{#2.pdf_tex}
}


\newcommand\DrawBlock[3]{
\ifx#1b\relax
  \path[draw]
    (lm\the\numexpr#2-1\relax) -- ++(0,0,#3) coordinate (blocklf)
    (bm\the\numexpr#2-1\relax) -- ++(0,0,#3) coordinate (blocklb)
    (lm#2) -- ++(0,0,#3) coordinate (blockrf)
    (bm#2) -- ++(0,0,#3) coordinate (blockrb);
  \filldraw[fill=white,draw=black]
    (lm\the\numexpr#2-1\relax) -- (blocklf) -- (blocklb) -- (blockrb) -- (blockrf) -- (lm#2);
\else  
  \ifx#1f\relax
    \path[draw]
      (fm\the\numexpr#2-1\relax) -- ++(0,0,#3) coordinate (blocklf)
      (lm\the\numexpr#2-1\relax) -- ++(0,0,#3) coordinate (blocklb)
      (fm#2) -- ++(0,0,#3) coordinate (blockrf)
      (lm#2) -- ++(0,0,#3) coordinate (blockrb);
    \filldraw[fill=white,draw=black]
      (fm\the\numexpr#2-1\relax) -- (blocklf) -- (blocklb) -- (blockrb) -- (blockrf) -- (fm#2);
  \fi
\fi
\draw (blocklf) -- (blockrf);
}

\pdfsuppresswarningpagegroup=1


\begin{document}

\title{Analisi 2}
\author{Guglielmo Bartelloni}

\maketitle
\tableofcontents
\newpage
\thispagestyle{empty}


\section{Lezione 1}

\subsection{Equazioni differenziali}

Le equazioni differenziali sono equazioni in cui l'incognita è un equazione insieme a qualche sua derivata.

\subsubsection{Equazioni differenziali ordinarie}

Noi vedremo quelle del primo ordine lineari e di secondo ordine con coefficienti costanti

Problema di Cauchy: problema con condizioni iniziali.


\defn{}{Una equazione di ordine n è una equazione del tipo:
\[
    F(x,y(x),y'(x),...,y^{(n-1)}(x),y^{(n)}(x))=0
\]
\[
    x \in I \subseteq \mathbb{R}
\]
dove l'incognita è la qualunque y(x). F è funzione di (n+2) variabili $x,y(x),y'(x)....$
}



L'\textbf{ordine} è dato dal massimo ordine di derivazione che compare.


Per esempio:
\[
    y'''+2y''+5y = e^x
\]
è di ordine 3

\defn{Soluzione (curva) integrale}{La soluzione di una EDO di ordine n sull'intervallo I \[
        (*) F(x,y(x),y'(x),...) = 0
\]
\[
    x \in I \subseteq \mathbb{R}
\]

$\varphi(x)$ che sia definita (almeno) in I e ivi derivabile fino all'ordine n per cui valga (*), ovvero:
\[
    F(x,\varphi(x),\varphi ' (x), ... ) = 0 
\]

$\forall x \in I$

Chiaramente cambia a seconda dell'intervallo
}

\defn{Integrale Generale}{Si chiama integrale \textbf{generale} di (*) in I l'insieme di tutte le soluzioni di (*) in I}


È possibile definire un espressione più esplicita

\defn{Forma normale}{Una Equazione Differenziale Ordinaria (EDO) di ordine n si dice in forma normale se è in forma

    \[
        y^{(n)} = f(x,y(x),y'(x),....,y^{(n-1)}), x \in I
    \]
    
    Esempio:
    \[
        y'''=-5y'+sinx
    \]
    Quella sopra è un EDO di III ordine normale.
}

\defn{EDO di ordine n lineare}{Una EDO di ordine n si dice lineare se è nella forma
    \[
        a_n(x)y^{(n)}(x)+a_{n-1}(x)y^{(n-1)}+... + a_2(x)y''(x)+a_1(x)y'(x)+a_0y(x)=f(x),x \in I
    \]

    Dove le funzioni \[
        a_0(x),a_1(x),a_2(x),...,a_n(x),f(x)
    \]

    sono assegnate (continue) in I

    Esempio:
    \[
        xy''+5y = sin x
    \]

}


Quando $f(x)=0$ allora l'equazione si dice l'\textbf{omogenea associata} 


Nel nostro caso le equazioni di secondo ordine lineari saranno a \textbf{coefficienti costanti}


Vediamo come si risolve il problema della determinazione delle soluzioni di EDO lineari


\subsubsection{I ordine (n=1)}

\[
    F(x,y(x),y'(x))=0
\]

La considero in forma normale:
\[
    (1)\ y'(x)+a(x)y(x)=f(x), x \in [a,b]
\]

dove le funzioni $a(x)$ e $f(x)$ sono continue in $[a,b]$


Se $f(x)=0$ si ottiene omogenea associata:
\[
    (2)\ y'(x)+a(x)y(x)=0
\]

Come si determina l'integrale generale di (1)?

Il teorema che enunciamo vale per tutte le EDO lineari di ordine n

\teorema{}{L'integrale generale di (1) in $[a,b]$ è dato dalla somma dell'integrale generale dell'omogenea associata (2)
con un integrale particolare noto di (1)
\[
\int_{{}}^{{}} {gen} (1) = \int_{{}}^{{}} {gen(2)}  + \int_{{}}^{{}} {particolare} (1)
\]
}

\begin{proof}
    



Sia $y(x)$ una soluzione qualsiasi di (1) ($y(x)$ appartiene all'integrale generale di (1))
e sia $\bar y(x)$ una soluzione particolare (nota) di (1). Voglio far vedere è che la loro differenza è una soluzione qualsiasi di (2)

Dunque per ipotesi n ha che:
\[
    y'(x)+a(x)y(x) = f(x), \forall x \in [a,b]
\]

\[
    \bar y'(x) + a(x) \bar y(x) = f(x)
\]

Entrambe soddisfano la (1)

Sottraggo membro a membro le due:

\[
    y'(x)-\bar y'(x) + a(x)y(x) - a(x) \bar y(x) = f(x) - f(x)
\]

\[
    y'(x)-\bar y'(x) + a(x)[y(x) - \bar y(x)]=0
\]

Si può scrivere anche (le derivate raccolte):
\[
    [y(x)-\bar y(x)]' + a(x)[y(x) - \bar y(x)]=0
\]

E dunque  la funzione $y(x) - \bar y(x) = z(x)$ è soluzione di (2)
Quindi:
\[
    y(x) = \bar y(x) + z(x)
\]

Viceversa se $z(x)$ è una qualsiasi soluzione di (2) e $\bar y(x)$ è una soluzione particolare di (1) 
voglio mostrare che la loro somma è soluzione di (1)

Pongo:
\[
    y(x) = z(x) + \bar y(x)
\]

Devo mostrare che $y(x)$ verifica (1)

sapendo che:
\[
    z'(x) + a(x)z(x) = 0
\]

\[
    \bar y'(x) + a(x) \bar y(x) = f(x)
\]

\[
    y'(x) = (z(x) + \bar y(x) )' = z'(x) + \bar y'(x) =
    -a(x)z(x)-a(x)\bar y(x) + f(x) = -a(x) [z(x) + \bar y(x)] +f(x)
\]

E quindi ho dimostrato che:
\[
    y'(x) = -a(x)y(x) + f(x)
\]

\[
    y'(x) +  a(x)y(x) = f(x)
\]

\[
    y(x) = z(x) + \bar y(x)
\]

\end{proof}

\newpage

\section{Lezione 2}
\subsection{Facciamo vedere che il teorema precedente valeva anche per $n>1$}

Supponiamo che $u$ e $v$ siamo due soluzioni di (1), cioè che:

$Lu=f$ e $Lv=f$ su $I$

La differenza di queste diventano soluzione su $I=[a,b]$ dell'omogenea associata

Usando la proprietà della linearità:

\[
    L(\lambda u+\mu v) = \lambda L u + \mu L v
\]

\[
    L(u-v) = Lu-Lv = f- f=0
\]

Se indichiamo con $V_0$ l'insieme di tutte le soluzioni dell'equazione omogenea associata ($Lw=0$ su $I=[a,b]$ e $V_0$ è l'insieme delle $w \in \mathbb{C}^n(I)$) e con $\bar u(t)$ una soluzione nota di (1)

\[
    u(x) = \bar u(x) +w(x)
\]

L'uguaglianza sopra, al variare di $w(x)$ in $V_0$ ci dà tutte le soluzioni del problema di partenza. 

(Il problema quindi, diventa solo di studiare il problema omogeneo)

\subsection{Torniamo al I ordine}

Adesso ritorniamo al problema di I ordine (in forma normale):

\[
    (1)\ y'(x)+a(x)y(x)=f(x)
\]

dove $a()$ e $f()$ sono continue su $[a,b]$

\[
    (2)\ y'(x)+a(x)y(x)=0
\]

Secondo il teorema della prima lezione:

\[
    y(x)=z(x)+\bar y(x)
\]

Come si determina l'insieme di tutte le soluzioni (integrale generale) di (2), cioè:

\[
    (2)\ y'(x)+a(x)y(x)=0,x \in [a,b]
\]

Sia $A(x)$ una \textbf{primitiva} di $a(x)$:

\[
    A(x) = \int_{{}}^{{}} {a(x)} \: d{x} {}
\]

Moltiplichiamo i due membri della (2) per $e^{A(x)}$:

\[
    e^{A(x)} y'(x) + e ^{A(x)}a(x) y(x)=0, x \in [a,b]
\]

La posso scrivere anche (la derivata di $e ^{A(x)}y(x)$):

\[
    (e ^{A(x)} y(x))' = e ^{A(x)}a(x)y(x) + e ^{A(x)}y'(x)
\]

quindi (sempre chiaramente nell'intervallo $[a,b]$):

\[
    (e ^{A(x)}y(x))'=0
\]

Questo mi dice che:

\[
    e ^{A(x)}y(x) = costante=c \in \mathbb{R}
\]

porto dall'altra parte:

\[
    y(x) = c e ^{-A(x)}
\]

espandendo $A(x)$:

\[
    y(x) = c e ^{\int_{{}}^{{}} {a(x)} \: d{x} {}}
\]

posso considerare le soluzioni come:

\[
    y(x) = c z_0
\]

dove $z_0$ è una soluzione particolare di (2).

Infatti $e ^{-A(x)}$ è soluzione di (2)

\begin{proof}
    \[
    e ^{-A(x)} = -a(x) e ^{-A(x)}
\]

ovvero

\[
    (e ^{-A(x)})'+a(x) e ^{-A(x)}=0
\]
    
\end{proof}


\subsubsection{Determinazione dell'integrale particolare}

Sappiamo:

\[
    (1)\ y'(x)+a(x)y(x)=f(x)
\]

\[
    (2)\ y'(x)+a(x)y(x)=0
\]

Cerco l'integrale particolare a \textbf{occhio} oppure uso il \textbf{metodo della variazione della costante}

\paragraph{Metodo della variazione della costante}

Cerco questa $c(x)$ in questa forma:

\[
    \bar y(x) = c(x) e ^{-A(x)}
\]

Ovviamente la cerco dopo che so che $\bar y(x)$ è soluzione del problema.

\begin{proof}
    Poiché $\bar y(x)$ è soluzione di (1) si ha che $\bar y'(x)+a(x) \bar y(x)=f(x)$ da cui sostituendo $\bar y(x) = c(x) e ^{-A(x)}$:

    \[
        (c(x) e ^{-A(x)})'+ a(x) c(x) e ^{-A(x)} = f(x)
    \]
    
    Deriviamo:

    \[
        c'(x) e ^{-A(x)} - c(x) a(x) e ^{-A(x)} + a(x) c(x) e ^{-A(x)}= f(x)
    \]
    
    semplifico

    \[
        c'(x) e ^{-A(x)} = f(x)
    \]

    \[
        c'(x) = f(x) e ^{A(x)} \rightarrow c(x) = \int_{{}}^{{}} {f(x) e ^{A(x)}} \: d{} {}
    \]

    e dunque:

    \[
        \bar y(x) = e ^{-A(x)} \int_{{}}^{{}} {f(x) e ^{A(x)}} \: d{x} {}
    \]

    Cioè l'integrale particolare
\end{proof}

Se metto tutto insieme l'integrale generale diventa:

\[
    y(x) = c e ^{-A(x)} + e ^{-A(x)} \int_{{}}^{{}} {f(x) e ^{A(x)}} \: d{x} {}
\]

\subsubsection{Osservazioni sulla formula}

$A(x)$ è \textbf{una} primitiva di $a(x)$ scelta una volta per tutte.

\textbf{Non} occorre mettere una costante arbitraria (ovvero considerare come $A(x) + K,K \in \mathbb{R}$ ) poiché l'integrale generale non cambia

\textbf{Non} serve neanche nell'integrale perché verrebbe buttato dentro $c$ dell'integrale generale


\subsubsection{Esempi}

\[
    y'(x) = 5y(x) + e ^{x}
\]

in questo caso $a(x) = -5$

\[
    A(x) = - \int_{{}}^{{}} {5} \: d{x} {}=-5x
\]

Quindi: 

\[
    e ^{-A(x)}=e ^{5x}
\]

\[
    y(x) = c e ^{5x} + e ^{5x} \int_{{}}^{{}} {e^x e ^{-5x}} \: d{x} {} = c e ^{5x} + e ^{5x} \int_{{}}^{{}} {e ^{-4x}} \: d{} {} = c e ^{5x} + e ^{5x} (\frac{1}{4} e ^{-4x}) = c e ^{5x} - \frac{1}{4} e ^{x}
\]

Esercizio per casa:

\[
    u' + \frac{u}{t} = e ^{t}
\]

\newpage

\section{Lezione 3}

Solitamente si suppongono delle condizioni iniziali nel risolvere le equazioni differenziali (problema di Cauchy).

\begin{equation}
    \begin{cases}
      y'(x)+a(x)y(x)=f(x)\\
      y(x_0)=y_0
    \end{cases}\,.
\end{equation}

Praticamente gli integrali della formula generale diventano definiti tra $x_0$ e $x$.

Quindi:

\[
    y(x) = ce ^{-A(x)}+ e ^{-A(x)} \int_{{}}^{{}} {e ^{A(x)}f(x)} \: d{x} {} = c e ^{-\int_{{x_0}}^{{x}} {a(t)} \: d{t} {}}+e ^{- \int_{{x_0}}^{{x}} {a(t)} \: d{t} {}}\int_{{x_0}}^{{x}} {e ^{\int_{{x_0}}^{{s}} {a(t)} \: d{t} {}}f(s)} \: d{s} {}\\
\]

\[
    y(x_0)=y_0=c
\]


Voglio trovare la soluzione generale in questo caso, parto dall'omogenea:

\[
    y'+x(x)y(x) = 0
\]

\[
    e ^{\int_{{x_0}}^{{x}} {a(x)} \: d{t} {}} = e ^{A(x)}
\]


\subsection{Il problema di Cauchy}

Quindi introduciamo il problema di Cauchy:

\begin{equation}
    \begin{cases}
      y'+a(x)y = f(x)\\
      y(x_0) = y_0
    \end{cases}\,.
\end{equation}

dove $x \in I = [a,b]$ e $x_0 \in I$

con le ipotesi fatte ($a(x)$ e $f(x)$ continue in I) ha una e una sola soluzione (SOLUZIONE UNICA)

con l'espressione esplicita determinata.

\textbf{Esempio 1}

Determinare la soluzione del problema di Cauchy:

\begin{equation}
    \begin{cases}
      y'(x)=5y(x) + e ^{x}\\
      y(0)=0
    \end{cases}\,.
\end{equation}

\[
    A(x)=\int_{{0}}^{{x}} {a(t)} \: d{t} {}= - \int_{{0}}^{{x}} {5} \: d{t} {}= -5x
\]

\[
    y(x)=0e ^{5x} + e ^{5x}\int_{{0}}^{{x}} {e ^{-5t}e ^{t}} \: d{t} {}=
\]

\[
    = e ^{5x} \Eval{[ -\frac{1}{4} e ^{-4t}]}{0}{x} = e ^{5x}(-\frac{1}{4} e ^{-4x}+\frac{1}{4})= - \frac{1}{4} e ^{x}+ \frac{1}{4} e ^{5x}
\]

\textbf{Esempio 2}

Determinare l'integrale generale della EDO:

\[
    y'+\frac{1}{\sqrt{x}} y=1
\]

e trovare le eventuali soluzioni tali che:

\[
    \lim_{x \to \infty} y(x) = +\infty
\]

Soluzione: 

l'equazione è definita per ogni $x>0$

\[
    a(x) = \frac{1}{\sqrt{x}} 
\]

\[
    A(x) = \int_{{}}^{{}} {\frac{1}{\sqrt{x}} } \: d{x} {}
\]


L'integrale generale:

\[
    y(x) = c e ^{-\int_{{}}^{{}} {\frac{1}{\sqrt{x}} } \: d{x} {}}+ e ^{-\int_{{}}^{{}} {\frac{1}{\sqrt{x}} } \: d{x} {}}( \int_{{}}^{{}} {e ^{\int_{{\frac{1}{\sqrt{x}} }}^{{}} {} \: d{x} {+1}}} \: d{x} {})=
\]

\[
    =e ^{2 \sqrt{x}} (e+ \int_{{}}^{{}} {e ^{2\sqrt{x}}} \: d{x} {})
\]

Risolvo l'integrale pongo $t = 2 \sqrt{x}$ quindi $ dt = \frac{1}{\sqrt{x}}dx \rightarrow dx = \frac{t}{2} dt $:

\[
    \int_{{}}^{{}} {e ^{2 \sqrt{x}}} \: d{x} {}= \int_{{}}^{{}} {e ^{t}\frac{t}{2} } \: d{t} {} = e ^{x}\frac{t}{2} - \int_{{}}^{{}} {e ^{t}\frac{1}{2} } \: d{t} {}=
\]

\[
    = e ^{t} \frac{t}{2} - \frac{1}{2 e ^{t}} 
    \overset{\text{risostituisco}}{=} e ^{2 \sqrt{x}} \frac{2 \sqrt{x}}{2} - \frac{1}{2} e ^{2 \sqrt{x}}
\]

Ora riscrivo l'integrale generale:

\[
    y(x) = e ^{-2 \sqrt{x}}[ c + e ^{2 \sqrt{x}}(\sqrt{x}- \frac{1}{2} )]= c e ^{-2 \sqrt{x}}+ \sqrt{x} - \frac{1}{2} 
\]

Adesso soddisfo la richiesta (quali sono le soluzioni che vanno all'infinito)

\[
    \lim_{x \to \infty} c ^{-2 \sqrt{x}} + \sqrt{x} - \frac{1}{2} = +\infty
\]

questo vale per $\forall c \in \mathbb{R}$


\textbf{Esempio 3}

\begin{equation}
    \begin{cases}
      y' + \frac{2y}{x} = \frac{1}{2} \\
      y(-1)=2
    \end{cases}\,.
\end{equation}

Considero l'intervallo dove sta il $x_0=-1$ quindi $(-\infty,0)$

\[
    A(x) = \int_{{-1}}^{{x}} {\frac{1}{t} } \: d{t} {} = \Eval{[2 log|t|]}{-1}{x} = 2 log|x| - 2 log|-1| = 2 log|x| = 
\]

per via dell'intervallo il valore assoluto viene preso col meno:

\[
    =2 log(-x)  
\]

quindi l'integrale generale:

\[
    y(x) = 2 e ^{-2log(-x)}+ e^{-2log(-x)}(\int_{{-1}}^{{x}} {e ^{2log(-t)}\frac{1}{t ^{2}} } \: d{t} {})=
\]

uso la proprietà dei logaritmi:

\[
    = 2 e ^{log \frac{1}{x ^{2}} }+ e ^{log \frac{1}{x ^{2}} }\int_{{-1}}^{{x}} {e ^{log t ^{2}}} \: d{t} {}= \frac{2}{x ^{2}} + \frac{1}{x ^{2}} \int_{{-1}}^{{x}} {1} \: d{t} {} = \frac{2}{x ^{2}} + \frac{1}{x ^{2}} \Eval{[t]}{-1}{x} 
     = \frac{2 }{x ^{2}} + \frac{1}{x ^{2}} (x+1)
\]



\newpage

\section{Lezione 4}

\subsection{Edo a variabili separabili}

Una EDO si dice a variabili separabili se è della forma:

\[
    y'(x) = f(x) g(y(x))
\]

Parte che dipende da y viene moltiplicata a quella che dipende da x.

Dove le funzioni f e g sono continue nei loro domini di definizione

Il procedimento per risolverle è il seguente:

\begin{enumerate}
    \item Si cercano le soluzioni costanti $g(y)=0$ (cioè gli zeri)

        Si determinano gli eventuali $\bar y$ reali t.c. $g(\bar y)$

        $y(x)= \bar y$ sono soluzioni singolari del problema 
        
    \item Se $y \neq \bar y$ si procede separando le variabili, ovvero dividiamo per $g(y)$
\end{enumerate}

E quindi alla fine abbiamo:

\[
    \frac{y'(x)}{g(y(x))} = f(x) \overset{\text{integro rispetto ad x}}{=} \int_{{}}^{{}} {\frac{y'(x)}{g(y(x))} } \: d{x} {} = \int_{{}}^{{}} {f(x)} \: d{x} {}
\]

Uso la sostituzione $y = y(x)$ e $dy = y'(x) dx$:

\[
    \int_{{}}^{{}} {\frac{1}{g(y)} } \: d{y} = {\int_{{}}^{{}} {f(x)} \: d{x} {}}
\]

Chiamate $G$ e $F$ una primitiva di $\frac{1}{g} $ e di $f$ rispettivamente:

\[
    G(y(x)) = F(x) + c
\]

Applico la funzione inversa di $G$ a entrambi i membri per scrivere esplicitamente la soluzione:

\[
    y(x) = G ^{-1} (F(x) + c)
\]

\textbf{Esempio}

Determinare tutte le soluzioni dell'equazione differenziale:

È non lineare

\[
    y'(x) = (1-y)(2-y)x
\]

Le prime due parentesi sono $g(y)$ il resto $f(x)$

\begin{enumerate}
    \item Trovare le soluzioni costanti

        Pongo $y'(x) = 0 $:

        \[
            (1-y)(2-y) = 0
        \]

        quindi $y=1$ e $y=2$

    \item Cerchiamo le altre soluzioni dividendo per $g(y)$

        \[
            \frac{y'(x)}{1-y(x)(2-y(x))} = x
        \]

        Quindi integro:

        \[
            \int_{{}}^{{}} {\frac{1}{(1-y)(2-y)} } \: d{y} {}= \int_{{}}^{{}} {x} \: d{x} {}
        \]

        Uso i fratti semplici per risolvere il primo membro:

        \[
            \frac{A}{1-y} + \frac{B}{2-y} = \frac{1}{(1-y)(2-y)} 
        \]

        \[
            A(2-y)+B(1-y) = 1
        \]

        \[
            (-A -B) y +2A + B = 1
        \]

        \begin{equation}
            \begin{cases}
              -A -B = 0\\
              2A+B= 1
            \end{cases}\,.
        \end{equation}

        $A=1$ e $B=1$
        
        Quindi:

        \[
            \int_{{}}^{{}} {\frac{1}{1-y}} \: d{y} {}- \int_{{}}^{{}} {\frac{1}{2-y} } \: d{y} {} = \int_{{}}^{{}} {x} \: d{x} {}
        \]

        \[
            -log|1-y| + log|2-y| = \frac{x ^{2}}{2} +c
        \]

        \[
            log|\frac{2-y}{1-y} | = \frac{x ^{2}}{2} +c
        \]

        Adesso devo esplicitare per $y$ quindi passo agli esponenziali:

        \[
            |\frac{2-y}{1-y} | = e^{(\frac{x ^{2}}{2} +c)}
        \]

        \[
            |\frac{2-y}{1-y} | = e^{(\frac{x ^{2}}{2})} e ^{c} = c_1 e ^{\frac{x ^{2}}{2} } >0
        \]

        Tolgo il valore assoluto:

        \[
            \frac{2-y}{1-y} = \pm c_1 e ^{\frac{x ^{2}}{2} }\overset{\text{usando un'altra costante}}{=}c_2 e ^{\frac{x ^{2}}{2} }
        \]

        \[
            \frac{2-y}{1-y} = c_2 e ^{\frac{x ^{2}}{2} }
        \]

        Con $c_2 \in \mathbb{R}$

        Noi vogliamo trovare la $y(x)$ (per semplicità pongo $c_2 = c$):

        \[
            \frac{2-y}{1-y} = c e ^{\frac{x ^{2}}{2} }
        \]

        Porto di la il denominatore:

        \[
            2-y = c ^{\frac{x ^{2}}{2} } (1-y)
        \]

        Porto di la le cose:

        \[
            (c e ^{\frac{x ^{2}}{2} })y = c ^{\frac{x ^{2}}{2} }-2
        \]

        E quindi le due soluzioni (quella costante e quella non) sono:

        \begin{equation}
            \begin{cases}
            y(x) = \frac{c e ^{\frac{x ^{2}}{2} }-2}{c e ^{\frac{x ^{2}}{2} }-1}   \\
            y=1
            \end{cases}\,.
        \end{equation}

\end{enumerate}

\textbf{Esercizio Problema di Cauchy}

Risolviamo ora il problema:

\begin{equation}
    \begin{cases}
      y'=(1-y)(2-y)x\\
      y(0)=3
    \end{cases}\,.
\end{equation}

e decidiamo qual è il più ampio intervallo su cui è definita la soluzione

Avendo già risolto la EDO imponiamo la condizione $y(0) = 3$:

\[
    y(0) = \frac{c-2}{c-1} = 3
\]

\[
    c-2 = 3c -3
\]

\[
    c = \frac{1}{2} 
\]

La soluzione del problema è quindi (sostituisco la c trovata all'equazione):

\[
    y(x) = \frac{\frac{1}{2} e ^{\frac{x ^{2}}{2} }-2}{\frac{1}{2} e ^{\frac{x ^{2}}{2} }-1} 
\]

\[
    y(x) = \frac{ e ^{\frac{x ^{2}}{2} }-4}{ e ^{\frac{x ^{2}}{2} }-2} 
\]

La soluzione è definita nel più ampio intervallo contenente $x_0 = 0 $ (per cui l'espressione ha senso) nel nostro caso il denominatore $\neq 0$

\[
    e ^{\frac{x ^{2}}{2} } - 2 \neq 0
\]

\[
    e ^{\frac{x ^{2}}{2} }  \neq 2
\]

\[
    x ^{2} \neq 2 log 2
\]

\[
    x \neq \pm \sqrt{2 log2}
\]

Quindi l'intervallo più ampio è quello che contiene zero ed è compreso tra le regole che abbiamo appena trovato:

\[
    0 \in (-\sqrt{2log2},+\sqrt{2log2}) 
\]

Osserviamo che la soluzione:

\[
    y(x) = \frac{ e ^{\frac{x ^{2}}{2} }-4}{ e ^{\frac{x ^{2}}{2} }-2} 
\]

è definita $\forall x \in \mathbb{R}$ con $x \neq \pm \sqrt{2log2}$


Il motivo per cui la soluzione del problema di Cauchy è definita su un intervallo si capisce bene se si pensa al significato fisico del nostro problema:

\begin{equation}
    \begin{cases}
        \text{x tempo}\\
        \text{y(x) evoluzione del sistema}\\
        \text{condizione iniziale}
    \end{cases}\,.
\end{equation}

Se partendo dall'istante iniziale ($x_0$) e procedendo in avanti o a ritroso nel tempo troviamo un istante per cui il sistema non esiste (nel caso di prima $\pm \sqrt{2log2}$) la $y(x)$ non esiste più, non ha senso domandarsi che cosa succede oltre quell'istante

Se lo vedo dal punto di vista matematico se accettassimo soluzioni definite su intervalli disgiunti non avremmo più l'unicità della soluzione (ce ne sarebbero 3 nel nostro caso e non una come volevo) perché avremmo rami distinti della funzione $y(x)$ definiti su intervalli disgiunti che non si raccordano tra di loro, dunque la condizione iniziale $y(x_0) = y_0$ non determina i valore della funzione $y(x)$ negli intervalli che non contengono l'istante iniziale $x_0$

\begin{itemize}
    \item \textbf{ Soluzione in piccolo (locale) } (è definita in un intorno di $x_0$)
    \item \textbf{ Soluzione in grande (globale) } (è definita in tutto l'intervallo)
\end{itemize}

\textbf{Esercizio per casa}

\begin{equation}
    \begin{cases}
      y'(x) = xy(x)+2x\\
      y(0) = 1
    \end{cases}\,.
\end{equation}

\textbf{Soluzione} 

Raccolgo: 

\[
    y'(x)=x(y+2)
\]

Trovo le soluzioni stazionarie:

\[
    y+2=0
\]

\[
    y=-2
\]

Trovo le altre:

\[
    \int_{}^{} {\frac{1}{y+2}} \: dy = \int_{}^{} {x} \: dx 
\]

\[
    y+2 = c e ^{ \frac{x^{2}}{2}}+c
\]

Impongo le condizioni di Cauchy e trovo c sostituendo:

\[
    y=3e ^{ \frac{x^{2}}{2}}-2
\]

Quindi la soluzione completa È:

    \begin{equation}
        \begin{cases}
            y=-2\\
            y=3e ^{ \frac{x^{2}}{2}}-2
        \end{cases}\,.
    \end{equation}



\subsection{EDO lineari del II ordine}

\[
    a_2(x)y''(x) + a_1(x) y'(x) + a_0(x) y(x) = f(x)
\]

con $a_0(),a_1(),a_2(),f()$ continue in I

In forma normale:

\[
    y''(x) + a(x) y'(x) + b(x)y(x) = f(x)
\]

se pongo $f(x) = 0$ ho la omogenea associata (2)

le sue soluzioni sono linearmente indipendenti

Se abbiamo due soluzioni $y_1$ e $y_2$ di:

\[
    a_2(x)y''(x) + a_1(x) y'(x) + a_0(x) y(x) = 0
\]

Poniamo:

\[
    y(x) = c_1 y_1(x) + c_2 y_2(x)
\]

io so che le soluzioni soddisfano l'equazione (per definizione):

\[
    a_2(x)y_1''(x) + a_1(x) y_1'(x) + a_0(x) y_1(x) = 0
\]

\[
    a_2(x)y_2''(x) + a_1(x) y_2'(x) + a_0(x) y_2(x) = 0
\]

adesso:

\[
    y(x) = c_1 y_1(x) + c_2 y_2(x)
\]

Derivo due volte:

\[
    y'(x) = c_1 y_1'(x) + c_2 y_2'(x)
\]

\[
    y''(x) = c_1 y_1''(x) + c_2 y_2''(x)
\]

\[
    a_2(x) [ c_1y_1''(x) + c_2 y_2 ''(x) ] + a_1(x)[ c_1 y_1'(x) + c_2 y_2'(x) ] + a_0(x) [ c_1 y_1(x) + c_2 y_2(x)]= 
\]

\[
    = c_1[a_2(x) y_1''(x) + a_1(x) y_1'(x) + a_0(x) y_1(x)] + c_2 [a_2(x) y_2''(x) + a_1(x) y_2'(x) + a_0(x) y_2(x)] \overset{\text{dato che è soluzione}}{=} 0
\]

\subsection{Lineare indipendenza}


$y_1(x)$ e $y_2(x)$ sono linearmente indipendenti su I se:

\[
    c_1y_1(x) +c_2y_2(x) = 0 \Leftrightarrow c_1=c_2=0
\]

\textbf{Esercizi per Casa} 

\textbf{Esercizio 1}

\[
    y(x) = ce ^{x^{2}-x}+e ^{x^{2}-x}\int_{}^{} {xe ^{x}} \: dx = c e ^{x^{2}-x}+xe ^{x^{2}}-e ^{x^{2}}
\]

Ponendo le condizioni di Cauchy:

\[
    y(0)=2
\]

La soluzione È:

\[
    2 e ^{x^{2}-x}+xe ^{x^{2}}-e ^{x^{2}}
\]

\textbf{Esercizio 2} 

\[
    y'=\sqrt[3]{x}y^{2}
\]

Una soluzione È:

\[
    y=0
\]

Le altre le trovo facendo l'integrale di:

\[
    \int_{}^{} { \frac{1}{y^{2}}} \: dy = \int_{}^{} {\sqrt[3]{x}} \: dx  
\]

quindi $y(x)$:

\[
    y(x)  = \frac{4}{3 \sqrt[3]{x^{4}}+c}
\]

impongo le condizioni e trovo c:

\[
    y(0) = \frac{-4}{0+c} = 2
\]

quindi:

\[
    c = -2
\]

ergo il la soluzione È:

\[
    y(x) = -\frac{4}{3 \sqrt[3]{x^{4}} - 2}
\]

il denominatore deve essere $\neq 0$:

\[
3 \sqrt[3]{x^{4}} - 2 \neq 0
\]

quindi:

\[
    x \neq (\frac{2}{3}^{ \frac{3}{4}})
\]

Il più ampio intervallo È:

\[
    0 \in (-\infty, \frac{2}{3}^{ \frac{3}{4}}) 
\]

\newpage

\section{Lezione 5}

Ritorniamo all'equazione del secondo ordine.

\[
    a_2(x)y''(x) + a_1(x) y'(x) + a_0(x) y(x) = f(x)
\]

con $a_0(),a_1(),a_2(),f()$ continue in I $ \in  [a,b]$

ci concentriamo nel caso in cui le $a$ sono costanti (coefficienti costanti).

L'altra volta abbiamo dimostrato che se abbiamo due soluzioni $y_1$ e $y_2$ esse sono linearmente indipendenti cioè il determinante della matrice di $y_1(x)y_2'(x)-y_2(x)y_2'(x)$ è diverso da 0 (determinante Wronskiano)

Se quindi l'equazione ha coefficienti costanti diventa:

\[
    ay''(x)+by'(x)+cy(x) = f(x)
\]

con $a,b,c \in \mathbb{R}$ con $a \neq 0$ se no non sarebbe di ordine II, $f(x)$ è continua in I

Adesso associamo il problema omogeneo:

\[
    ay''(x) + by'(x) + cy(x) = 0
\]

\textbf{Numeri complessi} 

Qui dobbiamo introdurre i numeri complessi perché ci servono per la soluzione, di solito questi sono formati da una parte reale e una parte immaginaria:

\[
    z = \alpha + i\beta
\]

$z$ può essere scritto come coppia $(\alpha,\beta)$ a $i$ assegno $i=\sqrt{-1}$

Tornando a noi vediamo il caso in cui $b=c=0$

\[
    ay''(x) = 0
\]

in I e in particolare:

\[
    y''(x) = 0, \forall x \in I
\]

\[
    y'(x) = c, c \in \mathbb{R}
\]

\[
    y(x) = c_1x+c_0,c_1,c_0 \in \mathbb{R}
\]

Questo caso è facile. Se invece $b$ e $c$ non sono contemporaneamente nulli, devo considerare la seguente equazione algebrica di secondo grado:

\[
    p(\lambda) = a \lambda^{2}+b \lambda + c =0
\]

La sua equazione associata a (2):

\[
    p(\lambda) =0 \Leftrightarrow  a \lambda^{2}+b \lambda + c =0, in\ \mathbb{C}
\]

\teorema{Teorema fondamentale dell'algebra}{
    L'equazione di II in $\mathbb{C}$ 

    \[
   a \lambda^{2}+b \lambda + c =0, in\ \mathbb{C}
    \]

    ha sempre due soluzioni in $\mathbb{C}$

}


\textbf{Proposizione} 

$y(x) = e ^{\lambda x}$ è soluzione di (2) $\Leftrightarrow $ $\lambda$ è soluzione (radice) di $p(\lambda)=0$ dell'equazione caratteristica associata a (2)

Indico con $Ly$ l'equazione $Ly= ay''+by'+cy$

\begin{proof}
    y è soluzione di (2) $\Leftrightarrow$ $Ly=0$ 

    Se considero $y(x) = e ^{\lambda x}$ 

    Devo dimostrare che:

    \[
        L(e ^{\lambda x}) = 0 \Leftrightarrow  p(\lambda) = 0
    \]

    Sostituisco a $x$ $e ^{\lambda x}$:

    \[
        L(e ^{\lambda x}) = a( e^{\lambda x})'' + b( e ^{\lambda x})' + c(e ^{\lambda x}) =
    \]

    \[
        =a \lambda ^{2} e ^{\lambda x} + b \lambda e ^{\lambda x} + c e^{\lambda x}= e ^{\lambda x}(a \lambda ^{2}+ b \lambda+ c)
    \]

    dunque

    \[
        L( e ^{\lambda x}) = 0 \Leftrightarrow a \lambda ^{2}+ b \lambda +c = 0 
    \]
           
\end{proof}


Adesso che ho dimostrato il mio problema è trovare le radici $p(\lambda) =0$ ($a \lambda ^{2} + b \lambda + c$):

Di solito le soluzioni di secondo grado si scrivono

\[
    \lambda_{1,2} = \frac{-b \pm \sqrt{b ^{2}-4 ac}}{2a}
\]
   
Le soluzioni $\lambda_1$ e $\lambda_2$ sono soluzioni di ((2) $e ^{\lambda_1x}$ e $e ^{\lambda_2x}$)

Distinguiamo tre casi per le soluzioni:

\begin{enumerate}
    \item soluzioni reali e distinte ($\Delta >0$)
    \item soluzioni reali e coincidenti ($\Delta = 0$)
    \item soluzioni complesse coniugate ($ \Delta <0$)
\end{enumerate}

1) $y_1(x) = e ^{\lambda_1x}$ e $y_2(x) = e ^{\lambda_2x}$ con $\lambda_1 e \lambda_2$ $\in \mathbb{R}$ con $\lambda_1 \neq \lambda_2$


2) $y_1(x) = e ^{\lambda x}$ e $y_2(x) = xe ^{\lambda x}$ con $\lambda = - \frac{b}{2a}=\lambda_1=\lambda_2$ $\in \mathbb{R}$ 

3) $y_1(x) = e ^{\alpha x} cos \beta x$ e $y_2(x) = e ^{\alpha x} sin \beta x$ 

questo caso corrisponde a soluzioni complesse coniugate  

\[
    \lambda_1 = \alpha- i \beta \in \mathbb{C} 
\]

\[
    \lambda_2 = \alpha+ i \beta \in \mathbb{C} 
\]

\[
    \lambda = \frac{-b \pm \sqrt{-(4ac-b^{2})}}{2a} = \frac{-b \pm \sqrt{-1(4ac - b^{2})}}{2a} \overset{\text{perche i} = \sqrt{-1}}{=} \frac{-b \pm  \sqrt{4ac -b^{2}}i}{2a} = \alpha \pm i \beta
\]

dove $\alpha = -\frac{b}{2a}$ e $\beta = \frac{\sqrt{4ac - b^{2}}}{2a} >0$


\teorema{}{L'integrale generale dell'equazione omogenea $a y''+by'+c=0$ è dato da:

    \[
        c_1 y_1(x) + c_2 y_2(x)
    \]

    al variare di $c_1,c_2 \in \mathbb{R}$ dove $y_1(x)$ e $y_2(x)$ sono definite come sopra
}

\begin{proof}
       1) $b^{2}-4ac >0$ con $\lambda_1,\lambda_2$ soluzioni dell'equazioni di $p(\lambda)=0$    

       scrivo la Wronskiana di $y_1,y_2$:
       \[
        \begin{bmatrix}
            
        e ^{\lambda_1 x} & e ^{\lambda_2 x} \\
        \lambda_1e ^{\lambda_1 x} & \lambda_2e ^{\lambda_2 x} \\
        
        \end{bmatrix}
       \]
        che è diverso da zero quindi le soluzioni sono linearmente indipendenti

        sia ora $y(x)$ una soluzione di (2):

        \[
            y(x) = e ^{\lambda_1 x}u(x)
        \]

        io devo determinare $u(x)$ per poi dimostrare che $y(x) = c_1e ^{\lambda_1 x}+c_2 e^{\lambda_2 x}$

        Poiché $y(x) = e ^{\lambda_1 x}u(x)$ è soluzione di (2) si ha derivando e sostituendo:

        \[
            a( e ^{\lambda_1 x} u(x))'' + b(e ^{\lambda_1 x}u(x))'+ c e ^{\lambda_1 x}u(x) =0
        \]

        \[
            a(\lambda_1 e ^{\lambda_1 x} u(x)+ e ^{\lambda_1 x}u'(x))' + b(\lambda_1e ^{\lambda_1 x}u(x) + e ^{\lambda_1 x}u'(x))+ c e ^{\lambda_1 x}u(x) =0
        \]

        \[
            e ^{(\lambda_1 x}[a \lambda_1 ^{2} + b \lambda_1+c)u(x)+\underbrace{(au''(x)+(2a \lambda_1 + b)u'(x))}_\text{impongo che sia zero}]=0
        \]

        estraggo solo l'ultima parentesi e impongo che sia uguale a zero perché il resto è già zero

        \[
            au''(x) + (2a \lambda_1 + b) u'(x) = 0
        \]

        divido per a:

        \[
            u''(x) +(2 \lambda_1 + \frac{b}{a}) u'(x) = 0
        \]

        sapendo che:

        \[
            a \lambda^{2} + b \lambda + c =0
        \]

        \[
             \lambda^{2} + \frac{b}{a} \lambda + \frac{c}{a} =0
        \]

        \[
            \lambda_1 + \lambda_2 = -\frac{b}{a}
        \]

        \[
            \lambda_1  \lambda_2 = \frac{c}{a}
        \]

        \[
            u''(x) + (2 \lambda_1 - \lambda_1 - \lambda_2)u'(x) = 0
        \]

        il meno per comodità:

        \[
            u''(x) - (\lambda_1 - \lambda_2)u'(x) = 0
        \]

        se adesso chiamo $u'(x)=v(x)$ e $v''(x) = u'(x)$ l'equazione diventa:

        \[
            v' -kv = 0
        \]

        Risolvendo 

        \[
            v(x) = ce ^{kx}
        \]

        \[
            v(x) = c e^{(\lambda_2 - \lambda_1)x}
        \]

        Risostituendo:

        \[
            u'(x) = c e ^{(\lambda_2- \lambda_1)x}
        \]

        Integrando:

        \[
            u(x)  = c_1 e ^{(\lambda_2 - \lambda_1)x}+c_2
        \]
        
        la nostra $y(x)$ diventa:

        \[
            y(x) = e ^{\lambda_1 x}u(x) = e ^{\lambda_1 x}( c_1 e ^{(\lambda_2 - \lambda_1)x}+c_2) = c_1 e ^{\lambda_2 x}+ c_2 e ^{\lambda_1 x}
        \]


\end{proof}

Adesso voglio per il caso 2)

\[
    \lambda_1 = \lambda_2 = \lambda = -\frac{b}{2a} \in \mathbb{R}
\]
   
\[
    p(\lambda) =0 \Leftrightarrow e ^{\lambda x} \text{È soluzione di (2)}
\]

sia quindi $y(x)$ una soluzione di (2) che scriviamo come:

\[
    y(x) = e ^{\lambda x}u(x) 
\]

Come prima si ottiene:

\[
    a(e ^{\lambda x}u(x) )''+ b(e ^{\lambda x}u(x))' + c e ^{\lambda x}u(x)=0
\]

\[
    \overbrace{e ^{\lambda x}}^{>0}(a u''(x) + \underbrace{(a \lambda^{2}+b \lambda +c )}_\text{=0} u(x) + (2a \lambda+b)u'(x))=0
\]

estraggo la parte che impongo a zero:

\[
    au''(x) + (2a \lambda+b)u'(x) = 0
\]

divido per a:

\[
 u''(x) + (2 \lambda + \frac{b}{a})u'(x) = 0
\]

sapendo che $-\frac{b}{a} = 2 \lambda$ :

\[
 u''(x) + (\cancel{2 \lambda} + \cancel{\frac{b}{a}})u'(x) = 0
\]


\[
    u'(x) = c_1
\]

\[
    u(x) = c_1 x +c_2
\]

e quindi ho la soluzione:

\[
    y(x) = e ^{\lambda x}(c_1x+c_2) = c_1x e^{\lambda x}+ c_2 e ^{\lambda x}
\]

\newpage

\section{Lezione 6}

\subsection{Determinazione della soluzione particolare per EDO II ordine}


Dobbiamo vedere ora come si determina la soluzione particolare.

\[
    (1)\ ay''+by'+cy = f(x)\ \in I=[a,b]
\]

\[
    (2)\ ay''+by'+cy = 0\ \in I=[a,b]
\]

\[
    y(t) = c_1 y_1(x) + c_2 y_2(x) + \bar{y} (x)
\]

La $\bar{y}$ è la soluzione particolare, ci sono due modi:

\begin{itemize}
    \item Si procede a occhio, per similitudine guardando l'espressione di $f(x)$
    \item Si usa il metodo di variazione delle costanti

        \[
            \bar{y} = c_1(x) y_1(x) + c_2(x) y_2(x)
        \]

        dove ${y_1(x),y_2(x)}$ soluzioni linearmente indipendenti di (2) con $c_1(x),c_2(x)$ funzioni di classe $\mathbb{C}^{2}(I)$ da determinare.
\end{itemize}

Vediamo come fare con quest'ultimo metodo.

Poiché $\bar{y} (x)$ è soluzione di (1) allora  $a\bar{y} ''+b \bar{y}'+c\bar{y}=f$

\[
    \bar{y} (x) = c_1(x) y_1(x) + c_2(x) y_2(x)
\]

\[
\bar{y} '(x) = c_1'(x) y_1(x) + c_1 y_1'(x) + c_2'(x) y_2(x) + c_2(x) y_2'(x)
\]

Adesso impongo che $c_1'(x) y_1(x) + c_2'(x) y_2(x) = 0$:

\[
    \bar{y} '(x) = c_1(x) y_1'(x) +c_2(x) y_2'(x)
\]

\[
    \bar{y} ''(x) = c_1'(x) y_1'(x) + c_1(x) y_1''(x) + c_2'(x) y_2'(x) + c_2(x) y_2''(x)
\]

sapendo che $a \bar{y} ''+ b \bar{y} ' + c \bar{y}  = f$ sostituisco quello che ho trovato sopra a questa espressione:

\[
    \textbf{a}[c_1'(x) y_1'(x) + c_1(x) y_1''(x) + c_2'(x) y_2'(x) +c_2(x) y_2''(x)]+ \textbf{b}[c_1(x) y_1'(x) + c_2(x) y_2'(x)]+ \textbf{c}[c_1(x) y_1(x) + c_2(x) y_2(x)] = f(x)
\]

Adesso raccolgo a fattore comune le $c_i$:

\[
    c_1(x) [ \underbrace{a y_1''(x) +b y_1'(x) + c y_1(x)}_\text{=0}] + c_2(x) [\underbrace{a y_2''(x) + b y_2'(x) + c y_2(x)}_\text{=0}]+ a [c_1'(x) y_1'(x)+ c_2'(x) y_2'(x)] = f(x)
\]

quindi mi rimane:

\[
    c_1'(x) y_1'(x) + c_2'(x) y_2'(x) = \frac{f(x)}{a}
\]

Ottengo il sistema di 2 equazioni nelle due incognite ($c_1'(x),c_2'(x)$) non omogeneo:

    \begin{equation}
        \begin{cases}
            c_1'(x)y_1(x) + c_2'(x) y_2(x) = 0\\
            c_1'(x) y_1'(x) + c_2'(x) y_2'(x)  = \frac{f(x)}{a}
        \end{cases}\,.
    \end{equation}

La matrice dei coefficienti del sistema È:

\[
\begin{bmatrix}
y_1(x) & y_2(x) \\
y_1'(x) & y_2'(x) \\
\end{bmatrix}
\neq 0
\]

È la matrice Wronskiana.

Uso il metodo di Cramer per risolvere il sistema:

\[
    A = \begin{pmatrix}
        a_{11} & a_{12}  \\
        a_{21} & a_{22}  \\
\end{pmatrix}
\]

$det A = a_{11} a_{22} - a_{12} a_{21}$

Il metodo:

\[
c_1'(x) = 
    \frac{
\begin{vmatrix}
0 & y_2(x)  \\
\frac{f(x)}{a} & y_2'(x)  \\
\end{vmatrix}
    }{
\begin{vmatrix}
y_1(x) & y_2(x)  \\
y_1'(x) & y_2'(x)  \\
\end{vmatrix}
    }
\]

\[
    = \frac{- y_2(x) \frac{f(x)}{a}}{y_1(x) y_2'(x) - y_2(x) y_1'(x)}
\]

\[
c_2'(x) = 
    \frac{
\begin{vmatrix}
y_1(x) & 0  \\
y_1'(x) & \frac{f(x)}{a}  \\
\end{vmatrix}
    }{
\begin{vmatrix}
y_1(x) & y_2(x)  \\
y_1'(x) & y_2'(x)  \\
\end{vmatrix}
    }
\]

\[
    = \frac{- y_1(x) \frac{f(x)}{a}}{y_1(x) y_2'(x) - y_2(x) y_1'(x)}
\]

Ora dobbiamo integrare

\[
    c_1(x) = \int_{}^{} {c_1'(x)} \: dx 
\]

e

\[
    c_2(x) = \int_{}^{} {c_2'(x)} \: dx 
\]

Si ha che l'insieme delle soluzioni di (1):

\[
    y(t)  = \underbrace{c_1 y_1(x) + c_2 y_2(x)}_\text{generale} + \underbrace{c_1(x) y_1(x) + c_2 y_2(x)}_\text{particolare}
\]

Assegnando le condizioni iniziali: 

\[
    y(x_0) = y_0
\]

\[
    y'(x_0) = y_0'
\]

Quindi il problema di Cauchy mi viene:

    \begin{equation}
        \begin{cases}
            ay''+ by'+cy= f(x)\\
            y(x_0)=y_0\\
            y'(x_0) = y_0'
        \end{cases}\,.
    \end{equation}

Trovare le soluzioni dei seguenti problemi di Cauchy

\textbf{Esempio 1} 

    \begin{equation}
        \begin{cases}
            3y'' + 5y' + 2y=3e ^{2x}\\
            y(0) = 0\\
            y'(0) = 1
        \end{cases}\,.
    \end{equation}

Scriviamo (1) e (2):

1)
\[
    3y''+5y'+2y = 3 e^{2x}
\]

2)

\[
    3y''+5y'+2y = 0
\]

Risolvo:

\[
    3 \lambda ^{2} + 5 \lambda + 2 = 0
\]

$p = 6$ $s=5= 3+2$ 

\[
    3 \lambda^{2} + 3 \lambda + 2 \lambda + 2 =0
\]

\[
    3 \lambda ( \lambda+1) + 2( \lambda + 1) =0
\]

\[
    (3 \lambda +2 ) ( \lambda +1 ) =0
\]

\[
    \lambda= -\frac{2}{3}, \lambda=-1
\]

quindi ho soluzioni linearmente indipendenti di (2):

\[
    y_1(x) = e ^{-\frac{2}{3}x}, y_2(x) = e ^{-x}
\]

Integrale generale di (1):

\[
    y_0(x) = c_1 e^{-\frac{2}{3}x} + c_2 e ^{-x},c_1,c_2 \in \mathbb{R}
\]


Una soluzione particolare di (1) è dunque:

\[
    \bar{y} (x) = A e ^{2x}
\]

Ora derivo due volte:

\[
    \bar{y} '(x) = 2A e ^{2x}
\]

\[
    \bar{y} ''(x) = 4A e ^{2x}
\]

sostituendo poi in (1):

\[
    12Ae ^{2x} + 10 A e ^{2x} + 2 A e ^{2x} = 3 e^{2x}
\]

\[
    24A e ^{2x} = 3 e ^{2x}
\]

\[
    A = \frac{1}{8}
\]

Adesso devo imporre le condizioni iniziali a queste due:

\[
    y(x) = c_1 e ^{-\frac{2}{3}x} + c_2 e ^{-x}+ \frac{1}{8}e ^{2x}
\]

\[
    y'(x) = -\frac{2}{3}c_1 e ^{-\frac{2}{3}x} - c_2 e ^{-x}+ \frac{1}{4} e ^{2x}
\]


    \begin{equation}
        \begin{cases}
            c_1+c_2+\frac{1}{8}=0\\
            -\frac{2}{3}c_1-c_2+\frac{1}{4}=1
        \end{cases}\,.
    \end{equation}

    \begin{equation}
        \begin{cases}
            c_1=c_2-\frac{1}{8}\\
            \frac{2}{3}c_2+\frac{1}{12}-c_2+\frac{1}{4}=1
        \end{cases}\,.
    \end{equation}


    \begin{equation}
        \begin{cases}
            c_1=c_2-\frac{1}{8}\\
            -\frac{1}{3}c_2= 1- \frac{1}{3}
        \end{cases}\,.
    \end{equation}

    \begin{equation}
        \begin{cases}
            c_1=c_2-\frac{1}{8}\\
            -\frac{1}{3}c_2=\frac{2}{3}
        \end{cases}\,.
    \end{equation}

    \begin{equation}
        \begin{cases}
            c_1=\frac{15}{8}\\
            c_2=-2
        \end{cases}\,.
    \end{equation}


Infine quindi la soluzione del problema:

\[
    y(x) = \frac{15}{8}e ^{-\frac{2}{3}x}- 2 e ^{-x}+ \frac{1}{8}e ^{2x}
\]


\textbf{Esempio 2} 

    \begin{equation}
        \begin{cases}
    y''(x) -2y'(x)+ y(x) = 5sinx \\
            y(0) = 0\\
            y'(0) = 1
        \end{cases}\,.
    \end{equation}


Risolvo l'equazione associata:

\[
    \lambda^{2}-2 \lambda + 1=0
\]

\[
    (\lambda-1)^{2}=0
\]

quindi $\lambda_1=\lambda_2=1$


Devo trovare:

\[
    y_0(x) = c_1 e ^{x} + c_2 x e ^{x},c_1,c_2 \in \mathbb{R}
\]

la soluzione particolare è per essere sicuri di prendere il termine noto che è un seno:

\[
    \bar{y} (x) = A cosx + B sinx
\]

le derivate:

\[
    \bar{y} '(x)  = -A sinx + B cosx
\]

\[
    \bar{y} ''(x)  = -A cosx - B sinx
\]
    

Sostituiamo a quella iniziale:

\[
    \cancel{-A cosx} \cancel{- Bsinx} + 2A sinx - 2B cosx+ \cancel{A cosx} + \cancel{Bsinx} = 5sinx
\]

e quindi $2A = 5$ e $B=0$:

\[
    \bar{y} (x) = \frac{5}{2} cosx
\]

adesso devo trovare:

\[
    y(x) = c_1 e ^{x}+ c_2 x e ^{x} + \frac{5}{2} cosx
\]

\[
    y'(x)  = c_1 e ^{x} + c_2 e ^{x} + c_2x e ^{x} - \frac{5}{2}sinx
\]

Adesso impongo le condizioni iniziali:

    \begin{equation}
        \begin{cases}
            c_1 +\frac{5}{2}=0\\
            c_1+c_2= 1
        \end{cases}\,.
    \end{equation}

    \begin{equation}
        \begin{cases}
            c_1 = -\frac{5}{2}\\
            c_2 = \frac{7}{2}
        \end{cases}\,.
    \end{equation}

La soluzione del problema quindi:

\[
    y(x) = -\frac{5}{2}e ^{x}+ \frac{7}{2} x e ^{x} + \frac{5}{2} cosx
\]

\textbf{Esempio 3} 

    \begin{equation}
        \begin{cases}
            y''+ 2y'+ 2y= 3x^{2}\\
            y(0) = 0\\
            y'(0) = 1
        \end{cases}\,.
    \end{equation}


risolvo:

\[
    \lambda^{2}+2 \lambda+2 =0
\]

\[
    \lambda= \frac{-1 \pm \sqrt{1-2}}{1}= -1 \pm 1
\]

le soluzioni sono complesse:

\[
    \alpha \pm  i \beta
\]

con $\alpha = -1$ e $\beta = 1$ quindi sostituisco:

\[
    y_0(x) = y_0(c_1,c_2) = e ^{-x}(c_1cosx + c_2 sinx)
\]

la soluzione particolare:

\[
    \bar{y} (x) = A x^{2}+Bx+C
\]

derivo due volte:

\[
    \bar{y} '(x) = 2Ax + B
\]

\[
    \bar{y} ''(x) = 2A
\]

sostituisco in (1):

\[
    2A + 4Ax + 2B + 2Ax^{2}+2Bx + 2C = 3x^{2}
\]

\[
    2Ax^{2} + 2(2A+B) x + 2(A+B+C) = 3x^{2}
\]

risolvo un sistema per le incognite $A,B,C$ e quindi trovo che:

    \begin{equation}
        \begin{cases}
            2A = 3\\
            2A + B = 0\\
            A+B+C= 0
        \end{cases}\,.
    \end{equation}

    \begin{equation}
        \begin{cases}
            A=\frac{3}{2}\\
            B=-3\\
            \frac{3}{2}-3+C=0
        \end{cases}\,.
    \end{equation}

    \begin{equation}
        \begin{cases}
            A=\frac{3}{2}\\
            B=-3\\
            C=\frac{3}{2}
        \end{cases}\,.
    \end{equation}


Quindi la soluzione particolare:

\[
    \bar{y} (x) = \frac{3}{2}x^{2}-3x+\frac{3}{2}
\]


l'espressione quindi è:

\[
    y(x) = e ^{-x}(c_1 cosx + c_2 sinx ) + \frac{3}{2}x^{2}-3x+\frac{3}{2}
\]

derivo:

\[
    y'(x) = e ^{-x}(c_1 cosx+ c_2 sinx) + e ^{-x}(-c_1sinx+c_2 cosx) +3x -3
\]

quindi impongo le condizioni:

    \begin{equation}
        \begin{cases}
            c_1 +\frac{3}{2}=0\\
            -c_1+c_2 -3 = 1
        \end{cases}\,.
    \end{equation}

    \begin{equation}
        \begin{cases}
            c_1=-\frac{3}{2}\\
            c_2=-\frac{5}{2}
        \end{cases}\,.
    \end{equation}

Quindi la soluzione del problema:

\[
    y(x) = e ^{-x}(-\frac{3}{2}cosx + \frac{5}{2} sinx) + \frac{3}{2}x^{2}-3x + \frac{3}{2}
\]

\textbf{Esempio 4} 

    \begin{equation}
        \begin{cases}
            y''-6y' + 9y = e ^{3x}\\
            y(0) = 1\\
            y'(0) = 0
        \end{cases}\,.
    \end{equation}

\[
    \lambda^{2} -6 \lambda + 9 = 0
\]

\[
    \lambda_1=\lambda_2=3
\]

\[
    y_0(x)  = c_1 e ^{3x}+ c_2 x e ^{3x}
\]

qua abbiamo il problema che $\lambda=3$ che è la stessa del termine noto, devo quindi modificarla:

\[
    \bar{y} (x) = A x^{2}e ^{3x}
\]

dove il due della x viene dalla molteplicità, 2 in questo caso.

Proviamo a risolverlo con il metodo di variazione delle costanti:

\[
    \bar{y} (x) = c_1(x) e ^{3x}+ c_2(x) x e^{3x}
\]

lo risolvo col sistema (porco dio):

    \begin{equation}
        \begin{cases}
    c_1'(x)e ^{3x} + c_2'(x) x e ^{3x}=0 \\
    c_1'(x) 3 e ^{3x} + c_2'(x) (e ^{3x} + 3x e ^{3x}) = e ^{3x}
        \end{cases}\,.
    \end{equation}

% \[
%     c_1'(x) = \frac{
%                 \begin{vmatrix}
% 0 & x e ^{3x}  \\
%  e ^{3x} & (3x+1) e ^{3x}  \\
% \end{vmatrix}
%     }
%         \begin{vmatrix}
%     e ^{3x} & x e ^{3x}  \\
%     3 e ^{3x} & (3x+1) e ^{3x}  \\
%         \end{vmatrix}
% \]

facendolo viene:

    \begin{equation}
        \begin{cases}
            c_1(x) = \int_{}^{} {-x} \: dx =-\frac{x^{2}}{2}\\
            c_2(x) = \int_{}^{} {} \: dx  = x
        \end{cases}\,.
    \end{equation}

\[
    \bar{y} (x) = - \frac{x^{2}}{2}e ^{3x} + x x e ^{3x} = \frac{x^{2}}{2}e ^{3x}
\]


\textbf{Risoluzione esercizio 6 appunti prof} 

    \begin{equation}
        \begin{cases}
            y'= \frac{sinx}{cosy}\\
            y( \frac{\pi}{2} = \frac{\pi}{6}
        \end{cases}\,.
    \end{equation}

questa è a variabili separabili:

\[
    y'(x) cosy(x) = sinx
\]

\[
    \int_{}^{} {y'(x) cosy(x)} \: dx = \int_{}^{} {sinx} \: dx 
\]

\[
    \int_{}^{} {cosy} \: dy = \int_{}^{} {sinx} \: dx 
\]

\[
    sin y = -cos x +c
\]

impongo adesso le condizioni:

\[
    sin y( \frac{\pi}{2}) = - cos \frac{\pi}{2} + c
\]

\[
    sin \frac{\pi}{6} = c
\]

quindi:

\[
    \frac{1}{2}=c
\]

quindi sostituisco la c:

\[
    sin y(x) = - cos x + \frac{1}{2}
\]

Devo fare l'arcoseno e vedo dove è definita la cosa:

\[
    -1 \le sin y(x) \le 1
\]

ma anche:

\[
    -1 \le cosx + \frac{1}{2} \le 1
\]

impongo il sistema:

    \begin{equation}
        \begin{cases}
            -cosx +\frac{1}{2}\ge -1\\
            -cos x + \frac{1}{2} \le 1
        \end{cases}\,.
    \end{equation}

    \begin{equation}
        \begin{cases}
            cosx \le  \frac{3}{2}\\
            cosx \ge -\frac{1}{2}
        \end{cases}\,.
    \end{equation}

% \begin{tikzpicture}
% \begin{axis}[
%     xmin = -5, xmax = 5,
%     ymin = -5, ymax = 5]
%     \addplot[
%         domain = -5:5,
%         samples = 200,
%         smooth,
%         thick,
%         blue,
%     ] {sinx};
% \end{axis}
% \end{tikzpicture}

Prendo la soluzione nell'intervallo:

\[
    -\frac{2}{3}\pi \le  x \le  \frac{2}{3}\pi
\]

la soluzione del problema: 

\[
    y(x) = arcsin(-cosx +\frac{1}{2})
\]
   
\newpage

\section{Lezione 7}

\subsection{Funzioni in più variabili}

In particolare:

\begin{itemize}
    \item funzioni reali di più variabili $f: A \in R^{n} \rightarrow \mathbb{R}$
    \item funzioni a valori vettoriali $g: A \in \mathbb{R}^{n}\rightarrow \mathbb{R}^{m}$
\end{itemize}

\defn{Vettore}{ Il vettore $n \in \mathbb{R}^{n}$ è una n-pla $x=(x_1,x_2,...,x_n)$ }



Operazioni in $\mathbb{R}^{n}$:

\begin{itemize}
    \item moltiplicazione per scalare

        \[
            \forall c \in \mathbb{R}, \forall \textbf{x} \in \mathbb{R}^{n}
        \]

        \[
            c \textbf{x} = (cx_1,...,cx_n)
        \]

    \item somma

        \[
            \forall \textbf{x}, \textbf{y}  \in \mathbb{R}^{n}
        \]

        \[
       \textbf{x}  =(x_1,...,x_n), \textbf{y}  = (y_1,...,y_n)
        \]

        \[
            \textbf{x}+ \textbf{y} = (x_1+y_1,...,x_n+y_n)
        \]

    \item prodotto scalare

        \[
            \forall \textbf{x} ,\textbf{y} \in \mathbb{R}^{n}
        \]

        \[
            \langle x,y \rangle  = x \bullet y := \sum^{n}_{i=1} x_i y_i
        \]

        L'operazione quindi va $\langle , \rangle : \mathbb{R}^{n} \times \mathbb{R}^{n} \rightarrow \mathbb{R}$

        \textbf{Esempio} 

        \[
            x=(1,2,0,3,5) \in \mathbb{R}^{5}
        \]

        \[
            y=(2,5,1,7,3) \in \mathbb{R}^{5}
        \]

        \[
            \langle x,y \rangle  = x \bullet y = 48
        \]

        Il prodotto scalare verifica le seguenti proprietà

        \begin{enumerate}
            \item \textbf{Bilinearità} (lineare su ogni fattore):

                \[
                    (\alpha x_1 + \beta x_2) \bullet y = \langle (\alpha x_1 + \beta x_2), y \rangle  = \alpha \langle x_1,y \rangle+ \beta \langle x_2,y \rangle = \alpha x_1 \bullet y + \beta x_2 \bullet y
                \]

                \[
                    \forall x_1,x_2,y \in \mathbb{R}, \forall \alpha, \beta \in \mathbb{R}
                \]

            \item \textbf{Simmetria} (l'ordine non conta)

                \[
                    \forall x,y \in \mathbb{R}^{n}
                \]

                \[
                    x \bullet y = y \bullet x
                \]

            \item \textbf{Positività} 
                
                \[
                    \forall x \in  \mathbb{R}^{n}
                \]

                \[
                    x=(x_1,...,x_n)
                \]

                \[
                    x \bullet x = \langle x,x \rangle = \sum^{n}_{i=1} x_i^{2}\ge 0
                \]

                \[
                    x \bullet x = \langle x,x \rangle = 0 \Leftrightarrow x = 0 = (0,...,0)\ \text{vettore nullo}
                \]

        \end{enumerate}


        \defn{Norma}{
        Il numero reale (non negativo)

        \[
          |x|:=  \sqrt{x \bullet x} = \sqrt{\langle x,x \rangle}
        \]


        si chiama \textbf{lunghezza} o \textbf{norma} del vettore

        }

           
\end{itemize}

\newpage

  % proposizione 
\proposizione{Formula di Carnot}{

    \[
        \forall x,y \in \mathbb{R}^{n}
    \]

    si ha:

    \[
    |x+y| = |x|^{2} + |y|^{2} + 2x\bullet y
    \]
}

\begin{proof}
       \[
           |x+ y| ^{2} = \langle x+y , x+y \rangle = (x+y) \bullet (x+y) \overset{\text{bilinearità}}{=} \langle x,x+y \rangle + \langle y,x+y \rangle \overset{\text{sempre bilinearità}}{=} 
       \]    

       \[
           =\langle x,x \rangle + \langle x,y \rangle + \langle y,x \rangle + \langle y,y \rangle = \langle x,x \rangle + 2\langle x,y \rangle + \langle y,y \rangle= 
       \]

       \[
            = |x|^{2} + |y|^{2} + 2x \bullet y
       \]
\end{proof}


Altra cosa interessante:

 \[
    |x+y|^{2}= |x|^{2} + |y|^{2} \Leftrightarrow x\bullet y  =0
 \]



 \proposizione{Disuguaglianza di Cauchy-Schwarz}{

     \[
         \forall x,y \in \mathbb{R}^{n}
     \]

     \[
         |\langle x,y \rangle| \le |x| |y|
     \]

     si ha:

     \[
         x\bullet y = |x| |y|  \Leftrightarrow y = 0 \lor x=\lambda y \text{ con } \lambda \in \mathbb{R}\ge 0
     \]

 }

 \begin{proof}
     Se
     \[
        y=0
     \]

     \[
         y=0=(0,...,0) 
     \]
     
     questo caso va bene.

     Sia dunque $\mathbb{R}^{n} \rightarrow y \neq 0$ e consideriamo la funzione reale di una variabile reale $t \rightarrow |x+ty|^{2}\ge 0$ polinomio di secondo grado in t

     \[
         |x + ty| ^{2} \overset{\text{Carnot}}{=} |x|^{2} + |ty|^{2} + 2\langle x,ty \rangle  = |x|^{2} + |y|^{2}t^{2} + 2 \langle x,y \rangle t
     \]

     è un polinomio di II grado in t dove $|y|^{2}> 0 $ essendo $y \neq 0$

     Il nostro $\frac{\Delta}{4}$ deve essere non positivo:

     \[
         (x\bullet y ) ^{2} - |x| ^{2}|y|^{2} \le 0
     \]

     \[
         (x\bullet y ) ^{2}\le  |x| ^{2}|y|^{2} 
     \]

     da cui si ha la tesi.

     Si verifica, se si ha che

     \[
         \langle x,y \rangle = |x| |y|
     \]

     si ha che il $\Delta$ del trinomio di II grado è nullo e dunque $t \in \mathbb{R}$ per cui $|x+ty|^{2}=0$ ovvero $x +ty=0$ $\rightarrow x=-ty$

     devo mostrare che $-t \ge 0$


     \[
         t = - \frac{\langle x,y \rangle}{|y|^{2}}
     \]

     si ricorda che $|y|>0$ essendo y non nullo

     \[
         -t = \frac{|x| |y|}{|y|^{2}}\ge 0
     \]

 \end{proof}


 Definiamo ora la funzione \textbf{lunghezza} che è una norma $\mathbb{R}^{n}\rightarrow \mathbb{R}_0^{+}$

 C'è una proprietà che è quella di omogeneità:

 \[
     |\lambda x| = \underbrace{|\lambda|}_\text{valore assoluto} |x| 
 \]

 \[
 \forall \lambda \in \mathbb{R},x \in \mathbb{R}^{n}    
 \]

 e anche 

 \defn{Disuguaglianza triangolare}{
 La disuguaglianza triangolare si definisce come:

 \[
     |x+y| \le |x|+|y|
 \]
   
se $|x+y| = |x| + |y|$ $\rightarrow $ $y =0$ $\lor$ $x= \lambda y $ con $\lambda \ge 0$:

 }

 

\begin{proof}
       Dimostriamo la disuguaglianza triangolare, considero:

       \[
           |x+y|^{2} = \langle x+y , x+y \rangle = |x|^{2}+|y|^{2} + 2 \langle x,y \rangle \le 
       \]

       \[
           \le |x|^{2} + |y| ^{2} + 2|\langle x,y \rangle| \le \underbrace{|x| ^{2} + |y|^{2} + 2|x|\bullet |y|}_{(|x|+|y|)^{2}}
       \]

       estraendo e passando alle radici si ha

       \[
           |x+y| \le |x|+|y|
       \]
\end{proof}



\defn{Distanza Euclidea}{
Distanza Euclidea si definisce come $d(x,y)$:

\[
    d(x,y) := |x-y| = \sqrt{\langle x-y,x-y \rangle} = \sqrt{\sum^{n}_{i=1} (x_i - y_i)^{2}} \ge 0
\]

questa è la norma

}

\newpage

\section{Lezione 8}

La distanza euclidea verifica le proprietà (le stesse del valore assoluto):

\begin{enumerate}
    \item  $d(x,y) \ge 0, d(x,y) \in \mathbb{R}$
    \item  $d(x,y) = 0 \Leftrightarrow x=y$
    \item  $d(x,y) = d(y,x)$ simmetria
    \item  $d(x,y) \le d(x,z) + d(z,y), \forall x,y,z \in \mathbb{R}^{n} $ disuguaglianza triangolare
\end{enumerate}

\defn{Spazio metrico}{ Uno spazio metrico è un insieme X dotato di un'applicazione definita: $X \times X \rightarrow \mathbb{R}$ che verifica la proprietà sopra:

    \[
        (\mathbb{R}^{n},\underbrace{d}_\text{distanza euclidea}) \text{ spazio metrico}
    \]

    esistono altre distanze che ci definiscono relative metriche equivalenti

}


\textbf{Esempio} 

\[
    \mathbb{R}^{2},\forall x,y \in \mathbb{R}^{2}
\]

\[
    x=(x_1,x_2),y=(y_1,y_2)
\]

\[
    d(x,y) = \sqrt{(x_1-y_1)^{2}+(x_2-y_2)^{2}}
\]

la distanza si può scrivere anche:
   
\[
    d_1(x,y) = |x_2-x_1| + |y_2-y_1| \text{ dove abbiamo usato i valori assoluti}
\]

\subsection{Successioni convergenti in $\mathbb{R}^{n}$}

\defn{Successione}{ Una successione è un elenco ordinato di numeri $\{x_n\} \subset \mathbb{R}^{n}$ (gli elementi della successione sono elementi di $\mathbb{R}^{n}$ ovvero n-ple di reali)

    \[
        x_n= (x_n^{1}, x_n^{2}, ..., x_n ^{n})
    \]

    si dice che converge a $x \in \mathbb{R}^{n}$ se:

    \[
    d(x_n,x) \xrightarrow[] {n \rightarrow +\infty} 0
    \]

    cioè:

    \[
        \forall \varepsilon >0 , \exists \bar{N} \in \mathbb{N}
    \]

    \[
        d(x_n, x ) < \varepsilon, \forall n \ge \bar{N} 
    \]

    \[
        |x_n - x| = \sqrt{(x_n^{1}-x^{1})^{2}+ ... +(x_n^{n}-x^{n})^{2}}
    \]

}

\proposizione{}{ Sia $\{x_n\} \subset \mathbb{R}^{n}$ una successione in $\mathbb{R}^{n}$. Si ha:

    \begin{enumerate}
        \item (\textbf{Unicità}) $\{x_n\}$ ha massimo un unico limite (se $\{x_n\}$ ammette limite questo è unico)
        \item (\textbf{Limitatezza}) ogni successione convergente è limitata: 

             \[
                 \exists x_0 \in \mathbb{R}^{n},M \in \mathbb{R}
             \]

             \[
                 d(x_n,x_0) \le M, \forall n
             \]

         \item (\textbf{Sottosuccessione}) Se $\{x_n\}$ convergente a $x \in  \mathbb{R}^{n}$ (o in $X$) allora ogni sottosuccessione ${x_{n_k}}$ estratta da ${x_n}$ converge allo stesso limite
    \end{enumerate}

}

\subsection{Elementi di topologia in $\mathbb{R}^{n}$ (in X)}

Sia $x_0 \in \mathbb{R}^{n}$ fissato e $r>0$

Si ha la seguente definizione

\defn{}{Si definisce palla aperta, disco aperto, intorno sferico di centro $x_0$ e raggio $r$ l'insieme e che si indica con $B(x_0,r)$ l'insieme:

    \[
        B(x_0,r) := \{ x \in \mathbb{R}^{n}, d(x,x_0)<r\} \subset \mathbb{R}^{n}
    \]

    praticamente un intorno di $x_0$ in $\mathbb{R}^{n}$


}

\begin{figure}[ht]
    \centering
    \incfig{disco-aperto}
    \caption{disco aperto}
    \label{fig:disco-aperto}
\end{figure}

\textbf{Esempio in $(\mathbb{R}^{2},d)$ e $(\mathbb{R}^{2},d_1)$} 

\[
    d(x,y) = \sqrt{\sum^{r}_{i=1} (x_i-y_i)^{2}}
\]

\[
    d_1(x,y) = |x_1-y_1| + |x_2-y_2|
\]

se $x_0=0$ e $r=1$

\[
    B(0,1)  = \{\bar{x} \in \mathbb{R}^{2}, d(x,0) <1\} = \{x \in  \mathbb{R}^{2}, \sqrt{x_1^{2}+y_1^{2}}<1\}
\]

questo è un cerchio


\[
    B(0,1)  = \{\bar{x} \in \mathbb{R}^{2}, d_1(x,0) <1\} = \{x \in  \mathbb{R}^{2}, \underbrace{|x_1-0|}_{x_1} + \underbrace{|y_1-0|}_{y_1}<1\}
\]

questo è un rombo (parallelogramma).


Abbiamo parlato di questa roba per definire i limiti attraverso gli intorni sferici

Quindi:

\[
\{ x_n\} \subset X \text{ converge a } x_0 \in X \Leftrightarrow \forall \varepsilon >0 , \exists n_\varepsilon, x_n \in B(x_0,\varepsilon), \forall n \ge n_\varepsilon
\]

\defn{Sottoinsieme limitato}{$A \subset X$ si dice limitato se esiste una palla aperta in cui $A$ risulta interamente contenuto:

    \[
        \exists r>0,\exists x_0 \in X \text{ t.c. } A \subset B(x_0,r)
    \]
}

\defn{Punto interno}{

Un punto di $x_0 \in X$ si dice interno ad $A$ dove $A \subset X$ e non solo $x_0 \in A$ ma esiste (almeno) un suo intorno sferico interamente contenuto in $A$:

\[
    \exists r>0\ B(x_0,r) \subset A
\]

$\dot A$ insieme di punti interni ad $A$

}

\defn{Punto esterno}{ $x_0 \in X$ si dice esterno ad $A$ ($A \subset X$) se non solo $x_0$ non appartiene ad $A$ ma vi è almeno un suo intorno sferico completamente disgiunto ad $A$


    \[
        x_0 \in A \text{ e } \exists r>0\ B(x_0,r)\cap A = \emptyset
    \]

}


I punti che non sono ne esterni ne interni si dicono di frontiera.


\defn{Insieme aperto}{$A \subset X$ si dice aperto se $A = \emptyset$ oppure se ogni punto è un punto interno di $A$ (ovvero per ogni punto di $A$ c'è un intorno sferico tutto contenuto in $A$)}


\defn{Insieme chiuso}{ Un insieme $C \subset X$ si chiuso se il suo complementare è un insieme aperto:

    \[
        X \setminus  C = A
    \]

}


\textbf{Esempio} 

Sia

\[
    A_2= \{ (x,y) \in \mathbb{R}^{2}, x \neq y^{2}\} \subset \mathbb{R}^{2}
\]

\begin{figure}[ht]
    \centering
    \incfig{esercizio-su-insieme-chiuso-1}
    \caption{esercizio su insieme chiuso 1}
    \label{fig:esercizio-su-insieme-chiuso-1}
\end{figure}


\textbf{Esercizio per casa} 

\[
    A_1= \{ (x,y) \in \mathbb{R}^{2}, x^{2}+ y^{2} < 25\}
\]

\[
    A_2= \{ (x,y) \in \mathbb{R}^{2}, x^{2}+ 2y +1  \le  0\}
\]


\section{Lezione 9}

\subsection{Limiti di Funzioni in più variabili}

\defn{Punto di accumulazione}{

    Un punto $x_0 \in R^{n}$ di accumulazione per $A \subseteq R^{n}$ si dice punto di accumulazione se in ogni intorno circolare di $x_0$ c'è almeno un punto di $A$ diverso da $x_0$

}


\textbf{Esempi} 

\begin{itemize}
    \item I punti che costituiscono l'insieme dei punti interni di $A: \dot A$  sono punti di accumulazione
    \item I punti di frontiera, ovvero i punti di $\delta A$ possono essere punti di accumulazione di $A$ oppure non esserlo in quest'ultimo caso si dice che è un punto isolato
\end{itemize}

\newpage

\defn{Convergenza in $\mathbb{R}^{n}$}{Data una successione $ \{x_n\}\in \mathbb{R}^{n}$ questa si dice che converge a $x_0 \in  \mathbb{R}^{n}$ se:

\[
    \lim_{ n \to +\infty } d(x_n,x_0)=0
\]

questo equivale a dire $\forall \varepsilon >0, \exists \bar{N} \in \mathbb{N}$ e $\forall n \ge \bar{N}$ si ha:

\[
    d(x_n,x_0) < \varepsilon
\]

}

\defn{Punto di accumulazione con limiti}{

    $x_0$ è di accumulazione per $A$ $\Leftrightarrow x_0$ è il limite di una successione di elementi di $A$ tutti diversi da $x_0$
}


\textbf{Esempio di punti di accumulazione} 

\[
    A \{\bar{x}  \in R^{2};1<\underbrace{|x|}_\text{d(x,0)}<2\}
\]


\begin{figure}[ht]
    \centering
    \incfig{disegno-punto-di-accumulazione-esempio-1}
    \caption{disegno punto di accumulazione esempio 1}
    \label{fig:disegno-punto-di-accumulazione-esempio-1}
\end{figure}


Tutti i punti di $A \in \mathbb{R}^{2}$ sono punti di accumulazione

I punti del disegno sono sia punti di frontiera che di accumulazione


Nel caso di $\emptyset, \mathbb{R}^{n}$ gli insiemi sono contemporaneamente sia aperti che chiusi.


\defn{Chiusura di un insieme $A \subset \mathbb{R}^{n}$}{

    Si indica con $\bar{A} $ è un sottoinsieme di $\mathbb{R}^{n}$ dato dall'unione di $A$ e dei suoi punti di accumulazione ($DA$)

    $\bar{A} $ è un insieme chiuso. Lo si può pensare come l'intersezione dei chiusi contenenti $A$.

    Si può inoltre dimostrare che:

    \[
        \bar{A} = A \cup \delta A
    \]

}

\defn{Dominio}{ Un dominio $D$ in $\mathbb{R}^{n}$ è la chiusura di un insieme aperto:

    \[
        D = \bar{A}  = A \cup \delta A
    \]
}

Consideriamo ora le funzioni in più variabili $f: A \subset \mathbb{R}^{n} \rightarrow \mathbb{R}$

\defn{Limite di funzioni in più variabili}{

Sia $x_0 \in \mathbb{R}^{n}$ un punto di accumulazione per $A$

Si dice che $f(\bar{x})$ tende (ha limite) a $l$ per $\bar{x} $ che tende a $x_0$:

\[
    \lim_{ \bar{x}  \to x_0 } f(\bar{x} ) = l
\]

scrivendolo tramite gli intorni: se $\forall $ intorno $U \subset \mathbb{R}$ di $l$ esiste un intorno di $x_0$ (sferico) $I(x_0,r)$ con $r>0$ 

tale che $f(\bar{x}) \in U$ $\forall x \in \underbrace{I(x_0,r)}_{B(x_0,r)}\cap (A\{x_0\})$ 

L'altra definizione con i delta:

\[
    \forall \varepsilon>0\  \exists \delta >0
\]

tale che 

\[
    \underbrace{|f(x) - l|}_{d(f(x),l) \in \mathbb{R}} <\varepsilon
\]

$\forall \bar{x} \in A \setminus \{x_0\}$ con $|x-x_0| < \delta$
}

\subsection{Proprietà dei limiti di funzioni in più variabili}

Adesso parliamo di un po' di proprietà:

\begin{itemize}
    \item Il limite quando esiste è \textbf{unico}
    \item I limiti di \textbf{somme} e di \textbf{prodotti} di funzioni sono dati dalla somma e dal prodotto dei limiti (se definito)
    \item Il limite del \textbf{quoziente} di due funzioni è il quoziente dei limiti (se definito)
\end{itemize}

\textbf{Esempi} 

Sia 

\[
    f(x,y) = \frac{x^{2}}{\sqrt{x^{2}+y^{2}}}
\]

La mia $f: \underbrace{\mathbb{R}^{2}\setminus \{(0,0)\}}_\text{A aperto}\rightarrow \mathbb{R}$

Il punto $(0,0)$ è punto di accumulazione per $A$ 

Vogliamo vedere che succede quando la funzione tende a questo punto di accumulazione:

\[
    \lim_{ (x,y) \to (0,0) } f(x,y) = 0
\]

questo significa che $\forall \varepsilon >0 , \exists \delta >0$ tale che $|f(x,y) -0| = |f(x,y)| <\varepsilon$ $\forall (x,y) \in A = \mathbb{R}^{2}\setminus \{(0,0)\}$ con $0< \sqrt{x^{2}+y^{2}}<\delta$:

\[
    \underbrace{0\le}_\text{sempre positiva} f(x,y) = \frac{x^{2}}{\sqrt{x^{2}+y^{2}}} \le \frac{x^{2}+y^{2}}{\sqrt{x^{2}+y^{2}}} \overset{\text{razionalizzo}}{=} \sqrt{x^{2}+y^{2}}
\]

E dunque  $\forall \varepsilon$ si ha:

\[
    0 \le f(x,y) < \varepsilon
\]

per ogni $(x,y) \neq (0,0)$ t.c. $\sqrt{x^{2}+y^{2}} < \varepsilon$


\textbf{Esercizio per casa}

Mostrare che 

\[
    \lim_{ (x,y) \to (0,0) } \frac{x}{\sqrt{x^{2}+y^{2}}} \text{ non esiste}
\] 

\textbf{Soluzione} 

\begin{figure}[ht]
    \centering
    \incfig{esercizio-limite-casa}
    \caption{esercizio limite casa}
    \label{fig:esercizio-limite-casa}
\end{figure}

Se calcolo la funzione:

\[
    f(x,0) = \frac{x}{\sqrt{x^{2}}}= \frac{x}{|x|}
\]

questa fa:

\begin{equation}
    \begin{cases}
           1,x>0\\
           -1,x<0
    \end{cases}\,.
\end{equation}

Il limite quindi non esiste perché ha valori diversi a seconda del caso e non va bene

Invece:

\[
    f(0,y) = 0
\]

\section{Lezione 10}

\proposizione{}{Se 

    \[
        \lim_{ (x,y) \to (x_0,y_0) } f(x,y)=l
    \]

    allora per ogni sottoinsieme $C$ di $A$ (si sottintende che $P_0=(x_0,y_0)$ sia punto di accumulazione per $C$)

    Si deve avere:

    \[
        \lim_{ \underbrace{(x,y) \to (x_0,y_0)}_{(x,y) \in C} } f(x,y)=l
    \]

}

\textbf{Esercizi} 

\textbf{1} 

Mostriamo che:

\[
    \lim_{ (x,y) \to (0,0) } \frac{xy}{x^{2}+y^{2}}
\]

non esiste.

Restringiamo lo studio di funzione lungo l'asse x ($y=0$):

\[
    f(x,0) = 0
\]

\[
    \lim_{ \underbrace{(x,y) \to (0,0)}_{y=0} } f(x,y)
\]

Stessa cosa lungo l'asse y ($x=0$):

\[
    \lim_{ \underbrace{(x,y) \to (0,0)}_{x=0} } f(x,y)
\]

Candidato limite è a 0 


Adesso ci spostiamo con altri parametri tipo la bisettrice del primo e del terzo quadrante $y=x$:

\[
    \lim_{ \underbrace{(x,y) \to (0,0)}_{y=x} } f(x,y) = \lim_{ x \to 0 } f(x,x)=
\]


\[
    = \lim_{ x \to 0 } \frac{x^{2}}{x^{2}+y^{2}} = \lim_{ x \to 0 } \frac{x^{2}}{2x^{2}}= \frac{1}{2}
\]

\textbf{2} 

\[
    \lim_{ (x,y) \to (0,0) } \frac{xy^{2}}{x^{2}+y^{4}}
\]

$f:\mathbb{R}^{2}\setminus (0,0) \rightarrow \mathbb{R}$

per $y=0$ viene a 0 e anche per $x=0$

Considero quindi qualunque retta passante per l'origine:

\[
    y= mx
\]

con $m \neq 0$ e $x \neq 0$:

\[
    \lim_{ \underbrace{(x,y) \to (0,0)}_{y=mx} } f(x,y) = \lim_{ x \to 0 } f(x,mx) = \lim_{ x \to 0 } \frac{m^{2}x^{3}}{x^{2}+m^{4}x^{4}}= 0
\]

ma questo non basta, devo controllare anche il caso della parabola $y^{2}=x$:


\[
    \lim_{ \underbrace{(x,y) \to (0,0)}_{x=y^{2}} } f(x,y) = \lim_{ y \to 0 } f(y^{2},y) = \lim_{ y \to 0 } \frac{y^{4}}{y^{4}+y^{4}} = \lim_{ y \to 0 } \frac{y^{4}}{2y^{4}} = \frac{1}{2} \neq 0
\]

\defn{Funzione continua in più variabili}{ Sia una funzione e sia $P_0$ un punto di accumulazione per a, si dice che la funzione è continua in $P_0$ se:

    \[
        \lim_{ P \to P_0 } f(P) = f(P_0)
    \]

    se $P_0$ è un punto isolato per $A$ per convenzione $f$ è continua

}

\textbf{Esempi} 

Avendo queste due funzioni 

\[
    f(x,y)=x
\]

\[
    g(x,y) = y
\]

devo mostrare che $f$ e $g$ sono continue in ogni punto:

Consideriamo la $f$
 
Sia dunque $(x_0,y_0) \in  \mathbb{R}^{2}$ e $\varepsilon >0$ dobbiamo mostrare che $\exists \delta= \delta(\varepsilon) >0$:

\[
     d(f(x,y)- f(x_0,y_0) ) < \varepsilon
\]

se $d(P,P_0) < \delta$ 

scritto meglio

\[
    d(P,P_0) = \sqrt{(x-x_0)^{2}+(y-y_0)^{2}}
\]

mostriamo che $\sqrt{(x-x_0)^{2}+(y-y_0)^{2}} < \delta $ si ha $|x-x_0|< \varepsilon$:


\[
|x-x_0| = \sqrt{(x-x_0)^{2}}\le \sqrt{(x - x_0) ^{2} + (y- y_0) ^{2}} < \delta
\]

dobbiamo prendere quindi $ \delta =\varepsilon$


\teorema{}{Siano $f$ e $g$ continue (sugli opportuni domini) allora:

    \begin{itemize}
        \item $f+g , f\cdot g$ sono continue 
        \item se $g \neq 0$ allora $\frac{f}{g}$ è continua
        \item se $g>0$ allora $f^{g}$ è continua
        \item la funzione comporta $g \circ f$ è continua (dove è definita)
    \end{itemize}
}

Sono dunque funzioni continue:

\begin{itemize}
    \item I polinomi in due variabili 
    \item Le funzioni razionali (rapporti, quoziente di polinomi)
    \item Le funzioni elementari 
\end{itemize}


Condizione necessaria affinché $f(x,y)$ ammetta limite $l$ quando $(x,y) \rightarrow  (x_0,y_0)$ è che per ogni curva regolare di equazione:

\begin{equation}
    \begin{cases}
           x=x(t)\\
           y=y(t)
    \end{cases}\,.
\end{equation}

questa è una curva  passante per $P_0= (x_0,y_0)$:

\[
    \lim_{ t \to t_0 } f(x(t), y(t)) 
\]

si arriva alla stessa conclusione di non esistenza del limite se la restrizione di $f(x,y)$ ad una curva (come sopra) non ha limite. Ovviamente non è vero il viceversa

\subsection{Coordinate polari}

Abbiamo $(\rho,\theta)$ dove:

\[
    \rho = \bar{ OP} = d(P,O) = \sqrt{x^{2}+y^{2}}
\]

\[
    \theta = arctan \frac{y}{x}
\]

\begin{equation}
    \begin{cases}
           x= x_0+ \rho cos \theta\\
           y = y_0+ \rho sin \theta
    \end{cases}\,.
\end{equation}

Scriviamo i limiti con le coordinate polari:

\[
    \lim_{ (x,y) \to (x_0,y_0) } f(x,y)
\]

\begin{figure}[ht]
    \centering
    \incfig{disegno-polari}
    \caption{disegno polari}
    \label{fig:disegno-polari}
\end{figure}

\teorema{}{Sia $f: D \subset \mathbb{R}^{2} \rightarrow  \mathbb{R}$ e sia $P_0=(x_0,y_0) \in D$ allora:

    \[
        \lim_{ (x,y) \to (x_0,y_0) } f(x,y)  = l \Leftrightarrow \lim_{ \rho \to 0^{+} } f(x_0+\rho cos \theta, y_0+\rho sin \theta) = l
    \]

    uniformemente rispetto a $\theta$

}

\newpage

\section{Lezione 11}

\defn{}{Limite per coordinate polari: 

    \[
        \lim_{ \rho \to 0^{+} } f(x_0+\rho cos\theta,y_0+\rho sen \theta) = l
    \]

    ovvero che:

    \[
        \forall \varepsilon>0,\exists \sigma>0
    \]

    per ogni:

    \[
        \underbrace{0<\rho<\sigma}_{\rho \rightarrow 0^{+}},\forall \theta \in (0,2\pi)
    \]

    si ha:

    \[
        |f(x_0+\rho cos\theta, y_0+\rho sin \theta ) -l| < \varepsilon
    \]
}

\begin{proof}
       per far vedere che vale il limite è sufficiente mostrare che esiste una funzione $g$ che dipende solo da $\rho$ (non negativa) $g(\rho)\ge 0$ tale che:

       \[
        |f(x_0+\rho cos\theta, y_0+\rho sin \theta ) -l| \le  g(\rho)
       \]

       dove $g(\rho) \rightarrow 0$ per $\rho \rightarrow 0^{+}$

       e poi faccio vedere che quindi (per il teorema dei due carabinieri):

       \[
        0\le |f(x_0+\rho cos\theta, y_0+\rho sin \theta ) -l| \le  g(\rho) = 0
       \]

\end{proof}


Se accade che il limite dipende da $\theta$:

\[
       \lim_{ \rho \to 0^{+} }  f(x_0+\rho cos\theta, y_0+\rho sin \theta )
\]

allora il limite non esiste


\textbf{Esempio già visto} 

Avevamo già mostrato che il limite non esiste:

\[
\lim_{ (x,y) \to (0,0) } \frac{xy}{x^{2}+y^{2}}
\]

\begin{equation}
    \begin{cases}
           x=\rho cos \theta\\
           y = \rho sin \theta
    \end{cases}\,.
\end{equation}

il limite diventa:

\[
    \lim_{ \rho \to 0^{+} } \frac{\rho cos\theta \rho sin \theta}{\rho ^{2} cos^{2}\theta + \rho ^{2}sin ^{2}\theta}  = \lim_{ \rho \to 0^{+} } \frac{\rho^{2} cos\theta sin \theta}{\rho^{2}}  = \frac{1}{2} sin 2\theta
\]


\textbf{Altro esempio} 

\[
    \lim_{ (x,y) \to (0,0) } \frac{xy}{\sqrt{x^{2}+y^{2}}}
\]

trasformiamo in coordinate polari:

\[
    \lim_{ \rho \to 0^{+} } \frac{\rho cos\theta \rho sin \theta}{\sqrt{\rho^{2}}} = \lim_{ \rho \to 0^{+} } \frac{\rho ^{2} cos\theta sin \theta}{\rho}=0
\]

infatti:

\[
    0 \le |\rho cos\theta sin \theta -0| = |\rho \frac{sin2\theta}{2} \le \frac{1}{2}\rho \rightarrow 0
\]

\textbf{Esercizio} 

Calcolare se esistono i seguenti limiti e far vedere che non esistono:

\textbf{1} 

\[
    \lim_{ (x,y) \to (0,0) } \frac{arctan(x+y)^{2}}{x^{2}}  
\]

consideriamo la funzione lungo l'asse x quindi con $y=0$:

\[
    \lim_{ (x,y) \to (0,0) } \frac{arctan(x+y)^{2}}{x^{2}}= \lim_{ x \to 0 }  \frac{arctan x^{2} }{x^{2}} \overset{\text{limite notevole}}{=} 1
\]

vedo per la bisettrice ($y=x$):

\[
     \lim_{ (x,y) \to (0,0) } \frac{arctan(x+y)^{2}}{x^{2}} =  \lim_{ (x,y) \to (0,0) } \frac{arctan(x+x)^{2}}{x^{2}} =  \lim_{ (x,y) \to (0,0) } \frac{arctan 4x^{2}}{x^{2}} = 4
\]

quindi il limite non esiste.

\textbf{2} 

\[
    \lim_{ (x,y) \to (0,0) } \frac{(x+y)^{2}}{x^{2}+y^{2}}
\]

vediamo cosa succede lungo l'asse x ($y=0$):

\[
    \lim_{ (x,y) \to (0,0) } \frac{(x+y)^{2}}{x^{2}+y^{2}} = \lim_{ (x,y) \to (0,0) } \frac{x^{2}}{x^{2}} = 1
\]

per $y=x$:

\[
    \lim_{ (x,y) \to (0,0) } \frac{(x+y)^{2}}{x^{2}+y^{2}} = \lim_{ (x,y) \to (0,0) } \frac{(x+x)^{2}}{x^{2}+x^{2}} = 2
\]

quindi il limite non esiste.

\[
    \lim_{ (x,y) \to (0,0) } \frac{x^{2}-y^{2}}{x^{2}+y^{2}+5} = 0
\]

questo perché è continua.

\textbf{4} 

\[
    \lim_{ (x,y) \to (0,0) } \frac{xlog(1+x^{3})}{y(x^{2}+y^{2})}
\]


passiamo in coordinate polari e dunque il limite diventa:

\[
    \lim_{ \rho \to 0^{+} } \frac{\rho cos \theta log(1+\rho^{3}cos^{3}\theta}{\rho sin\theta (\rho ^{2} cos^{2} \theta + \rho^{2}sin^{2}\theta}=\lim_{ \rho \to 0^{+} } \frac{\rho cos \theta log(1+\rho^{3}cos^{3}\theta)}{\rho^{3}sin\theta}  =
\]

\[
    =\lim_{ \rho \to 0^{+} } \frac{\rho^{4}cos^{4}\theta}{\rho^{3}sin \theta}
\]

\[
    = \lim_{ \rho \to 0^{+} } \rho M(\theta)
\]

dove $M(\theta)= \frac{cos^{4}\theta}{sin\theta}$ si nota che per ogni $\theta$ finito il limite è zero. Candidato limite è 0

Per poter applicare il teorema del confronto (caramba) valutiamo dunque:

\[
    \underbrace{sup}_{\theta} | \rho M(\theta)| = sup |\rho \frac{cos^{4}\theta}{sin\theta}|
\]

se la nostra funzione è limitata per ogni $\theta$ allora:

\[
    |M(\theta)| \le \bar{e} 
\]

il sup cioè è finito sono a posto ma $M(\theta)$ non è limitata (per esempio $\theta= \pi$ è un asintoto verticale):

\[
    \lim_{ \theta \to \pi } \frac{cos^{4}\theta}{sin\theta}= +\infty
\]

allora per studiare il limite vediamo che succede muovendoci verso l'origine lungo curve che sono tangenti all'asse x.

Consideriamo allora $y=x^{2}$:

\[
    f(x,x^{2})= \frac{xlog(1+x^{3})}{x^{2}(x^{2}+x^{4}} = \frac{xlog(1+x^{3})}{x^{4}(1+x^{2}}= \frac{log(1+x^{3})}{x^{3}(1+x^{2})} = 1
\]
 

\textbf{5} 

Studiare, al variare di $\alpha >0 \in \mathbb{R}$, l'esistenza del seguente limite:

\[
    \lim_{ (x,y) \to (1,0) } \frac{x^{2}-2x+1+y^{2}}{(x^{2}-2x+1)^{\alpha}} = \lim_{ (x,y) \to (1,0) } \frac{(x-1)^{2}y}{((x-1)^{2}+y^{2})^{\alpha}} 
\]

passiamo in coordinate polari:

\[
    (x,y) \rightarrow (1,0)
\]

\begin{equation}
    \begin{cases}
           x = 1+\rho cos \theta\\
           y = \rho sin\theta
    \end{cases}\,.
\end{equation}


allora:

\[
    \lim_{ \rho \to 0^{+} } \frac{(1 + \rho cos\theta -1 )^{2} \rho sin\theta}{((1+\rho cos\theta -1)^{2}+\rho^{2}sin^{2}\theta)^{\alpha}} = \lim_{ \rho \to 0^{+} } \frac{\rho^{2}cos^{2}\theta\rho sin\theta}{(\rho^{2}cos^{2}\theta+\rho^{2}sin^{2}\theta)^{\alpha}} = \lim_{ \rho \to 0^{+} } \frac{\rho^{3}cos^{2}\theta sin \theta}{\rho ^{2 \alpha}} =
\]


\[
    = \lim_{ \rho \to 0^{+} } \rho^{3-2 \alpha} cos^{2}\theta sin \theta
\]

poiché $|cos^{2}\theta sin\theta|\le 1$ se dunque $3-2 \alpha >0$ si ha:

\[
    0 \le |\rho^{3-2 \alpha}cos^{2}\theta sin \theta | \le \rho^{3-2 \alpha}
\]


dunque se:

\[
    0 < \alpha < \frac{3}{2} \text{ il limite vale 0}
\]

quindi dobbiamo studiare questa condizione ($\alpha=\frac{3}{2}$):

\[
    \lim_{ \rho \to 0^{+} } cos^{2}\theta sin \theta = cos^{2}\theta sin \theta
\]

dipende da $\theta$ quindi il limite non esiste. Stessa conclusione per $\alpha>\frac{3}{2}$ il limite viene $\pm \infty$ a seconda della scelta di $\theta$

\textbf{6 - Studio di continuità}

Studiamo la continuità della funzione nel sottoinsieme di definizione:

\[
f(x,y)=
    \begin{cases}
        2x+3y+10 & \text{se $(x-1)^{2}+(y-3)^{2}\ge 4$} \\
        x+4y+10 & \text{se $(x-1)^{2}+(y-3)^{2}<4$}
    \end{cases}
\]

devo vedere cosa succede al confine (in 4) quindi:

\[
    \lim_{ P \to P_0 } f(P)
\]

con $P_0 \in $ circonferenza, io voglio:

\begin{equation}
    \begin{cases}
    2x+3y+10 = x +4y +10\\
    \underbrace{(x-1)^{2} + (y-3)^{2}=4}_\text{dentro la circonferenza}    \end{cases}\,.
\end{equation}

quindi risolviamo:

\begin{equation}
    \begin{cases}
           x-y=0\\
           (x-1)^{2}+(y-3)^{2} = 4
    \end{cases}\,.
\end{equation}

\begin{equation}
    \begin{cases}
           x=y\\
           y^{2}-2y+1+y^{2}-6y+9=4
    \end{cases}\,.
\end{equation}

\begin{equation}
    \begin{cases}
           x=y\\
           2y^{2}-8y + 6 = 0
    \end{cases}\,.
\end{equation}

\begin{equation}
    \begin{cases}
           x=y\\
           y^{2}+4y+3=0
    \end{cases}\,.
\end{equation}

\begin{equation}
    \begin{cases}
           x=y\\
           (y-3)(y-1) = 0
    \end{cases}\,.
\end{equation}

\begin{equation}
    \begin{cases}
           x=1,x=3\\
           y=1,y=3
    \end{cases}\,.
\end{equation}

\section{Lezione 12}

\subsection{Rappresentazioni di funzioni}

Per poter rappresentare le funzioni ci serviamo delle \textbf{sezioni} che ci permettono di usare dei sottoinsiemi per le rappresentazioni e quindi di semplificare.

\textbf{Esempio} 

\[
    f(x,y)=x^{2}-y
\]

il suo grafico sarà:

\[
    \{(x,y,z) \in \mathbb{R}^{3},z=x^{2}-y\}
\]

Provo a sezionarlo con $x=k$.

Nel piano $(y,z)$ la sezione è data dal grafico di:

\[
    z= k^{2}-y
\]

questo sara' rappresentato da un fascio di rette a $z=-y$:

\begin{figure}[ht]
    \centering
    \incfig{grafico-sezione-esempio}
    \caption{grafico-sezione-esempio}
    \label{fig:grafico-sezione-esempio}
\end{figure}


\subsection{Insiemi di livello (curva di livello)}

Nel caso in cui $z=k$.

\textbf{Esempio} 

Consideriamo i punti $(x,y)$ che stanno nell'insieme di livello $k=1$ della funzione $f(x,y)=x^{2}+y^{2}$:

\[
    E_k = \{(x,y) \in \mathbb{R}^{2},x^{2}+y^{2}=k\}
\]

sono i punti che stanno sulla circonferenza di centro $(0,0)$ di raggio 1:

\[
    \{(x,y) \in \mathbb{R}^{2},x^{2}+y^{2}=1\} = E_1
\]


\begin{figure}[ht]
    \centering
    \incfig{curva-di-livello-esempio}
    \caption{curva di livello esempio}
    \label{fig:curva-di-livello-esempio}
\end{figure}

Vediamo qualche altro esempio.

\textbf{Esempio} 

Determinare l'insieme di definizione delle seguenti funzioni:

\textbf{1} 

\[
    z=\sqrt{1-x^{2}-y^{2}}
\]


\begin{figure}[ht]
    \centering
    \incfig{disegno-curva-1}
    \caption{disegno-curva-1}
    \label{fig:disegno-curva-1}
\end{figure}

\textbf{2} 

\[
    z=\sqrt{1-x^{2}}+\sqrt{1-y^{2}}
\]

devo imporre il dominio:

\[
    \{(x,y) \in \mathbb{R}^{2},x^{2}\le 1,y^{2}\le 1\}
\]


\begin{figure}[ht]
    \centering
    \incfig{curva-di-livello-2}
    \caption{curva di livello 2}
    \label{fig:curva-di-livello-2}
\end{figure}


\textbf{3} 

\[
    z= \frac{1}{\sqrt{y-\sqrt{x}}}
\]

\[
    D=\{(x,y) \in \mathbb{R}^{2} y-\sqrt{x}>0,x \ge 0\}
\]


\begin{figure}[ht]
    \centering
    \incfig{curva-di-livello-3}
    \caption{curva di livello 3}
    \label{fig:curva-di-livello-3}
\end{figure}

\section{Lezione 13}

\textbf{Risoluzione esercizio}

\[
    z = x^{2}+9y^{2}
\]

Il dominio:

\[
    E_k= \{(x,y) \in \mathbb{R}^{2};x^{2}+9y^{2}=k\}
\]

se $k=0$:

\[
    E_0=(0,0)\text{ origine}
\]

se $k>0$ allora otteniamo delle ellissi:

\[
    \frac{x^{2}}{a^{2}}+ \frac{y^{2}}{ \frac{k}{9}} = 1
\]

\textbf{Altro esercizio} 

\[
    z= \frac{y}{x^{2}}
\]

dominio

\[
    E_k = \{(x,y) \in \mathbb{R}^{2};x+2y=k\}
\]

\[
    y=-\frac{1}{2} + \frac{x}{2}
\]


Fascio di rette parallele a $y = -\frac{1}{2}x$

% \begin{figure}[ht]
%     \centering
%     \incfig{fascio-di-rette-parallelo}
%     \caption{fascio di rette parallelo}
%     \label{fig:fascio-di-rette-parallelo}
% \end{figure}

\textbf{Altro esercizio} 

\[
    z= \underbrace{\frac{y}{x^{2}}}_{f(x,y)}
\]

dominio: 

\[
    E_k = \{(x,y) \in \mathbb{R}^{2}; x \neq  0; \frac{y}{x^{2}}=k\}
\]

se $k=0$, è $E_0$ privata dell'origine:

\[
    \frac{y}{x^{2}}=0
\]

se $k>0$:

\[
    \frac{y}{x^{2}} = k
\]

\[
    y = kx^{2}
\]

quindi sono parabole con concavità verso l'alto.

% \begin{figure}[ht]
%     \centering
%     \incfig{disegno-bello-parabole-verso-l'altro}
%     \caption{disegno bello parabole verso l'altro}
%     \label{fig:disegno-bello-parabole-verso-l'altro}
% \end{figure}


se $k<0$ avranno concavità verso il basso

% \begin{figure}[ht]
%     \centering
%     \incfig{disegno-bello-parabole-verso-il-basso}
%     \caption{disegno bello parabole verso il basso}
%     \label{fig:disegno-bello-parabole-verso-il-basso}
% \end{figure}

\newpage

\textbf{Esempio limiti} 

\[
    \lim_{ (x,y) \to (0,0) } \frac{xy(2y^{2}+x^{3})}{x^{4}+y^{2}}
\]

Consideriamo le restrizione della funzione lungo le rette $y=mx$:


\[
    \underbrace{\lim_{ (x,y) \to (0,0) }}_{y=mx} f(x,y) = \underbrace{\lim_{ (x,y) \to (0,0) }}_{y=mx} \frac{xmx(2m^{2}x^{2}+x^{3})}{x^{4}+m^{2}x^{2}}= \lim_{ x \to 0 } \frac{mx^{2}(2m^{2}+x)}{(x^{2}+m^{2})} = 0
\]

il limite quindi se esiste deve essere zero. Passiamo adesso alle coordinate polari per valutare $f$:

\[
    |f(\rho, \theta)| = | \frac{\rho cos \theta \rho sin \theta (2 \rho^{2}sin^{2}\theta+\rho^{3}cos^{3}\theta)}{\rho^{4}cos^{4}\theta+\rho^{2}sin^{2}\theta}| = |\frac{\rho^{2}cos \theta sin \theta(2sin^{2}\theta+\rho cos^{2}\theta}{\rho^{2}cos^{4}\theta+sin^{2}\theta}|
\]

quindi:

\[
    |f(\rho, \theta)| = |\frac{2 \rho^{2}cos \theta sin^{3}\theta}{\rho^{2}cos^{4}\theta+sin^{2}\theta}+ \frac{\rho^{3}cos^{4}\theta sin \theta}{\rho^{2} cos^{4}\theta + sin^{2}\theta}| \le |\frac{\rho^{2}cos \theta sin^{3}\theta}{\rho^{2}cos^{4}\theta+sin^{2}\theta}| + |\frac{\rho^{3}cos^{4}\theta sin \theta}{\rho^{2} cos^{4}\theta + sin^{2}\theta}|
\]

\[
    |f(\rho, \theta)| \le \frac{2 \rho^{2} |cos \theta sin^{3} \theta|}{\underbrace{\rho^{2}cos^{4}\theta+sin^{2}\theta}_{\ge 0}} + \frac{\rho^{3}cos^{4}\theta|sin \theta|}{\rho^{2}cos^{4}\theta+\underbrace{sin^{2}\theta}_{\ge 0}} \le \frac{2 \rho^{2}|cos \theta sin^{3}\theta|}{sin^{2}\theta} = 2 \rho^{2} |cos \theta sin \theta| + \rho | sin \theta|
\]

infine quindi (data la limitatezza del $sin$ e del $cos$):

\[
 |f(\rho, \theta)| \le 2 \rho^{2} | cos \theta sin \theta| + \rho|sin \theta| \le 2 \rho^{2} + \rho \underbrace{\rightarrow}_{p \rightarrow 0^{+}} 0
\]


Potevamo pero' usare un altro metodo senza coordinate polari:

\[
    |f(x,y)|=  |\frac{xy(2y^{2}+x^{3})}{x^{4}+y^{2}}| = |\frac{2xy^{3}+x^{4}y}{x^{4}+y^{2}}| = | \frac{2xy^{3}}{x^{4}+y^{2}}+ \frac{x^{4}y}{x^{4}+y^{2}}| \le |\frac{2xy^{3}}{x^{4}+y^{2}}|+ |\frac{x^{4}y}{x^{4}+y^{2}}|
\]

quindi:

\[
   |f(x,y)|  \le \frac{2|xy^{3}|}{x^{4}+y^{2}} + \frac{x^{4}|y|}{x^{4}+y^{2}} \le \frac{2|xy^{3}|}{y^{2}} + \frac{x^{4}|y|}{x^{4}}
\]

\[
   |f(x,y)|  \le \underbrace{2 |xy| + |y|}_{g(x,y)}
\]

e quindi il limite di $g(x,y)$ (dato che la funzione è continua):

\[
    \lim_{ (x,y) \to (0,0) } g(x,y) = g(0,0) = 0
\]


\textbf{Esercizio limite 2} 

\[
    \lim_{ (x,y) \to (0,0) } \frac{x sin^{2}y+ 3xy^{4}}{x^{2}+2y^{4}}  
\]

riscrivo la $f$ come somma di due funzioni:

\[
    f(x,y) = \frac{x sin^{2}y+ 3xy^{4}}{x^{2}+2y^{4}}  = \underbrace{\frac{x sin^{2}y}{x^{2}+2y^{4}}}_{f_1(x,y)} + \underbrace{\frac{3xy^{4}}{x^{2}+2y^{4}}}_{f_2(x,y)}
\]

\[
    |f_2(x,y)| = | \frac{3xy^{4}}{x^{2}+2y^{4}}| =\frac{3y^{4}|x|}{x^{2}+2y^{4}} \le \frac{3y^{4}|x|}{2y^{4}} \rightarrow 0
\]

vediamo la prima funzione:

\[
    f_1(x,y) = \frac{xsin^{2}y}{x^{2}+2y^{4}}
\]

il numeratore di questa, se trasformiamo in coordinate polari, è come $\rho^{3}$, controllo quindi il denominatore:

\[
    x^{2} + 2y^{4}
\]

il limite potrebbe non esistere, controlliamolo. Per $y=x$:

\[
    \underbrace{\lim_{ (x,y) \to (0,0) }}_{y=x} f_1(x,y) = \lim_{ x \to 0 } \frac{x sin^{2}x}{x^{2}+2x^{4}} = \lim_{ x \to 0 } \frac{x sin^{2}x}{x^{2}(1+2x^{2})} = 0
\]

Per $x = y^{2}$:

\[
    \underbrace{\lim_{ (x,y) \to (0,0) }}_{x=y^{2}} f_1(x,y)  = \lim_{ y \to 0 } \frac{y^{2}sin^{2}y}{y^{4}+2y^{4}} = \lim_{ y \to 0 } \frac{y^{2}sin^{2}y}{3y^{2}}  = \frac{1}{3}
\]

quindi il limite non esiste perché $f= f_1+f_2$ e $\lim_{ (x,y) \to (0,0) } f_2=0$ e $\lim_{ (x,y) \to (0,0) } f_1$ non esiste.

\subsection{Scelta delle curve di restrizione}

Nell'esercizio di prima come ho fatto a restringere la $f_1$?

Vediamolo:

\begin{itemize}
    \item $y=x$ peso le variabili allo stesso modo e dunque il denominatore ($x^{2}+2y^{4}$) ottengo $x^{2}+2x^{4}$ voglio capire che succede per $x \rightarrow 0$.
    \item $x=y^{2}$ il denominatore ($x^{2}+2y^{4}$) ottengo $y^{4}+2y^{4} = 3y^{4}$ quindi qua non trascurare gli addendi.
\end{itemize}

\textbf{Esempio}:

\[
    x^{6} + 3y^{4}
\]

quindi $x=y$ e poi $y = x^{ \frac{2}{3}}$

\textbf{Altro esempio} 

\[
\lim_{ (x,y) \to (0,0) } ( \frac{xy^{2}+2y^{ \frac{1}{3}}sin^{2}x}{x^{2}+y^{2}}) e ^{ \frac{x^{2}-y^{2}}{x^{2}+y^{2}}}
\]

l'esponente non ha limite pero':

\[
    0< e ^{ \frac{x^{2}-y^{2}}{x^{2}+y^{2}}} 
\]

vediamo l'esponente:

\[
    \frac{x^{2}-y^{2}}{x^{2}+y^{2}} \le |\frac{x^{2}-y^{2}}{x^{2}+y^{2}}| = \frac{|x^{2}-y^{2}|}{x^{2}+y^{2}}  \le \frac{|x^{2}|+|y^{2}|}{x^{2}+y^{2}} = \frac{x^{2}+y^{2}}{x^{2}+y^{2}} = 1
\]

quindi l'esponenziale È:

\[
    0< e ^{ \frac{x^{2}-y^{2}}{x^{2}+y^{2}}} < e
\]


ora studiamo l'altra parte della funzione $f(x,y)$ in coordinate polari:

\[
    | \frac{\rho cos \theta \rho^{2}sin^{2}\theta + 2 \rho^{ \frac{1}{3}}(sin \theta)^{ \frac{1}{3}}sin^{2}(\rho cos \theta)}{\rho^{2}}| \le  | \frac{\rho^{3}cos \theta sin^{2}\theta}{\rho^{2}}| + |\frac{2 \rho^{ \frac{1}{3}}\rho^{2}cos^{2}\theta (sin \theta)^{ \frac{1}{3}}}{\rho^{2}}| \le 
\]

\[
    \le | \rho cos \theta sin ^{2}\theta| + 2 \rho^{ \frac{1}{3}} | cos^{2}\theta| \le \rho + 2 \rho^{ \frac{1}{3}} \rightarrow 0
\]

quindi alla fine il limite:

\[
    \lim_{ (x,y) \to (0,0) } f(x,y)=0
\]

\section{Lezione 14}

\subsection{Calcolo differenziale di funzioni di più variabili}

\textbf{Derivate Parziali}

$f'(x_0)$ misura il tasso di variazione istantanea di f sul punto $x_0$:

\[
    \lim_{ h \to 0 } \frac{f(x_0+h)-f(x_0)}{h}
\]

se esiste ed è finito allora f è derivabile in $x_0$ ed $f'(x_0) = limite$.

Quindi $f'(x_0)$ rappresenta la pendenza della retta tangente al grafico f nel punto $p_0=(x_0,f(x_0))$

\textbf{Esempio} 

sia $f: \mathbb{R}^{2}\rightarrow \mathbb{R}$ e $P_0=(x_0,y_0)$ 

Vogliamo calcolare la derivata parziale di $f$ fatta rispetto a $x$ calcolata in $(x_0,y_0)$:

\[
    \lim_{ h \to 0 } \frac{f(x_0+h,y_0) -f(x_0,y_0)}{h}
\]

purché esista, finito.

Anche la derivata parziale di $f$ rispetto ad $y$ nel punto $(x_0,y_0)$:

\[
    \lim_{ h \to 0 } \frac{f(x_0,y_0+h) - f(x_0,y_0)}{h}
\]

purché esista, finito.

Queste due derivate si indicano:

\[
    \frac{\partial f}{\partial x}(x_0,y_0) = f_x(x_0,y_0) = D_xf(x_0,y_0)
\]

\[
    \frac{\partial f}{\partial y}(x_0,y_0) = f_y(x_0,y_0)= D_yf(x_0,y_0)
\]

\textbf{In generale}:

\[
    f: A \subseteq \mathbb{R}^{n} \rightarrow \mathbb{R}
\]

$A$ definito

\[
    \bar{n} = (x_1,...,x_n) \rightarrow f(\bar{x} ) = f(x_1,...,x_n) \in \mathbb{R}
\]

Sia $P_0= (x_1,...,x_i,...,x_n) \in A$

Si definisce \textbf{derivata parziale di $f$ fatta rispetto a $x_i$ nel punto $P_0=(x_1,...,x_n)$}:

\[
    \lim_{ h \to 0 } \frac{f(x_1,...,x_{i-1},x_{i+h},x_{i+1},...,x_n) - f(x_1,...,x_{i-1},x_{i+h},x_{i+1},...,x_n)}{h}
\]

se tale limite esiste ed è finito.

Si indica:

\[
    \frac{\partial f}{\partial x_i}; D_{x_i}f;f_{x_i}
\]

Vediamo quindi avendo:

\[
    f_x(x_0,y_0),f_y(x_0,y_0)
\]

Derivata parziale rispetto ad $x$ nel punto $P_0=(x_0,y_0)$ tengo fermo $y=y_0$ rispetto a $x$:

\[
    g_1(x)= f(x,y_0) 
\]

questa è una funzione che dipende solo da $x$ (quindi diventa di una sola variabile), se questa è derivabile in $x_0$ cioè:

\[
    \lim_{ h \to 0 } \frac{g_1(x_0+h)-g_1(x_0)}{h} := g_1'(x_0) = \frac{\partial f}{\partial x}(x_0,y_0)
\]

se esiste finito.

Analogamente per $x=x_0$:

\[
    f(x_0,y) := g_2(y)
\]

se $g_2(y)$ è derivabile in $y_0$:

\[
    g_2'(y_0)  = \lim_{ h \to 0 } \frac{g_2(y_0+h)-g_2(y_0)}{h} = \frac{\partial f}{\partial y}(x_0,y_0)
\]


\textbf{Esempio di calcolo delle derivate parziali} 

Sia 

\[
f(x,y) = x \sin(xy^{3})+x^{5}y+2y^{2} 
\]

($f: \mathbb{R}^{2}\rightarrow \mathbb{R})$


\[
    \frac{\partial f}{\partial x}(x,y) = \sin(xy^{3}) + x \cos(xy^{3}) y^{3}+5x^{4}y = \sin(xy^{3}) +xy^{3} \cos(xy^{3}) + 5x^{4}y
\]

\[
    \frac{\partial f}{\partial y}(x,y)  = x \cos(xy^{3}) 3xy^{2}+x^{5}+4y = 3 x^{2}y^{2} \cos(xy^{3}) + x^{5}+4y
\]

Se $P_0 = (2,1) = (x_0,y_0)$:

\[
    \frac{\partial f}{\partial x}(2,1) = \sin(2) + 2 \cos(2) + 5\cdot 2^{4} 
\]

\[
    \frac{\partial f}{\partial y}(2,1)  = 3\cdot 4 \cos(2) + 2^{5}+4 = 12 \cos(2) +32 + 4
\]

Usiamo il metodo delle tracce (quello di prima) con $y=1$:

\[
    f(x,1) = g_1(x)  = x \sin(x) + x^{5}+2 \cdot 1^{2} = c \sin x + x^{5}+2
\]

Sto semplicemente sostituendo e poi derivo:

\[
    g_1'(x) = \sin x+ x \cos x + 5x^{4}
\]

\[
    g_1'(2) = \sin 2 + 2 \cos 2 + 5 \cdot 2^{4} = \frac{\partial f}{\partial x}(2,1)
\]

infatti torna uguale all'altro metodo.

Adesso facciamo per $x=2$:

\[
    f(2,y) := g_2(y) = 2 \sin(2y^{3}) + 2^{5} \cdot y + 2y^{2} = 2 \sin(2y^{3}) + 32y + 2y^{2}
\]

\[
    g_2'(y) = 2 \cos(2y^{3}) 6y^{2} + 32 + 4y
\]

\[
    g_2'(1) = 12 \cos(2) + 32 + 4 = \frac{\partial f}{\partial x}(2,1)
\]


Per le funzioni in due variabili la derivabilità non implica la continuità:

\[
    derivabile \centernot\implies continua
\]

\textbf{Esempio} 

Calcolare, se esistono, le derivate parziali in $(0,0)$:


    \[
        f(x,y)=
     \begin{cases}
        \frac{xy^{2}}{x^{2}+y^{4}} & \text{se $(x,y)\neq (0,0)$} \\
        0 & \text{se $(x,y)=(0,0)$}
    \end{cases}
    \]

Consideriamo le tracce:

\[
    f(x,0) = 0 ; f(0,y) =0
\]

Dunque ottengo due funzioni identicamente nulle che sono derivabili con derivata nulla:

\[
    f_x(0,0) = f_y(0,0) = 0
\]

Osserviamo pero che in $(0,0)$ la funzione non è continua perché:

\[
    \lim_{ (x,y) \to (0,0) } \frac{xy^{2}}{x^{2}+y^{4}} \text{ non esiste}
\]

quello che ci servirà sarà il concetto di \textbf{differenziabilità}.

\subsection{Gradiente}

Se $f: A \subset \mathbb{R}^{2}\rightarrow \mathbb{R}$ è derivabile in $P_0=(x_0,y_0)$:

\[
    \nabla f(x_0,y_0)=(\frac{\partial f}{\partial x}(x_0,y_0), \frac{\partial f}{\partial y}(x_0,y_0)) \in \mathbb{R}^{2}
\]

gradiente di $f$ in $P_0$ (È un vettore).

In generale in $n$ variabili:

\[
    \nabla f(\bar{x} ) = (\frac{\partial f}{\partial x_1}(\bar{x} ), \frac{\partial f}{\partial x_2}(\bar{x} ),..., \frac{\partial f}{\partial x_n}(\bar{x} )) \in \mathbb{R}^{n}
\]

\subsection{Significato geometrico delle derivate parziali}

Nelle funzioni di una variabile:

$f'(x_0)$: rappresenta retta tangente al grafico di f nel punto $P_0=(x_0,f(x_0))$

Nelle due:

\[
    \frac{\partial f}{\partial x}(x_0,y_0), \frac{\partial f}{\partial y}(x_0,y_0)
\]

\[
    graf(f) = \{(x,y,z) \in \mathbb{R}^{3}, z = f(x,y), \forall (x,y) \in A \subset \mathbb{R}^{3}\}
\]

\begin{tikzpicture}[bullet/.style={circle,fill,inner sep=1pt},
 declare function={f(\x,\y)=2-0.5*pow(\x-1.25,2)-0.5*pow(\y-1,2);}]
 \begin{axis}[view={150}{45},colormap/blackwhite,axis lines=middle,%
    zmax=2.2,zmin=0,xmin=-0.2,xmax=2.4,ymin=-0.2,ymax=2,%
    xlabel=$x$,ylabel=$y$,zlabel=$z$,
    xtick=\empty,ytick=\empty,ztick=\empty]
  \addplot3[surf,shader=interp,domain=0.6:2,domain y=0.5:1.2,opacity=0.7] 
   {f(x,y)};
  \addplot3[thick,domain=0.6:2,samples y=1]  ({x},1.2,{f(x,1.2)}); 
  \draw[dashed] (1.75,0,0) node[above left]{$x_0$} -- (1.75,1.2,0)
  node[bullet] (b1) {}  -- (0,1.2,0) node[above right]{$y_0$}
  (1.75,1.2,0) -- (1.75,1.2,{f(1.75,1.2)})node[bullet] {};
  \draw (1.75,1.2,{f(1.75,1.2)}) -- (0.75,1.2,{f(1.75,1.2)+0.5})
  coordinate[pos=0.5] (aux1);
  \draw[opacity=0.5,upper left=gray!80!black,upper right=gray!60,
lower left=gray!60,lower right=gray!80!black] (2,1.2,0) -- (0.6,1.2,0)
   -- (0.6,1.2,2.2) -- (2,1.2,2.2) -- cycle;
  \addplot3[surf,shader=interp,domain=0.6:2,domain y=1.2:1.9,opacity=0.7] 
   {f(x,y)};
 \end{axis}
 \draw (aux1) -- ++ (-1,1) node[above,align=center]{slope in $x$ direction\\
  $\partial_xf(x,y)|_{x=x_0,y=y_0}$};
 \node[anchor=north west] at (b1) {$(x_0,y_0)$}; 
 %
 \begin{axis}[xshift=6.5cm,view={150}{45},colormap/blackwhite,axis lines=middle,%
    zmax=2.2,zmin=0,xmin=-0.2,xmax=2.4,ymin=-0.2,ymax=2,%
    xlabel=$x$,ylabel=$y$,zlabel=$z$,
    xtick=\empty,ytick=\empty,ztick=\empty]
  \addplot3[surf,shader=interp,domain=0.6:1.75,domain y=0.5:1.9,opacity=0.7] 
   {f(x,y)};
   \addplot3[thick,domain=0.5:1.9,samples y=1]  (1.75,{x},{f(1.75,x)}); 
  \draw[dashed] (1.75,0,0) node[above left]{$x_0$} -- (1.75,1.2,0)
  node[bullet] (b2){}
  -- (0,1.2,0) node[above right]{$y_0$}
  (1.75,1.2,0) -- (1.75,1.2,{f(1.75,1.2)})node[bullet] {};
  \draw (1.75,1.2,{f(1.75,1.2)}) -- (1.75,0.2,{f(1.75,1.2)+0.2})
   coordinate[pos=0.5] (aux2);
  \draw[opacity=0.5,upper left=gray!80!black,upper right=gray!60,
lower left=gray!60,lower right=gray!80!black] (1.75,0.5,0) -- (1.75,1.9,0)
   -- (1.75,1.9,2.2) -- (1.75,0.5,2.2) -- cycle;
  \addplot3[surf,shader=interp,domain=1.75:2,domain y=0.5:1.9,opacity=0.7] 
   {f(x,y)};
 \end{axis}
 \draw (aux2) -- ++ (0.3,1) node[above,align=center]{slope in $y$ direction\\
  $\partial_yf(x,y)|_{x=x_0,y=y_0}$};
 \node[anchor=north east] at (b2) {$(x_0,y_0)$};
\end{tikzpicture}

L'equazione del piano tangente alla funzione nel punto $P_0$ È:

\[
    z = f(x_0,y_0) + f_x(x_0,y_0) (x-x_0) + f_y(x_0,y_0) (y-y_0) 
\]


\textbf{Esempio} 

$z= x^{2}+y^{2}= f(x,y)$

Scrivere le equazioni del piano:

dove $S$ è il grafico della funzione
\[
    P_0=(0,0) \Leftrightarrow (0,0,0) \in S
\]

Usando la formula dell'equazione del piano tangente:

\[
    z=0
\]


\[
    P_0=(1,2) \Leftrightarrow (1,2,5) \in S
\]

\[
    z= 5+ 2(x-1) + 4(y-2) = 2x+4y -5
\]

quindi:

\[
    z = 2x+4y-5
\]

\newpage

\section{Lezione 15}

\subsection{Differenziabilità}

Per dire che $z = f(x_0,y_0) + f_x(x_0,y_0) (x-x_0) + f_y(x_0,y_0) (y-y_0) $ è l'equazione del piano tangente, devo vedere anche la differenziabilità (il piano contiene le rette $T_1$ e $T_2$ che hanno in comune $P_0$).


Il fatto che la funzione ad \textbf{una variabile} sia derivabile in un punto mi dice che esiste la retta tangente in quel punto.


Nelle due variabili questo non accade, bisogna vedere altro.

Prima parliamo delle curve:

\subsubsection{Curve in $\mathbb{R}^{n}$}

\defn{Curva}{
Una curva è un'applicazione continua:

\[
    \varphi: I \subset \mathbb{R} \rightarrow \mathbb{R}^{n}
\]

per $I=[a,b]$:

\[
   \bar{\varphi}(t) = (\varphi_1(t),\varphi_2(t),...,\varphi_n(t)) \in \mathbb{R}^{n}
\]

Le equazioni parametriche sono:

$\varphi=$\begin{equation}
    \begin{cases}
           x_1(t)=\varphi_1(t)\\
           x_2(t) = \varphi_2(t)\\
   &\;\;\vdots \notag \\
           x_n(t) = \varphi_n(t)

    \end{cases}\,.
\end{equation}

}


l'immagine della curva è chiamata sostegno della curva:

\[
    \varphi(I) \subset \mathbb{R}^{n}
\]

\textbf{Esempi} 

Curve cartesiane

Sia $f:[a,b]\rightarrow \mathbb{R}$ continua il suo grafico è il sostegno della curva piana data da:

\[
    \varphi:[a,b]\rightarrow \mathbb{R}^{2}
\]

\[
    \varphi(t) = (\varphi_1(t), \varphi_2(t))
\]

definita da:

\begin{equation}
    \begin{cases}
           x_1(t) = t\\
           x_2(t) = f(t) 
    \end{cases}\,.
\end{equation}


\defn{Curva regolare}{
Una curva si dice regolare se l'applicazione $\varphi$ è di classe $C^{1}$ (le derivate prime sono continue) e $\varphi'(t) \neq 0$

In particolare $\varphi'(t) \neq 0$ significa che il vettore:

\[
    (\varphi_1'(t),...,\varphi_n'(t)) = \varphi'(t)
\]

non ha mai tutte le componenti contemporaneamente nulle.
}


\textbf{Esempio} 

Se $f \in \mathbb{C}^{1}([a,b])$ e $\varphi'(t) = (1,f'(t))$

allora la retta tangente alla curva nel punto $\varphi(t_0)$ è proprio la retta tangente al grafico di f nel punto $(x_0,f(x_0))=(t_0,f(t))$

Se $\varphi(t)$ è regolare allora il vettore $\varphi'(t_0)$ si chiama vettore tangente alla curva nel punto $\varphi(t_0) \in \mathbb{R}^{3}$

\textbf{Esempio di due curve con stesso sostegno} 

Consideriamo le due applicazioni a valori vettoriali:

\[
    \underbrace{\varphi:[0,2\pi] \rightarrow \mathbb{R}^{2}}_{\varphi(t) = (\varphi_1(t),\varphi_2(t))}
\]

\[
    \underbrace{\psi: [0,4\pi] \rightarrow  \mathbb{R}^{2}}_{\psi(t) = (\psi_1(t),\psi_2(t))}
\]


\begin{equation}
    \begin{cases}
           \varphi_1(t) = \cos t\\
           \varphi_2(t) = \sin t
    \end{cases}\,.
\end{equation}


\begin{equation}
    \begin{cases}
           \psi_1(t) = \cos t\\
           \psi_2(t) = \sin t
    \end{cases}\,.
\end{equation}

Il sostegno delle due curve è lo stesso vanno entrambe a $\mathbb{R}^{2}$

Le curve però sono diverse perché il dominio è diverso.


Possiamo scrivere le derivati parziali con notazione vettoriale.

\[
    f: A\subseteq \mathbb{R}^{n}\rightarrow \mathbb{R} \text{ con A aperto}
\]

quindi posso scrivere la derivata parziale:

\[
f_{x_i} (\bar{x} ) = \frac{\partial f}{\partial x_i}(\bar{x} )
\]

nel modo seguente, definisco $\forall i = 1,...,n$:

\[
    \bar{e_i}  = (0,...0,1,0,....,0) 
\]

quindi la derivata è data dal seguente limite, purché esista e sia finito:

\[
\lim_{ h \to 0 } \frac{f(\bar{x} + h \bar{e_i}) - f(\bar{x} )}{h} = f_{x_i}(\bar{x} )
\]

espandendo:

\[
    f( \bar{x} +h \bar{e_i} ) = f( x_1,...,x_{i-1},x_i+h , x_{i+1},...,x_n)
\]

vediamo che dipende solo da h:

\[
    g(h) = f(\bar{x} ,h \bar{e_i} )
\]

Se g è derivabile allora si ha:

\[
    g'(0) = \frac{\partial f}{\partial x_i}(\bar{x} )
\]

perché espandendo:

\[
    g'(0) = \lim_{ h \to 0 } \frac{g(h) - g(0)}{h-0} = \lim_{ h \to 0 } \frac{g(\bar{x} +h \bar{e_i} ) - f(\bar{x} )}{h}
\]

se io ho $\bar{v} $ direzione in $\mathbb{R}^{n}$ di modulo 1:

\[
    |\bar{v} | = \sqrt{\sum^{n}_{i=1} v_i^{2}}
\]

Derivata direzionale:

\[
    D_v f(\bar{x} ) = \lim_{ h \to 0 } \frac{f(\bar{x} +h \bar{v} ) -f(\bar{x} )}{h}
\]


\textbf{Esempio} 

\[
    f: A \subseteq \mathbb{R}^{3} \rightarrow  \mathbb{R}
\]

\[
    \bar{v} = (a,b,c) \in \mathbb{R}^{3}
\]

\[
    a^{2}+b^{2}+c^{2}= 1\ (||\bar{v} || =1)
\]

Calcoliamo la derivata direzionale:

\[
    D_v f(\bar{x_0} ) = \lim_{ h \to 0 } \frac{f(\bar{x_0} + h \bar{v} ) -f(\bar{x_0} )}{h}
\]

dove:

\[
    \bar{x_0} = (x_0,y_0,z_0)
\]

\[
    D_v f(P_0) = D_v f(\bar{x_0} ) = \lim_{ h \to 0 } \frac{f(x_0+ha,y_0+hb,z_0+hc) - f(x_0,y_0,z_0)}{h}
\]

questo esiste se il limite esiste ed è finito.

dire che quindi $f$ è derivabile in $\bar{x} \in A$ è come dire che esiste il vettore gradiente:

\[
   \nabla f(\bar{x} ) 
\]

\end{document}

