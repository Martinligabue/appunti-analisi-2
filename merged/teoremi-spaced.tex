\documentclass[11pt]{article}


\usepackage[italian]{babel}
\usepackage[utf8]{inputenc}	% Para caracteres en español
\usepackage{amsmath,amsthm,amsfonts,amssymb,amscd}
\usepackage{multirow,booktabs}
\usepackage[table]{xcolor}
\usepackage{fullpage}
\usepackage{lastpage}
\usepackage{enumitem}
\usepackage{fancyhdr}
\usepackage{mathrsfs}
\usepackage{wrapfig}
\usepackage{graphicx}
\graphicspath{ {./images/} }
\usepackage{setspace}
\usepackage{calc}
\usepackage{multicol}
\usepackage{bookmark}
\usepackage[normalem]{ulem}
\usepackage{cancel}
\usepackage[margin=3cm]{geometry}
\usepackage{amsmath}
\newlength{\tabcont}
\setlength{\parindent}{0.0in}
\setlength{\parskip}{0.05in}
\usepackage{empheq}
\usepackage{framed}
\usepackage[most]{tcolorbox}
\usepackage{xcolor}
\usepackage{FiraSans}
\usepackage{esint}
\usepackage{listings}
\usetikzlibrary{shadings,intersections,calc,trees,positioning,arrows,fit,shapes}
\usepackage{centernot}

\usepackage{import}
\usepackage{pdfpages}
\usepackage{transparent}
\usepackage{mathrsfs}
\usetikzlibrary{arrows}


\usepackage{pgfplots}
\pgfplotsset{compat=1.18}
\usepackage[framemethod=TikZ]{mdframed}
\usepackage{hyperref}
\hypersetup{
    colorlinks,
    citecolor=black,
    filecolor=black,
    linkcolor=black,
    urlcolor=black
}

\usepackage{subfiles}

\geometry{margin=1in, headsep=0.25in}
\setcounter{secnumdepth}{0}

\newlength{\totbarheight}

\newlength{\bardepth}

%%%%%%%%%%%%%%%%%%%%%%%%%%%%%%
% Theorem, Corollari ecc.
%%%%%%%%%%%%%%%%%%%%%%%%%%%%%%

\mdfsetup{skipabove=\topskip,skipbelow=\topskip}
%%% upto here
\newcounter{theo}[section]
\newenvironment{theo}[1][]{%
\stepcounter{theo}%
\ifstrempty{#1}%
 {\mdfsetup{%
   frametitle={%
    \tikz[baseline=(current bounding box.east),outer sep=0pt]
    \node[anchor=east,rectangle,fill=blue!20]
         {\strut Teorema~\thetheo};}}
 }%
{\mdfsetup{%
  frametitle={%
   \tikz[baseline=(current bounding box.east),outer sep=0pt]
   \node[anchor=east,rectangle,fill=blue!20]
        {\strut Teorema~\thetheo:~#1};}}%
 }%
\mdfsetup{innertopmargin=10pt,linecolor=blue!20,%
       linewidth=2pt,topline=true,
       frametitleaboveskip=\dimexpr-\ht\strutbox\relax,}
   \begin{mdframed}[]\relax%
}
{\end{mdframed}}

\newcounter{lemm}[section]
\newenvironment{lemm}[1][]{%
\stepcounter{lemm}%
\ifstrempty{#1}%
 {\mdfsetup{%
   frametitle={%
    \tikz[baseline=(current bounding box.east),outer sep=0pt]
    \node[anchor=east,rectangle,fill=blue!20]
         {\strut Lemma~\thelemm};}}
 }%
{\mdfsetup{%
  frametitle={%
   \tikz[baseline=(current bounding box.east),outer sep=0pt]
   \node[anchor=east,rectangle,fill=blue!20]
        {\strut Lemma~\thelemm:~#1};}}%
 }%
\mdfsetup{innertopmargin=10pt,linecolor=blue!20,%
       linewidth=2pt,topline=true,
       frametitleaboveskip=\dimexpr-\ht\strutbox\relax,}
   \begin{mdframed}[]\relax%
}
{\end{mdframed}}


\newcounter{corollario}[section]
\newenvironment{corollario}[1][]{%
\stepcounter{corollario}%
\ifstrempty{#1}%
 {\mdfsetup{%
   frametitle={%
    \tikz[baseline=(current bounding box.east),outer sep=0pt]
    \node[anchor=east,rectangle,fill=blue!20]
         {\strut Corollario~\thecorollario};}}
 }%
{\mdfsetup{%
  frametitle={%
   \tikz[baseline=(current bounding box.east),outer sep=0pt]
   \node[anchor=east,rectangle,fill=blue!20]
        {\strut Corollario~\thecorollario:~#1};}}%
 }%
\mdfsetup{innertopmargin=10pt,linecolor=blue!20,%
       linewidth=2pt,topline=true,
       frametitleaboveskip=\dimexpr-\ht\strutbox\relax,}
   \begin{mdframed}[]\relax%
}
{\end{mdframed}}

\newcounter{definizione}[section]
\newenvironment{definizione}[1][]{%
\stepcounter{definizione}%
\ifstrempty{#1}%
 {\mdfsetup{%
   frametitle={%
    \tikz[baseline=(current bounding box.east),outer sep=0pt]
    \node[anchor=east,rectangle,fill=blue!20]
         {\strut Definizione~\thedefinizione};}}
 }%
{\mdfsetup{%
  frametitle={%
   \tikz[baseline=(current bounding box.east),outer sep=0pt]
   \node[anchor=east,rectangle,fill=blue!20]
        {\strut Definizione:~#1};}}%
 }%
\mdfsetup{innertopmargin=10pt,linecolor=blue!20,%
       linewidth=2pt,topline=true,
       frametitleaboveskip=\dimexpr-\ht\strutbox\relax,}
   \begin{mdframed}[]\relax%
}
{\end{mdframed}}

\newcounter{propo}[section]
\newenvironment{propo}[1][]{%
\stepcounter{propo}%
\ifstrempty{#1}%
 {\mdfsetup{%
   frametitle={%
    \tikz[baseline=(current bounding box.east),outer sep=0pt]
    \node[anchor=east,rectangle,fill=blue!20]
         {\strut Proposizione~\thepropo};}}
 }%
{\mdfsetup{%
  frametitle={%
   \tikz[baseline=(current bounding box.east),outer sep=0pt]
   \node[anchor=east,rectangle,fill=blue!20]
        {\strut Proposizione~\thepropo~ - #1};}}%
 }%
\mdfsetup{innertopmargin=10pt,linecolor=blue!20,%
       linewidth=2pt,topline=true,
       frametitleaboveskip=\dimexpr-\ht\strutbox\relax,}
   \begin{mdframed}[]\relax%
}
{\end{mdframed}}

\newcounter{ese}[section]
\newenvironment{ese}[1][]{%
\stepcounter{ese}%
\ifstrempty{#1}%
 {\mdfsetup{%
   frametitle={%
    \tikz[baseline=(current bounding box.east),outer sep=0pt]
    \node[anchor=east,rectangle,fill=blue!20]
         {\strut Esempio~\theese};}}
 }%
{\mdfsetup{%
  frametitle={%
   \tikz[baseline=(current bounding box.east),outer sep=0pt]
   \node[anchor=east,rectangle,fill=blue!20]
        {\strut Esempio~\theese~#1};}}%
 }%
\mdfsetup{innertopmargin=10pt,linecolor=blue!20,%
       linewidth=2pt,topline=true,
       frametitleaboveskip=\dimexpr-\ht\strutbox\relax,}
   \begin{mdframed}[]\relax%
}
{\end{mdframed}}

%%%%%%%%%%%%%%%%%%%%%%%%%%%%%%
% SELF MADE COMMANDS
%%%%%%%%%%%%%%%%%%%%%%%%%%%%%%


\newcommand*\Eval[3]{\left.#1\right\rvert_{#2}^{#3}}
\newcommand{\teorema}[2]{\begin{theo}[#1]{}{}#2\end{theo}}
\newcommand{\cor}[2]{\begin{corollario}[#1]{}{}#2\end{corollario}}
\newcommand{\lemma}[2]{\begin{lem}[#1]{}{}#2\end{lem}}
\newcommand{\proposizione}[2]{\begin{propo}[#1]{}{}#2\end{propo}}
% \newcommand{\mprop}[2]{\begin{Prop}{#1}{}#2\end{Prop}}
% \newcommand{\clm}[3]{\begin{claim}{#1}{#2}#3\end{claim}}
% \newcommand{\wc}[2]{\begin{wconc}{#1}{}\setlength{\parindent}{1cm}#2\end{wconc}}
% \newcommand{\thmcon}[1]{\begin{Theoremcon}{#1}\end{Theoremcon}}
% \newcommand{\ex}[2]{\begin{Example}{#1}{}#2\end{Example}}
\newcommand{\defn}[2]{\begin{definizione}[#1]{}{}#2\end{definizione}}
\newcommand{\esempio}[2]{\begin{ese}[#1]{}{}#2\end{ese}}
% \newcommand{\dfnc}[2]{\begin{definition}[colbacktitle=red!75!black]{#1}{}#2\end{definition}}
% \newcommand{\qs}[2]{\begin{question}{#1}{}#2\end{question}}
% \newcommand{\pf}[2]{\begin{myproof}[#1]#2\end{myproof}}
% \newcommand{\nt}[1]{\begin{note}#1\end{note}}


\newcommand{\incfig}[2][1]{%
    \def\svgwidth{#1\columnwidth}
    \import{\subfix{./figures/}}{#2.pdf_tex}
}


\newcommand\DrawBlock[3]{
\ifx#1b\relax
  \path[draw]
    (lm\the\numexpr#2-1\relax) -- ++(0,0,#3) coordinate (blocklf)
    (bm\the\numexpr#2-1\relax) -- ++(0,0,#3) coordinate (blocklb)
    (lm#2) -- ++(0,0,#3) coordinate (blockrf)
    (bm#2) -- ++(0,0,#3) coordinate (blockrb);
  \filldraw[fill=white,draw=black]
    (lm\the\numexpr#2-1\relax) -- (blocklf) -- (blocklb) -- (blockrb) -- (blockrf) -- (lm#2);
\else  
  \ifx#1f\relax
    \path[draw]
      (fm\the\numexpr#2-1\relax) -- ++(0,0,#3) coordinate (blocklf)
      (lm\the\numexpr#2-1\relax) -- ++(0,0,#3) coordinate (blocklb)
      (fm#2) -- ++(0,0,#3) coordinate (blockrf)
      (lm#2) -- ++(0,0,#3) coordinate (blockrb);
    \filldraw[fill=white,draw=black]
      (fm\the\numexpr#2-1\relax) -- (blocklf) -- (blocklb) -- (blockrb) -- (blockrf) -- (fm#2);
  \fi
\fi
\draw (blocklf) -- (blockrf);
}

\pdfsuppresswarningpagegroup=1


\begin{document}
\teorema{}{L'integrale generale di (1) in $[a,b]$ è dato dalla somma dell'integrale generale dell'omogenea associata (2)
con un integrale particolare noto di (1)
\[
\int_{{}}^{{}} {gen} (1) = \int_{{}}^{{}} {gen(2)}  + \int_{{}}^{{}} {particolare} (1)
\]
}
\begin{proof}


Sia $y(x)$ una soluzione qualsiasi di (1) ($y(x)$ appartiene all'integrale generale di (1))
e sia $\bar y(x)$ una soluzione particolare (nota) di (1). Voglio far vedere è che la loro differenza è una soluzione qualsiasi di (2)

Dunque per ipotesi n ha che:
\[
    y'(x)+a(x)y(x) = f(x), \forall x \in [a,b]
\]

\[
    \bar y'(x) + a(x) \bar y(x) = f(x)
\]

Entrambe soddisfano la (1)

Sottraggo membro a membro le due:

\[
    y'(x)-\bar y'(x) + a(x)y(x) - a(x) \bar y(x) = f(x) - f(x)
\]

\[
    y'(x)-\bar y'(x) + a(x)[y(x) - \bar y(x)]=0
\]

Si può scrivere anche (le derivate raccolte):
\[
    [y(x)-\bar y(x)]' + a(x)[y(x) - \bar y(x)]=0
\]

E dunque  la funzione $y(x) - \bar y(x) = z(x)$ è soluzione di (2)
Quindi:
\[
    y(x) = \bar y(x) + z(x)
\]

Viceversa se $z(x)$ è una qualsiasi soluzione di (2) e $\bar y(x)$ è una soluzione particolare di (1) 
voglio mostrare che la loro somma è soluzione di (1)

Pongo:
\[
    y(x) = z(x) + \bar y(x)
\]

Devo mostrare che $y(x)$ verifica (1)

sapendo che:
\[
    z'(x) + a(x)z(x) = 0
\]

\[
    \bar y'(x) + a(x) \bar y(x) = f(x)
\]

\[
    y'(x) = (z(x) + \bar y(x) )' = z'(x) + \bar y'(x) =
    -a(x)z(x)-a(x)\bar y(x) + f(x) = -a(x) [z(x) + \bar y(x)] +f(x)
\]

E quindi ho dimostrato che:
\[
    y'(x) = -a(x)y(x) + f(x)
\]

\[
    y'(x) +  a(x)y(x) = f(x)
\]

\[
    y(x) = z(x) + \bar y(x)
\]

\end{proof}
\newpage
\teorema{}{Siano $f$ e $g$ continue (sugli opportuni domini) allora:

    \begin{itemize}
        \item $f+g , f\cdot g$ sono continue 
        \item se $g \neq 0$ allora $\frac{f}{g}$ è continua
        \item se $g>0$ allora $f^{g}$ è continua
        \item la funzione comporta $g \circ f$ è continua (dove è definita)
    \end{itemize}
}
\teorema{}{Sia $f: D \subset \mathbb{R}^{2} \rightarrow  \mathbb{R}$ e sia $P_0=(x_0,y_0) \in D$ allora:

    \[
        \lim_{ (x,y) \to (x_0,y_0) } f(x,y)  = l \Leftrightarrow \lim_{ \rho \to 0^{+} } f(x_0+\rho cos \theta, y_0+\rho sin \theta) = l
    \]

    uniformemente rispetto a $\theta$

}
\begin{proof}
       per far vedere che vale il limite è sufficiente mostrare che esiste una funzione $g$ che dipende solo da $\rho$ (non negativa) $g(\rho)\ge 0$ tale che:

       \[
        |f(x_0+\rho cos\theta, y_0+\rho sin \theta ) -l| \le  g(\rho)
       \]

       dove $g(\rho) \rightarrow 0$ per $\rho \rightarrow 0^{+}$

       e poi faccio vedere che quindi (per il teorema dei due carabinieri):

       \[
        0\le f(x_0+\rho cos\theta, y_0+\rho sin \theta ) -l \le  g(\rho) = 0
       \]

\end{proof}
\newpage
\teorema{}{ $f$ è differenziabile in $\bar{x_0} \subset A \rightarrow f$ è continua in $\bar{x_0} $ }
\begin{proof}
     Per dimostrare che $f$ è continua in $\bar{x_0} $ devo far vedere che:

     \[
         \lim_{ h \to 0 } f(x+h) = f( \bar{x_0} ) 
     \]

     \[
        \lim_{ h \to 0 } f(x_0 + h ) \overset{\text{poiché $f$ è differenziabile}}{=} \lim_{ h \to 0 } [f(x_0) + \langle \nabla f(x_0),h \rangle + \underbrace{o(|h|)}_\text{$\rightarrow 0$}]
     \]

     usiamo Cauchy-Schwarz:

     \[
        \langle \nabla f(x_0),h \rangle   \le  \underbrace{\langle \nabla f(x_0),h \rangle}_\text{valore assoluto}| \le  \underbrace{|\nabla f(x_0)|}_\text{lunghezza del vettore} | \underbrace{h}_\text{$\rightarrow 0$}| \overset{\text{numero moltiplicato $0$}}{=} 0
     \]

     quindi abbiamo che:

     \[
         \lim_{ h \to 0 } [f(x_0) + \underbrace{\langle \nabla f(x_0),h \rangle}_\text{$\rightarrow 0$} + \underbrace{o(|h|)}_\text{$\rightarrow 0$}] = f(x_0)
     \]
\end{proof}
\newpage
\teorema{Teorema del differenziale}{ Sia $f: A \subseteq \mathbb{R}^{n} \rightarrow \mathbb{R}$, $A$ aperto, derivabile in $A$. 

    Se le derivate parziali $f_{x_1}, \ldots, f_{x_n}$ sono continue in $\bar{x} \in A$ allora $f$ è differenziabile in $\bar{x} $

}
\begin{proof}
       Per $n=2$, $f = f(x,y)$ con $(x,y) \in A \subseteq \mathbb{R}^{2}$ 

       $\bar{h} = (h,k)$:

       \[
           f(x+h,y+k) - f(x,y) \overset{\text{aggiungo e tolgo $f(x,y+k)$}}{=} 
       \]

       \[
           = f(x+h,y+k) + f(x,y+k) - f(x,y+k) - f(x,y) = 
       \]

       \[
           = [f(x+h,y+k) - f(x,y+k) ] + [f(x,y+k) - f(x,y)] \overset{\text{uso Lagrange a ognuna delle funzioni}}{=}
       \]

       Applico due volte il teorema di Lagrange sugli intervalli di estremi $x,x+h$ e $y,y+k$:

       \[
           \exists x_1 \in  \text{ intervallo aperto di estremi } x,x+h
       \]

       \[
           \exists y_1 \in  \text{ intervallo aperto di estremi } y,y+k
       \]

       \[
           = \frac{\partial f}{\partial x}(x_1,y+k) (\cancel{x} +h \cancel{-x}) + \frac{\partial f}{\partial y}(x,y_1) ( \cancel{y} + k \cancel{-y}) = 
       \]

       \[
           =\frac{\partial f}{\partial x}(x_1,y+k) h + \frac{\partial f}{\partial y}(x,y_1)k 
       \]

       a questo punto faccio vedere la definizione di differenziabilità e sostituisco quello sopra:

       \[
           \frac{f(x+h,y+k) - f(x,y) - f_x(x,y)h - f_y(x,y) k}{\sqrt{h^{2}+k^{2}}} = 
       \]

       \[
           = \frac{f_x(x_1,y+k) h + f_y(x,y_1) k -f_x(x,y) - f_y(x,y) k}{\sqrt{h^{2}+k^{2}}} = 
       \]

       \[
           = \frac{[f_x(x_1,y+k) - f_x(x,y)] h + [f_y(x,y_1) - f_y(x,y)] k}{\sqrt{h^{2}+k^{2}}}
       \]

       metto tutto in valore assoluto e maggioro:

       \[
           = |\frac{[f_x(x_1,y+k) - f_x(x,y)] h + [f_y(x,y_1) - f_y(x,y)] k}{\sqrt{h^{2}+k^{2}}}| \le 
       \]

       \[
       \le |f_x(x_1,y+k) - f_x(x,y_1)| \frac{|h|}{\sqrt{h^{2}+k^{2}}} + | f_y(x,y_1) - f_y(x,y)| \frac{|k|}{\sqrt{h^{2}+k^{2}}} \le 
       \]

       \[
            \le  |f_x(x_1,y+k) - f_x(x,y) | + |f_y(x,y_1) - f_y(x,y) |
       \]

       vediamo che succede quando $(h,k) \rightarrow (0,0)$:

        $(x_1,y_1) \rightarrow (x,y)$ le funzioni $f_x$ e $f_y$ sono continue in $(x,y)$

        dunque se passo al limite:

        \[
            |\underbrace{f_x(x,y) - f_x(x,y)}_\text{$\rightarrow 0$} | + |\underbrace{f_y(x,y) - f_y(x,y)}_\text{$\rightarrow 0$} |
        \]

\end{proof}
\newpage
\teorema{Regola della catena}{

consideriamo una curva continua $\gamma : [-1,1] \subset \mathbb{R} \rightarrow A\subseteq \mathbb{R}^{n}\ curva$ e supponiamo $\gamma(t)$ vettore (di $n$ componenti) sia derivabile, cioè:

\[
    \gamma(t) = (\gamma_1(t),\ldots,\gamma_n(t) )
\]

esiste:

\[
    \gamma'(t) = (\gamma_1'(t),\ldots,\gamma_n'(t))
\]

e supponiamo che $\gamma(0) = x_0 \in A$ e $\gamma'(0) = \bar{v} \in \mathbb{R}^{n}$, 


allora se $f: A \rightarrow \mathbb{R}$ è differenziabile in $x_0$, la funzione composta $F \rightarrow f(\gamma(t))$ da $[-1,1] \rightarrow \mathbb{R}$:

\[
    F= f \circ g : [-1,1] \rightarrow \mathbb{R}
\]

\[
    F(t) = (f \circ g) (t) = f(\gamma(t))
\]

è differenziabile in $0$:

\[
    F'(0) = \frac{\partial F}{\partial t} (0) = \frac{\partial (f \circ g)}{\partial t}(0) = \langle \nabla f(x_0), \underbrace{\gamma'(0)}_\text{direzione $\bar{v} $} \rangle
\]

Questo si chiama \textbf{teorema delle derivate delle funzioni composte} o regola della catena.

}
\begin{proof}
       Dimostrato dalle considerazioni fatte fino a ora.    
\end{proof}
\newpage
\teorema{Derivazione della funzione composta}{
    Supponiamo $\gamma(t)$ derivabile $\forall t \in I$ ovvero $\gamma'(t)$ è definito $\forall t \in I$ con $\gamma'(t) = (\gamma_1'(t),\ldots,\gamma_n'(t))$ e supponiamo che $f$ sia differenziabile in $\gamma(t) \in A$ (data $f: A \subseteq \mathbb{R}^{n}\rightarrow \mathbb{R}$) allora la funzione composta $F= f \circ \gamma: I \rightarrow \mathbb{R}$ è derivabile in t.


    Inoltre:

    \[
        F'(t) = \langle \nabla f(\gamma(t)), \gamma'(t) \rangle = \sum^{n}_{i=1} \frac{\partial f}{\partial x_i}(\gamma(t))\gamma_i'(t)
    \]

}
\begin{proof}
       Inizio dimostrando che $f$ è derivabile, ovvero che esiste finito il limite del rapporto incrementale:

       \[
           \frac{F(t+h) - F(t)}{h} = \frac{F(\gamma(t+h)) - F(\gamma(t))}{h} = \langle \nabla f(\gamma(t+h)), \underbrace{\frac{\gamma(t+h) - \gamma(t)}{h}}_\text{$1$} \rangle + \underbrace{\frac{o(\gamma(t+h) - \gamma(t)|)}{h}}_\text{$2$}
       \]

       quando $h \rightarrow 0$:

       \begin{enumerate}
        \item
            \[\lim_{ h \to 0 } \frac{\gamma(t+h) - \gamma(t)}{h} = \gamma'(t) \hspace{19em}\]

        \item
        \[
            \lim_{ h \to 0 } \frac{o(|\gamma(t+h) - \gamma(t)|)}{h} {=} \lim_{ h \to 0 } {\frac{|o(|\gamma(t+h) - \gamma(t)|)|}{|h|}} \cdot {\frac{| \gamma(t+h) - \gamma(t)|}{|\gamma(t+h) - \gamma(t)|}} =
        \]
        \[
            = \lim_{ h \to 0 } \underbrace{\frac{|o(|\gamma(t+h) - \gamma(t)|)|}{|\gamma(t+h) - \gamma(t)|}}_\text{$0$ per definizione di o-piccolo} \cdot \underbrace{\frac{| \gamma(t+h) - \gamma(t)|}{|h|}}_\text{quantità finita} = 0
        \]
    Quantità finita perché:

    \[
        \lim_{ h \to 0 } \frac{|\gamma(t+h) - \gamma(t)|}{|h|} = \lim_{ h \to 0 } ( \sum^{n}_{i=1} (\frac{\gamma_i(t+h) - \gamma_i(t)}{h})^{2})^{ \frac{1}{2}} =
    \]

    \[
        = \sqrt{\sum^{n}_{i=1} ( \gamma_i'(t))^{2}} = \underbrace{| \gamma'(t)|}_\text{lunghezza di un vettore} >0
    \]

       \end{enumerate}

\end{proof}
\newpage
\teorema{Formula del gradiente}{Se $f(x,y)$ è differenziabile in $P=(x,y)$ allora $f$ ammette derivate direzionali in $(x,y)$ per ogni direzione. Inoltre per ogni versore $\bar{v}= (a,b)$, vale:

    \[
        D_{\overrightarrow{v} }f(x,y) = \langle \nabla f(x,y), \overrightarrow{v}  \rangle = \frac{\partial f}{\partial x}(x,y)\cdot a + \frac{\partial f}{\partial y}(x,y) \cdot b
    \]

}
\begin{proof}
    \[
    e ^{-A(x)} = -a(x) e ^{-A(x)}
\]

ovvero

\[
    (e ^{-A(x)})'+a(x) e ^{-A(x)}=0
\]
    
\end{proof}
\newpage
\begin{proof}
    Poiché $\bar y(x)$ è soluzione di (1) si ha che $\bar y'(x)+a(x) \bar y(x)=f(x)$ da cui sostituendo $\bar y(x) = c(x) e ^{-A(x)}$:

    \[
        (c(x) e ^{-A(x)})'+ a(x) c(x) e ^{-A(x)} = f(x)
    \]
    
    Deriviamo:

    \[
        c'(x) e ^{-A(x)} \cancel{- c(x) a(x) e ^{-A(x)}} \cancel{+ a(x) c(x) e ^{-A(x)}}= f(x)
    \]
    
    semplifico

    \[
        c'(x) e ^{-A(x)} = f(x)
    \]

    \[
        c'(x) = f(x) e ^{A(x)} \rightarrow c(x) = \int_{{}}^{{}} {f(x) e ^{A(x)}} \: d{} {}
    \]

    e dunque:

    \[
        \bar y(x) = e ^{-A(x)} \int_{{}}^{{}} {f(x) e ^{A(x)}} \: d{x} {}
    \]

    Cioè l'integrale particolare
\end{proof}
\newpage
\teorema{Schwarz}{ Sia $f: A \subseteq \mathbb{R}^{n} \rightarrow \mathbb{R}$, $A$ aperto

    e supponiamo che la $f$ sia derivabile due volte su $A$, quindi esistono tutte le $f_{x_i x_j}$, $\forall i,j=1,\ldots,n$ e sia $\bar{x_0} \in A$.


    Se le $f_{x_i x_j}$, e le $f_{x_j x_i}$ con $i \neq j$ sono \textbf{continue}  in $x_0$, allora:

    \[
        f_{x_i x_j}(x_0) = f_{x_j x_i}(x_0)
    \]

}
\begin{proof}
       Sia $P_0=(x_0,y_0)$ e $P=(x,y)$ un punto qualsiasi su $A$, con $P \neq P_0$ (quindi $x \neq x_0, y \neq y_0$)

       Consideriamo il valore della funzione nei punti:

       \[
           f(x_0,y_0) = f(x,y_0), f(x,y), f(x_0,y)
       \]

       \[
           \underbrace{F(x)}_\text{dipende solo da $x$ ($y$ fissato)} = f(x,y) - f(x,y_0)
       \]

       \[
           \underbrace{G(y)}_\text{dipende solo da $y$ ($x$ fissato)}  = f(x,y) - f(x_0,y)
       \]

       Applico il teorema di Lagrange (teorema del valore intermedio)

       Lagrange a $F(x)$ nell'intervallo di estremi $x_0$, $x$, si ha che esiste un elemento $x_1$ in questo intervallo per cui:

       \[
           F(x)- F(x_0) = F'(x_1) (x-x_0) = [f_x(x_1,y) - f_x(x_1,y_0) ] (x- x_0)  \overset{\text{applico Lagrange due volte come spiegato sotto}}{=}
       \]

       Sappiamo che $f$ è derivabile due volte, posso quindi applicare Lagrange a $f_x(x_1,y)$ nell'intervallo di estremi $y_0,y$. Quindi $\exists y_1$ nell'intervallo tale che:

       \[
           f_x(x_1,y) - f_x(x_1,y_0) = \frac{\partial }{\partial y}(f_x(x_1,y_1))(y-y_0) = f_{xy}(x_1,y_1)(y-y_0)
       \]

       quindi la nostra espressione diventa:

       \[
           = f_{xy}(x_1,y_1)(x-x_0) (y-y_0)
       \]

       quindi abbiamo fatto vedere che:

       \[
           F(x) -F(x_0) = f_{xy}(x_1,y_1) (x-x_0)(y-y_0)
       \]

       Analogamente per $G(y)$ applico Lagrange quindi $\exists y_2$ nell'intervallo di estremi $y,y_0$ tale che:

       \[
           G(y) -G(y_0) = G'(y_2) (y-y_0) = [f_y(x,y_2) - f_y(x_0,y_2)] (y-y_0) \overset{\text{applico di nuovo Lagrange}}{=}
       \]

       applico quindi Lagrange a $f_y(x,y_2)$ nell'intervallo di estremi $x,x_0$ quindi $\exists x_2$ in questo intervallo:

       \[
           f_y(x,y_2) - f_y(x_0,y_2) = f_{yx}(x_2,y_2)(x-x_0)
       \]

       infine quindi:

       \[
           = f_{yx}(x_2,y_2)(x-x_0)(y-y_0)
       \]


       Notiamo che:

       \[
           F(x) - F(x_0) = G(x) -G(x_0)
       \]

       \[
           G(x) - G(x_0) = f(x,y) - f(x_0,y) - (f(x,y_0) -f(x_0,y_0))
       \] 

       Le due espressioni sono quindi uguali, di conseguenza anche le espressioni ottenute precedentemente

       Essendo $F(x) - F(x_0) = G(x) - G(x_0)$, segue che:

       \[
           f_{xy}(x_1,y_1)(x-x_0)(y-y_0) = f_{yx}(x_2,y_2)(x-x_0)(y-y_0)
       \]

       noi sappiamo che $(x,y) \neq (x_0,y_0)$ per ipotesi, quindi deve essere che:

       \[
           f_{xy}(x_1,y_1) = f_{yx}(x_2,y_2)
       \]

       $(x_1,y_1)$ e $(x_2,y_2)$ stanno nell'intervallo del rettangolo tratteggiato:


       \begin{center}
       \begin{tikzpicture}
           \draw[->] (-3, 0) -- (4.2, 0) node[right] {$x$};
           \draw[->] (0, -3) -- (0, 4.2) node[above] {$y$};
           \draw[densely dotted] (1,1) -- (2,1) node[right] {$P_0$};
           \draw[densely dotted] (1,1) -- (1,2);
           \draw[densely dotted] (2,1) -- (2,2);
           \draw[densely dotted] (1,2) -- (2,2);
       \end{tikzpicture}
       \end{center}

       Passando al limite (per $P \rightarrow P_0$) succede che:

       \[
           (x,y) \rightarrow (x_0,y_0)
       \]

       \[
           (x,y_0) \rightarrow (x_0,y_0)
       \]

       \[
           (x_2,y_2) \rightarrow (x_0,y_0)
       \]

       ed essendo la funzione $f_{xy},f_{yx}$ continue in $P_0=(x_0,y_0)$, si ha:

       \[
           f_{xy}(x_0,y_0) = f_{yx}(x_0,y_0)
       \]


\end{proof}
\newpage
\begin{proof}
       Immediata applicando Taylor con k=1:

       \[
           F(1) = F(0) + F'(\theta)
       \] 
\end{proof}
\newpage
\begin{proof}
       Di nuovo viene da Taylor per $k=2$:

       \[
           F(1) = F(0) + F'(0) + \frac{F''(\theta)}{2}
       \]
\end{proof}
\newpage
\teorema{Teorema di Fermat per funzioni in più variabili}{

    Sia $f:D \subset \mathbb{R}^{n} \rightarrow  \mathbb{R}$, sia $x_0 \in A$ punto di estremo locale per $f$. Se $f$ è differenziabile in $x_0$ allora:

    \[
        \nabla f(x_0) = 0
    \]

}
\begin{proof}
    Supponiamo $x_0$ punto di massimo relativo. Allora $x_0$ è punto di massimo relativo anche per la restrizione di $f$ lungo una qualsiasi retta passante per $x_0$. Dunque consideriamo $v \in \mathbb{R}^{n}$ la direzione di tale retta, quindi $x_0 + tv$ sono i punti su tale retta.

    La funzione in \textbf{una} variabile:

        \[
        F(t) = f(x_0 + tv)
    \] 

    è definita in un intorno di $t=0$ e per ipotesi, siccome $x_0$ è punto di massimo per $f$, allora $t=0$ è punto di massimo per $F$.

    Per il Teorema di Fermat (in una variabile) su $F$ si ha:

    \[
        F'(0) = \langle \nabla f(x_0),v \rangle = 0
    \]

    per ogni direzione $v$. Siccome $v \neq \emptyset$ (perché $|v| = 1$), allora necessariamente $\nabla f(x_0) = 0$.

\end{proof}
\newpage
\begin{proof}
       È basata sull'approssimazione al secondo ordine della nostra funzione attraverso la formula di Taylor (al II ordine) col resto di Peano.

       \[
           \underbrace{f(\bar{x_0} + \bar{h} )}_\text{$f(x,y)$} = \underbrace{f(\bar{x_0} )}_\text{$f(x_0,y_0)$} + \langle \underbrace{\nabla f(\bar{x_0} )}_\text{$=0$},\bar{h}  \rangle + \frac{1}{2} \langle Hf(\bar{x_0} )\bar{h} ,\bar{h}   \rangle + o(|\bar{h} |^{2})
       \]

       osserviamo che:

       \begin{equation} \label{quadratica hessiana}
           \frac{1}{2} \langle Hf(\bar{x_0} )\bar{h} ,\bar{h}   \rangle
       \end{equation}

       è un polinomio di II grado in $h,k$ i cui coefficienti sono le derivate seconde quindi ci fornisce il \textbf{segno} 


       per $h \rightarrow 0$ abbiamo:

       \begin{itemize}
        \item $\bar{x_0} +\bar{h} =x$
        \item $\bar{h} = \bar{x} - \bar{x_0} $
        \item $\bar{h} =(h,k)$
       \end{itemize}

       Vediamo cosa succede:

       \[
           f(x,y) - f(x_0,y_0) = \underbrace{\frac{\partial f}{\partial x}(x_0,y_0)}_\text{$=0$}(x-x_0) + \underbrace{\frac{\partial f}{\partial y}(x_0,y_0)}_\text{$=0$} (y-y_0) + \frac{1}{2}[f_{xx}(x_0,y_0)(x-x_0)^{2} + 2 f_{xy}(x_0,y_0) (x-x_0) (y-y_0) 
       \]

       \[
           + f_{yy}(x_0,y_0) (y-y_0)^{2}] + o((x-x_0)^{2}+(y-y_0)^{2})
       \]


       Osserviamo adesso \ref{quadratica hessiana} forma quadratica dell'hessiana in $\bar{h} \in \mathbb{R}^{n}$ è un polinomio di grado 2 omogeneo nelle variabili $h_1,\ldots,h_n$

       Ad ogni forma quadratica è associata una matrice:

       \[
           q(\bar{h} ) = \sum^{n}_{i,j=1} a_{ij} h_i h_j \leftrightarrow \langle A \bar{h} , \bar{h}  \rangle
       \]

       dove $A = (a_{ij})$. Notiamo che tutti i $h_i^{2}$ hanno coefficienti $a_{ii}$ (stanno sulla diagonale).
       
       Nel nostro caso la matrice associata è la matrice Hessiana, che è \textbf{simmetrica} ($a_{ij} = a_{ji}$ per il teorema di Schwarz).

       Quindi per avere il coefficiente di posto $ij$, siccome $a_{ij} = a_{ji} \rightarrow a_{ij} + a_{ji} = 2a_{ij}$, devo dividere il coefficiente per 2.


\textbf{Esempio} 

$n=2$ e $\bar{h} =(h_1,h_2)$

Sappiamo in generale che:

\[
    q(h_1,h_2) = a_{11} h_1^{2} + \underbrace{2a_{12} h_1 h_2}_\text{A simmetrica} + a_{22}h_2^{2}
\]

per una matrice $2\times 2$:

\[
    A= \begin{pmatrix}
        \overbrace{1}^\text{$a_{11}$} & \overbrace{5}^\text{$2a_{12}$}\\
        \underbrace{5}_\text{$2a_{21}$} & \underbrace{4}_\text{$a_{22}$}
    \end{pmatrix} \leftrightarrow 
    q(h_1,h_2) = h_1^{2}+4 h_2^{2}+ 10 h_1 h_2
\]

per una matrice $3\times 3$:

\[
    A = \begin{pmatrix}
    2 & -2 & 5\\
    -2 & 3 & 0\\
    5 & 0 & 4
    \end{pmatrix} \leftrightarrow
    q(\bar{h} ) = 2 h_1^{2}+3 h_2^{2}+ 4 h_3^{2} - 4 h_1 h_2 + 10 h_1 h_3
\]

\newpage

\textbf{Studiamo il segno} 

Adesso studiamo il segno della quadratica hessiana

\defn{}{ $q(\bar{h} )$ si dice \textbf{definita positiva} se $\forall h \neq 0$ si ha $q(\bar{h} ) >0$}

\defn{}{ $q(\bar{h} )$ si dice \textbf{definita negativa} se $\forall h \neq 0$ si ha $q(\bar{h} ) <0$}

\defn{}{ $q(\bar{h} )$ si dice \textbf{indefinita} se $\exists \bar{h_1}, \bar{h_2} \in \mathbb{R}^{2}$ t.c. $q(\bar{h_1} ) <0<q(\bar{h_2} )$ cioè cambia segno}

\textbf{Nota} 

Per studiare il segno possiamo usare anche il segno degli autovalori.

\textbf{Conclusione} 

Vediamo quindi cosa succede:

\begin{itemize}
    \item $det(Hf(x_0,y_0)>0$ e $f_{xx}(x_0,y_0)>0 \rightarrow $ la forma quadratica corrispondente è definita positiva
    \item $det(Hf(x_0,y_0)>0$ e $f_{xx}(x_0,y_0)<0 \rightarrow $ la forma quadratica corrispondente è definita negativa
    \item $det(Hf(x_0,y_0)<0$ $\rightarrow $ la forma quadratica corrispondente è indefinita
\end{itemize}

\end{proof}
\newpage
\teorema{Weistrass}{ Sia $f: K \rightarrow \mathbb{R}$ con $K$ \textbf{limitato e chiuso} di $\mathbb{R}^{2}$ continua allora $f$ è \textbf{limitata} ed assume minimo e massimo su $K$, ovvero:

    \[
        \exists (x_m,y_m),(x_n,y_n) \in K
    \]

    t.c.

    \[
        f(x_m,y_m) \le f(x,y) \le f(x_n, y_n) \forall x,y \in K
    \]

    cioè
    \begin{itemize}
        \item $f(x_m, y_m)$ valore minimo assoluto di $f$ su $K$ 
        \item $f(x_n, y_n)$ valore massimo assoluto di $f$ su $K$
    \end{itemize}


}
\teorema{Moltiplicatori di Lagrange}{ Supponiamo che $f,g \in \mathbb{C}^{1}(A)$ dove $A \subseteq \mathbb{R}^{2}$ aperto.

    Se $(x_0,y_0) \in A$ è un punto di estremo (minimo o massimo) per la $f$ nell'insieme $V$ ($(x_0,y_0)$ è punto di estremo vincolato):

    \[
        V=\{(x,y) \in A, g(x,y) = k\}
    \]

    e vale anche:

    \[
        \underbrace{\nabla g(x_0,y_0) \neq 0}_\text{$(x_0,y_0) \in V$ regolare per $g$}
    \]

    \textbf{allora} esiste $\lambda_0 \in \mathbb{R}$ t.c.:

    \[
        \underbrace{\nabla f(x_0,y_0) = \lambda_0 \nabla g(x_0,y_0)}_\text{equazione vettoriale}
    \]

}
\teorema{}{ Sia $f: R \rightarrow \mathbb{R}$ limitata, allora $f$ è integrabile secondo Riemann su $R ( f \in \mathbb{R}(R)) \Leftrightarrow  \forall \varepsilon$ esiste una suddivisione $D_\varepsilon$ di $R$ per cui:

    \[
        S(f,D_{\varepsilon}) - s(f,D_{\varepsilon}) < \varepsilon
    \]

}
\teorema{di riduzione}{ Sia $f \in \mathbb{R}(R)$ dove $R=[a,b]\times [c,d]$

    \begin{enumerate}
        \item Se, per ogni $y \in [c,d]$, esiste l'integrale:

            \[
                G(y)=\int_{a}^{b} {f(x,y)} \: dx 
            \]

            allora la funzione $y \rightarrow G(y)$ è integrabile in $[c,d]$ e vale la formula:

            \[
                \iint_R {f} = \int_{c}^{d} {G(y)} \: d y = \int_{c}^{d} {\left(\int_{a}^{b} {f(x,y)} \: dx \right)} \: d y 
            \]

        \item Se, per ogni $x \in [a,b]$ esiste l'integrale 

            \[
                H(x) = \int_{c}^{d} {f(x,y)} \: d x d y
            \]

            allora la funzione $x \rightarrow H(x)$ è integrabile in $[a,b]$ e vale la formula:
            
            \[
                \iint_R {f} = \int_{a}^{b} {H(x)} \: dx = \int_{a}^{b} {\left(\int_{c}^{d} {f(x,y)} \: d y \right)} \: dx 
            \]

    \end{enumerate}
}
\teorema{Formule di riduzione}{Sia $f: R=[a,b]\times [c,d] \rightarrow \mathbb{R}$ continua, allora $f \in R(\mathbb{R}) $ e si ha:

\[
    \iint_R {f(x,y)} \: d x d y = \int_{a}^{b} {\left(\int_{c}^{d} {f(x,y)} \: d y \right)} \: dx   = \int_{c}^{d} {\left(\int_{a}^{b} {f(x,y)} \: dx \right)} \: d y 
\]

}
\teorema{Formule di riduzione}{ Ogni funzione continua su un'insieme semplice $D \subset \mathbb{R}^{2}$ è integrabile su tale insieme e valgono le formule di riduzione:

    \begin{enumerate}
        \item Se $D$ è y-semplice allora:

            \[
                \iint_D {f(x,y)} \: d x d y = \int_{a}^{b} {d x \int_{g_1(x)}^{g_2(x)} {f(x,y)} \: d y } 
            \]
        \item Se $D$ è x-semplice allora:

            \[
                \iint_D {f(x,y)} \: d x d y = \int_{c}^{d} {d y \int_{h_1(y)}^{h_2(y)} {f(x,y)} \: d x } 
            \]
    \end{enumerate}

}
\teorema{}{Consideriamo l'applicazione verticale $F(\rho,\theta) = \left(F_1(\rho,\theta), F_2(\rho,\theta)\right)$ in cui $F: A \rightarrow \mathbb{R}^{2}$ in cui $A = (0,+\infty) \times (0,2\pi)$.

    Sia $S \subset (0,+\infty) \times (0,2\pi)$ un aperto misurabile nel piano $(\rho,\theta)$ con $\bar{S} \subset (0,\infty) \cdot (0,2\pi)$ e sia $T=F(S)$ (fig. \ref{fig:polari})


Allora per ogni funzione $f$ integrabile su $T$, continua e limitata, vale la sequente formula:

\[
    \iint_{T=F(S)} {f(x,y)} \: d x d y  = \iint_{S=F'(T)} {f( \rho \cos \theta, \rho \sin \theta) \underbrace{\rho}_\text{det $JF(\rho,\theta)$}} \: d \rho d \theta  
\]

}
\teorema{Teorema fondamentale dell'algebra}{
    L'equazione di II in $\mathbb{C}$ 

    \[
   a \lambda^{2}+b \lambda + c =0, in\ \mathbb{C}
    \]

    ha sempre due soluzioni in $\mathbb{C}$

}
\begin{proof}
    y è soluzione di \ref{IIomogenea} $\Leftrightarrow$ $Ly=0$ 

    Se considero $y(x) = e ^{\lambda x}$ 

    Devo dimostrare che:

    \[
        L(e ^{\lambda x}) = 0 \Leftrightarrow  p(\lambda) = 0
    \]

    Sostituisco a $x$ $e ^{\lambda x}$:

    \[
        L(e ^{\lambda x}) = a( e^{\lambda x})'' + b( e ^{\lambda x})' + c(e ^{\lambda x}) =
    \]

    \[
        =a \lambda ^{2} e ^{\lambda x} + b \lambda e ^{\lambda x} + c e^{\lambda x}= e ^{\lambda x}(a \lambda ^{2}+ b \lambda+ c)
    \]

    dunque

    \[
        L( e ^{\lambda x}) = 0 \Leftrightarrow a \lambda ^{2}+ b \lambda +c = 0 
    \]
           
\end{proof}
\newpage
\teorema{}{L'integrale generale dell'equazione omogenea $a y''+by'+c=0$ è dato da:

    \[
        c_1 y_1(x) + c_2 y_2(x)
    \]

    al variare di $c_1,c_2 \in \mathbb{R}$ dove $y_1(x)$ e $y_2(x)$ sono definite come sopra
}
\begin{proof}
       1) $b^{2}-4ac >0$ con $\lambda_1,\lambda_2$ soluzioni dell'equazioni di $p(\lambda)=0$    

       scrivo la Wronskiana di $y_1,y_2$:
       \[
        \begin{bmatrix}
            
        e ^{\lambda_1 x} & e ^{\lambda_2 x} \\
        \lambda_1e ^{\lambda_1 x} & \lambda_2e ^{\lambda_2 x} \\
        
        \end{bmatrix}
       \]
        che è diverso da zero quindi le soluzioni sono linearmente indipendenti

        sia ora $y(x)$ una soluzione di \ref{IIomogenea}:

        \[
            y(x) = e ^{\lambda_1 x}u(x)
        \]

        io devo determinare $u(x)$ per poi dimostrare che $y(x) = c_1e ^{\lambda_1 x}+c_2 e^{\lambda_2 x}$

        Poiché $y(x) = e ^{\lambda_1 x}u(x)$ è soluzione di \ref{IIomogenea} si ha derivando e sostituendo:

        \[
            a( e ^{\lambda_1 x} u(x))'' + b(e ^{\lambda_1 x}u(x))'+ c e ^{\lambda_1 x}u(x) =0
        \]

        \[
            a(\lambda_1 e ^{\lambda_1 x} u(x)+ e ^{\lambda_1 x}u'(x))' + b(\lambda_1e ^{\lambda_1 x}u(x) + e ^{\lambda_1 x}u'(x))+ c e ^{\lambda_1 x}u(x) =0
        \]

        \[
            e ^{(\lambda_1 x}[a \lambda_1 ^{2} + b \lambda_1+c)u(x)+\underbrace{(au''(x)+(2a \lambda_1 + b)u'(x))}_\text{impongo che sia zero}]=0
        \]

        estraggo solo l'ultima parentesi e impongo che sia uguale a zero perché il resto è già zero

        \[
            au''(x) + (2a \lambda_1 + b) u'(x) = 0
        \]

        divido per a:

        \[
            u''(x) +(2 \lambda_1 + \frac{b}{a}) u'(x) = 0
        \]

        sapendo che:

        \[
            a \lambda^{2} + b \lambda + c =0
        \]

        \[
             \lambda^{2} + \frac{b}{a} \lambda + \frac{c}{a} =0
        \]

        \[
            \lambda_1 + \lambda_2 = -\frac{b}{a}
        \]

        \[
            \lambda_1  \lambda_2 = \frac{c}{a}
        \]

        \[
            u''(x) + (2 \lambda_1 - \lambda_1 - \lambda_2)u'(x) = 0
        \]

        il meno per comodità:

        \[
            u''(x) - (\lambda_1 - \lambda_2)u'(x) = 0
        \]

        se adesso chiamo $u'(x)=v(x)$ e $v''(x) = u'(x)$ l'equazione diventa:

        \[
            v' -kv = 0
        \]

        Risolvendo 

        \[
            v(x) = ce ^{kx}
        \]

        \[
            v(x) = c e^{(\lambda_2 - \lambda_1)x}
        \]

        Risostituendo:

        \[
            u'(x) = c e ^{(\lambda_2- \lambda_1)x}
        \]

        Integrando:

        \[
            u(x)  = c_1 e ^{(\lambda_2 - \lambda_1)x}+c_2
        \]
        
        la nostra $y(x)$ diventa:

        \[
            y(x) = e ^{\lambda_1 x}u(x) = e ^{\lambda_1 x}( c_1 e ^{(\lambda_2 - \lambda_1)x}+c_2) = c_1 e ^{\lambda_2 x}+ c_2 e ^{\lambda_1 x}
        \]


\end{proof}
\newpage
\begin{proof}
       \[
           |x+ y| ^{2} = \langle x+y , x+y \rangle = (x+y) \bullet (x+y) \overset{\text{bilinearità}}{=} \langle x,x+y \rangle + \langle y,x+y \rangle \overset{\text{sempre bilinearità}}{=} 
       \]    

       \[
           =\langle x,x \rangle + \langle x,y \rangle + \langle y,x \rangle + \langle y,y \rangle = \langle x,x \rangle + 2\langle x,y \rangle + \langle y,y \rangle= 
       \]

       \[
            = |x|^{2} + |y|^{2} + 2x \bullet y
       \]
\end{proof}
\newpage
\begin{proof}
     Se
     \[
        y=0
     \]

     \[
         y=0=(0, \ldots ,0) 
     \]
     
     questo caso va bene.

     Sia dunque $\mathbb{R}^{n} \rightarrow y \neq 0$ e consideriamo la funzione reale di una variabile reale $t \rightarrow |x+ty|^{2}\ge 0$ polinomio di secondo grado in t

     \[
         |x + ty| ^{2} \overset{\text{Carnot}}{=} |x|^{2} + |ty|^{2} + 2\langle x,ty \rangle  = |x|^{2} + |y|^{2}t^{2} + 2 \langle x,y \rangle t
     \]

     è un polinomio di II grado in t dove $|y|^{2}> 0 $ essendo $y \neq 0$

     Il nostro $\frac{\Delta}{4}$ deve essere non positivo:

     \[
         (x\bullet y ) ^{2} - |x| ^{2}|y|^{2} \le 0
     \]

     \[
         (x\bullet y ) ^{2}\le  |x| ^{2}|y|^{2} 
     \]

     da cui si ha la tesi.

     Si verifica, se si ha che

     \[
         \langle x,y \rangle = |x| |y|
     \]

     si ha che il $\Delta$ del trinomio di II grado è nullo e dunque $t \in \mathbb{R}$ per cui $|x+ty|^{2}=0$ ovvero $x +ty=0$ $\rightarrow x=-ty$

     devo mostrare che $-t \ge 0$


     \[
         t = - \frac{\langle x,y \rangle}{|y|^{2}}
     \]

     si ricorda che $|y|>0$ essendo y non nullo

     \[
         -t = \frac{|x| |y|}{|y|^{2}}\ge 0
     \]

 \end{proof}
\newpage
\begin{proof}
       Dimostriamo la disuguaglianza triangolare, considero:

       \[
           |x+y|^{2} = \langle x+y , x+y \rangle = |x|^{2}+|y|^{2} + 2 \langle x,y \rangle \le 
       \]

       \[
           \le |x|^{2} + |y| ^{2} + 2|\langle x,y \rangle| \le \underbrace{|x| ^{2} + |y|^{2} + 2|x|\bullet |y|}_{(|x|+|y|)^{2}}
       \]

       estraendo e passando alle radici si ha

       \[
           |x+y| \le |x|+|y|
       \]
\end{proof}
\end{document}
