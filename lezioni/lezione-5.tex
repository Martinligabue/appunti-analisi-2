\documentclass[../appunti-analisi.tex]{subfiles}

\begin{document}

\section{Lezione 5}

Ritorniamo all'equazione del secondo ordine.

\[
    a_2(x)y''(x) + a_1(x) y'(x) + a_0(x) y(x) = f(x)
\]

con $a_0(),a_1(),a_2(),f()$ continue in I $ \in  [a,b]$

ci concentriamo nel caso in cui le $a$ sono costanti (coefficienti costanti).

L'altra volta abbiamo dimostrato che se abbiamo due soluzioni $y_1$ e $y_2$ esse sono linearmente indipendenti cioè il determinante della matrice di $y_1(x)y_2'(x)-y_2(x)y_2'(x)$ è diverso da 0 (determinante Wronskiano)

Se quindi l'equazione ha coefficienti costanti diventa:

\[
    ay''(x)+by'(x)+cy(x) = f(x)
\]

con $a,b,c \in \mathbb{R}$ con $a \neq 0$ se no non sarebbe di ordine II, $f(x)$ è continua in I

Adesso associamo il problema omogeneo:

\begin{equation}\label{IIomogenea}
    ay''(x) + by'(x) + cy(x) = 0
\end{equation}

\textbf{Numeri complessi} 

Qui dobbiamo introdurre i numeri complessi perché ci servono per la soluzione, di solito questi sono formati da una parte reale e una parte immaginaria:

\[
    z = \alpha + i\beta
\]

$z$ può essere scritto come coppia $(\alpha,\beta)$ a $i$ assegno $i=\sqrt{-1}$

Tornando a noi vediamo il caso in cui $b=c=0$

\[
    ay''(x) = 0
\]

in I e in particolare:

\[
    y''(x) = 0, \forall x \in I
\]

\[
    y'(x) = c, c \in \mathbb{R}
\]

\[
    y(x) = c_1x+c_0,c_1,c_0 \in \mathbb{R}
\]

Questo caso è facile. Se invece $b$ e $c$ non sono contemporaneamente nulli, devo considerare la seguente equazione algebrica di secondo grado:

\[
    p(\lambda) = a \lambda^{2}+b \lambda + c =0
\]

La sua equazione associata a \ref{IIomogenea}:

\[
    p(\lambda) =0 \Leftrightarrow  a \lambda^{2}+b \lambda + c =0, in\ \mathbb{C}
\]

\teorema{Teorema fondamentale dell'algebra}{
    L'equazione di II in $\mathbb{C}$ 

    \[
   a \lambda^{2}+b \lambda + c =0, in\ \mathbb{C}
    \]

    ha sempre due soluzioni in $\mathbb{C}$

}

\proposizione{}{
$y(x) = e ^{\lambda x}$ è soluzione di \ref{IIomogenea} $\Leftrightarrow $ $\lambda$ è soluzione (radice) di $p(\lambda)=0$ dell'equazione caratteristica associata a \ref{IIomogenea}

Indico con $Ly$ l'equazione $Ly= ay''+by'+cy$
}



\begin{proof}
    y è soluzione di \ref{IIomogenea} $\Leftrightarrow$ $Ly=0$ 

    Se considero $y(x) = e ^{\lambda x}$ 

    Devo dimostrare che:

    \[
        L(e ^{\lambda x}) = 0 \Leftrightarrow  p(\lambda) = 0
    \]

    Sostituisco a $x$ $e ^{\lambda x}$:

    \[
        L(e ^{\lambda x}) = a( e^{\lambda x})'' + b( e ^{\lambda x})' + c(e ^{\lambda x}) =
    \]

    \[
        =a \lambda ^{2} e ^{\lambda x} + b \lambda e ^{\lambda x} + c e^{\lambda x}= e ^{\lambda x}(a \lambda ^{2}+ b \lambda+ c)
    \]

    dunque

    \[
        L( e ^{\lambda x}) = 0 \Leftrightarrow a \lambda ^{2}+ b \lambda +c = 0 
    \]
           
\end{proof}


Adesso che ho dimostrato il mio problema è trovare le radici $p(\lambda) =0$ ($a \lambda ^{2} + b \lambda + c$):

Di solito le soluzioni di secondo grado si scrivono

\[
    \lambda_{1,2} = \frac{-b \pm \sqrt{b ^{2}-4 ac}}{2a}
\]
   
Le soluzioni $\lambda_1$ e $\lambda_2$ sono soluzioni di (\ref{IIomogenea} $e ^{\lambda_1x}$ e $e ^{\lambda_2x}$)

Distinguiamo tre casi per le soluzioni:

\begin{enumerate}
    \item soluzioni reali e distinte ($\Delta >0$)
    \item soluzioni reali e coincidenti ($\Delta = 0$)
    \item soluzioni complesse coniugate ($ \Delta <0$)
\end{enumerate}

1) $y_1(x) = e ^{\lambda_1x}$ e $y_2(x) = e ^{\lambda_2x}$ con $\lambda_1 e \lambda_2$ $\in \mathbb{R}$ con $\lambda_1 \neq \lambda_2$


2) $y_1(x) = e ^{\lambda x}$ e $y_2(x) = xe ^{\lambda x}$ con $\lambda = - \frac{b}{2a}=\lambda_1=\lambda_2$ $\in \mathbb{R}$ 

3) $y_1(x) = e ^{\alpha x} cos \beta x$ e $y_2(x) = e ^{\alpha x} sin \beta x$ 

questo caso corrisponde a soluzioni complesse coniugate  

\[
    \lambda_1 = \alpha- i \beta \in \mathbb{C} 
\]

\[
    \lambda_2 = \alpha+ i \beta \in \mathbb{C} 
\]

\[
    \lambda = \frac{-b \pm \sqrt{-(4ac-b^{2})}}{2a} = \frac{-b \pm \sqrt{-1(4ac - b^{2})}}{2a} \overset{\text{perche i} = \sqrt{-1}}{=} \frac{-b \pm  \sqrt{4ac -b^{2}}i}{2a} = \alpha \pm i \beta
\]

dove $\alpha = -\frac{b}{2a}$ e $\beta = \frac{\sqrt{4ac - b^{2}}}{2a} >0$


\teorema{}{L'integrale generale dell'equazione omogenea $a y''+by'+c=0$ è dato da:

    \[
        c_1 y_1(x) + c_2 y_2(x)
    \]

    al variare di $c_1,c_2 \in \mathbb{R}$ dove $y_1(x)$ e $y_2(x)$ sono definite come sopra
}

\begin{proof}
       1) $b^{2}-4ac >0$ con $\lambda_1,\lambda_2$ soluzioni dell'equazioni di $p(\lambda)=0$    

       scrivo la Wronskiana di $y_1,y_2$:
       \[
        \begin{bmatrix}
            
        e ^{\lambda_1 x} & e ^{\lambda_2 x} \\
        \lambda_1e ^{\lambda_1 x} & \lambda_2e ^{\lambda_2 x} \\
        
        \end{bmatrix}
       \]
        che è diverso da zero quindi le soluzioni sono linearmente indipendenti

        sia ora $y(x)$ una soluzione di \ref{IIomogenea}:

        \[
            y(x) = e ^{\lambda_1 x}u(x)
        \]

        io devo determinare $u(x)$ per poi dimostrare che $y(x) = c_1e ^{\lambda_1 x}+c_2 e^{\lambda_2 x}$

        Poiché $y(x) = e ^{\lambda_1 x}u(x)$ è soluzione di \ref{IIomogenea} si ha derivando e sostituendo:

        \[
            a( e ^{\lambda_1 x} u(x))'' + b(e ^{\lambda_1 x}u(x))'+ c e ^{\lambda_1 x}u(x) =0
        \]

        \[
            a(\lambda_1 e ^{\lambda_1 x} u(x)+ e ^{\lambda_1 x}u'(x))' + b(\lambda_1e ^{\lambda_1 x}u(x) + e ^{\lambda_1 x}u'(x))+ c e ^{\lambda_1 x}u(x) =0
        \]

        \[
            e ^{(\lambda_1 x}[a \lambda_1 ^{2} + b \lambda_1+c)u(x)+\underbrace{(au''(x)+(2a \lambda_1 + b)u'(x))}_\text{impongo che sia zero}]=0
        \]

        estraggo solo l'ultima parentesi e impongo che sia uguale a zero perché il resto è già zero

        \[
            au''(x) + (2a \lambda_1 + b) u'(x) = 0
        \]

        divido per a:

        \[
            u''(x) +(2 \lambda_1 + \frac{b}{a}) u'(x) = 0
        \]

        sapendo che:

        \[
            a \lambda^{2} + b \lambda + c =0
        \]

        \[
             \lambda^{2} + \frac{b}{a} \lambda + \frac{c}{a} =0
        \]

        \[
            \lambda_1 + \lambda_2 = -\frac{b}{a}
        \]

        \[
            \lambda_1  \lambda_2 = \frac{c}{a}
        \]

        \[
            u''(x) + (2 \lambda_1 - \lambda_1 - \lambda_2)u'(x) = 0
        \]

        il meno per comodità:

        \[
            u''(x) - (\lambda_1 - \lambda_2)u'(x) = 0
        \]

        se adesso chiamo $u'(x)=v(x)$ e $v''(x) = u'(x)$ l'equazione diventa:

        \[
            v' -kv = 0
        \]

        Risolvendo 

        \[
            v(x) = ce ^{kx}
        \]

        \[
            v(x) = c e^{(\lambda_2 - \lambda_1)x}
        \]

        Risostituendo:

        \[
            u'(x) = c e ^{(\lambda_2- \lambda_1)x}
        \]

        Integrando:

        \[
            u(x)  = c_1 e ^{(\lambda_2 - \lambda_1)x}+c_2
        \]
        
        la nostra $y(x)$ diventa:

        \[
            y(x) = e ^{\lambda_1 x}u(x) = e ^{\lambda_1 x}( c_1 e ^{(\lambda_2 - \lambda_1)x}+c_2) = c_1 e ^{\lambda_2 x}+ c_2 e ^{\lambda_1 x}
        \]


\end{proof}

Adesso voglio per il caso 2)

\[
    \lambda_1 = \lambda_2 = \lambda = -\frac{b}{2a} \in \mathbb{R}
\]
   
\[
    p(\lambda) =0 \Leftrightarrow e ^{\lambda x} \text{È soluzione di \ref{IIomogenea}}
\]

sia quindi $y(x)$ una soluzione di \ref{IIomogenea} che scriviamo come:

\[
    y(x) = e ^{\lambda x}u(x) 
\]

Come prima si ottiene:

\[
    a(e ^{\lambda x}u(x) )''+ b(e ^{\lambda x}u(x))' + c e ^{\lambda x}u(x)=0
\]

\[
    \overbrace{e ^{\lambda x}}^{>0}(a u''(x) + \underbrace{(a \lambda^{2}+b \lambda +c )}_\text{=0} u(x) + (2a \lambda+b)u'(x))=0
\]

estraggo la parte che impongo a zero:

\[
    au''(x) + (2a \lambda+b)u'(x) = 0
\]

divido per a:

\[
 u''(x) + (2 \lambda + \frac{b}{a})u'(x) = 0
\]

sapendo che $-\frac{b}{a} = 2 \lambda$:

\[
 u''(x) + (\cancel{2 \lambda} + \cancel{\frac{b}{a}})u'(x) = 0
\]


\[
    u'(x) = c_1
\]

\[
    u(x) = c_1 x +c_2
\]

e quindi ho la soluzione:

\[
    y(x) = e ^{\lambda x}(c_1x+c_2) = c_1x e^{\lambda x}+ c_2 e ^{\lambda x}
\]

\end{document}
