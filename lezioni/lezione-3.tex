\documentclass[../appunti-analisi.tex]{subfiles}

\begin{document}

\section{Lezione 3}

Solitamente si suppongono delle condizioni iniziali nel risolvere le equazioni differenziali (problema di Cauchy).

\begin{equation}
    \begin{cases}
      y'(x)+a(x)y(x)=f(x)\\
      y(x_0)=y_0
    \end{cases}\,.
\end{equation}

Praticamente gli integrali della formula generale diventano definiti tra $x_0$ e $x$.

Quindi:

\[
    y(x) = ce ^{-A(x)}+ e ^{-A(x)} \int_{{}}^{{}} {e ^{A(x)}f(x)} \: d{x} {} = c e ^{-\int_{{x_0}}^{{x}} {a(t)} \: d{t} {}}+e ^{- \int_{{x_0}}^{{x}} {a(t)} \: d{t} {}}\int_{{x_0}}^{{x}} {e ^{\int_{{x_0}}^{{s}} {a(t)} \: d{t} {}}f(s)} \: d{s} {}\\
\]

\[
    y(x_0)=y_0=c
\]


Voglio trovare la soluzione generale in questo caso, parto dall'omogenea:

\[
    y'+x(x)y(x) = 0
\]

\[
    e ^{\int_{{x_0}}^{{x}} {a(x)} \: d{t} {}} = e ^{A(x)}
\]


\subsection{Il problema di Cauchy}

Quindi introduciamo il problema di Cauchy:

\begin{equation}
    \begin{cases}
      y'+a(x)y = f(x)\\
      y(x_0) = y_0
    \end{cases}\,.
\end{equation}

dove $x \in I = [a,b]$ e $x_0 \in I$

con le ipotesi fatte ($a(x)$ e $f(x)$ continue in I) ha una e una sola soluzione (SOLUZIONE UNICA)

con l'espressione esplicita determinata.

\textbf{Esempio 1}

Determinare la soluzione del problema di Cauchy:

\begin{equation}
    \begin{cases}
      y'(x)=5y(x) + e ^{x}\\
      y(0)=0
    \end{cases}\,.
\end{equation}

\[
    A(x)=\int_{{0}}^{{x}} {a(t)} \: d{t} {}= - \int_{{0}}^{{x}} {5} \: d{t} {}= -5x
\]

\[
    y(x)=0e ^{5x} + e ^{5x}\int_{{0}}^{{x}} {e ^{-5t}e ^{t}} \: d{t} {}=
\]

\[
    = e ^{5x} \Eval{[ -\frac{1}{4} e ^{-4t}]}{0}{x} = e ^{5x}(-\frac{1}{4} e ^{-4x}+\frac{1}{4})= - \frac{1}{4} e ^{x}+ \frac{1}{4} e ^{5x}
\]

\textbf{Esempio 2}

Determinare l'integrale generale della EDO:

\[
    y'+\frac{1}{\sqrt{x}} y=1
\]

e trovare le eventuali soluzioni tali che:

\[
    \lim_{x \to \infty} y(x) = +\infty
\]

Soluzione: 

l'equazione è definita per ogni $x>0$

\[
    a(x) = \frac{1}{\sqrt{x}} 
\]

\[
    A(x) = \int_{{}}^{{}} {\frac{1}{\sqrt{x}} } \: d{x} {}
\]


L'integrale generale:

\[
    y(x) = c e ^{-\int_{{}}^{{}} {\frac{1}{\sqrt{x}} } \: d{x} {}}+ e ^{-\int_{{}}^{{}} {\frac{1}{\sqrt{x}} } \: d{x} {}}( \int_{{}}^{{}} {e ^{\int_{{\frac{1}{\sqrt{x}} }}^{{}} {} \: d{x} {+1}}} \: d{x} {})=
\]

\[
    =e ^{2 \sqrt{x}} (e+ \int_{{}}^{{}} {e ^{2\sqrt{x}}} \: d{x} {})
\]

Risolvo l'integrale pongo $t = 2 \sqrt{x}$ quindi $ dt = \frac{1}{\sqrt{x}}dx \rightarrow dx = \frac{t}{2} dt $:

\[
    \int_{{}}^{{}} {e ^{2 \sqrt{x}}} \: d{x} {}= \int_{{}}^{{}} {e ^{t}\frac{t}{2} } \: d{t} {} = e ^{x}\frac{t}{2} - \int_{{}}^{{}} {e ^{t}\frac{1}{2} } \: d{t} {}=
\]

\[
    = e ^{t} \frac{t}{2} - \frac{1}{2 e ^{t}} 
    \overset{\text{risostituisco}}{=} e ^{2 \sqrt{x}} \frac{2 \sqrt{x}}{2} - \frac{1}{2} e ^{2 \sqrt{x}}
\]

Ora riscrivo l'integrale generale:

\[
    y(x) = e ^{-2 \sqrt{x}}[ c + e ^{2 \sqrt{x}}(\sqrt{x}- \frac{1}{2} )]= c e ^{-2 \sqrt{x}}+ \sqrt{x} - \frac{1}{2} 
\]

Adesso soddisfo la richiesta (quali sono le soluzioni che vanno all'infinito)

\[
    \lim_{x \to \infty} c ^{-2 \sqrt{x}} + \sqrt{x} - \frac{1}{2} = +\infty
\]

questo vale per $\forall c \in \mathbb{R}$


\textbf{Esempio 3}

\begin{equation}
    \begin{cases}
      y' + \frac{2y}{x} = \frac{1}{2} \\
      y(-1)=2
    \end{cases}\,.
\end{equation}

Considero l'intervallo dove sta il $x_0=-1$ quindi $(-\infty,0)$

\[
    A(x) = \int_{{-1}}^{{x}} {\frac{1}{t} } \: d{t} {} = \Eval{[2 log|t|]}{-1}{x} = 2 log|x| - 2 log|-1| = 2 log|x| = 
\]

per via dell'intervallo il valore assoluto viene preso col meno:

\[
    =2 log(-x)  
\]

quindi l'integrale generale:

\[
    y(x) = 2 e ^{-2log(-x)}+ e^{-2log(-x)}(\int_{{-1}}^{{x}} {e ^{2log(-t)}\frac{1}{t ^{2}} } \: d{t} {})=
\]

uso la proprietà dei logaritmi:

\[
    = 2 e ^{log \frac{1}{x ^{2}} }+ e ^{log \frac{1}{x ^{2}} }\int_{{-1}}^{{x}} {e ^{log t ^{2}}} \: d{t} {}= \frac{2}{x ^{2}} + \frac{1}{x ^{2}} \int_{{-1}}^{{x}} {1} \: d{t} {} = \frac{2}{x ^{2}} + \frac{1}{x ^{2}} \Eval{[t]}{-1}{x} 
     = \frac{2 }{x ^{2}} + \frac{1}{x ^{2}} (x+1)
\]

\end{document}
