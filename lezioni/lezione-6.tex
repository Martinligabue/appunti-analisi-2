\documentclass[../appunti-analisi.tex]{subfiles}

\begin{document}

\section{Lezione 6}

\subsection{Determinazione della soluzione particolare per EDO II ordine}


Dobbiamo vedere ora come si determina la soluzione particolare.

\[
    (1)\ ay''+by'+cy = f(x)\ \in I=[a,b]
\]

\[
    (2)\ ay''+by'+cy = 0\ \in I=[a,b]
\]

\[
    y(t) = c_1 y_1(x) + c_2 y_2(x) + \bar{y} (x)
\]

La $\bar{y}$ è la soluzione particolare, ci sono due modi:

\begin{itemize}
    \item Si procede a occhio, per similitudine guardando l'espressione di $f(x)$
    \item Si usa il metodo di variazione delle costanti

        \[
            \bar{y} = c_1(x) y_1(x) + c_2(x) y_2(x)
        \]

        dove ${y_1(x),y_2(x)}$ soluzioni linearmente indipendenti di (2) con $c_1(x),c_2(x)$ funzioni di classe $\mathbb{C}^{2}(I)$ da determinare.
\end{itemize}

Vediamo come fare con quest'ultimo metodo.

Poiché $\bar{y} (x)$ è soluzione di (1) allora  $a\bar{y} ''+b \bar{y}'+c\bar{y}=f$

\[
    \bar{y} (x) = c_1(x) y_1(x) + c_2(x) y_2(x)
\]

\[
\bar{y} '(x) = c_1'(x) y_1(x) + c_1 y_1'(x) + c_2'(x) y_2(x) + c_2(x) y_2'(x)
\]

Adesso impongo che $c_1'(x) y_1(x) + c_2'(x) y_2(x) = 0$:

\[
    \bar{y} '(x) = c_1(x) y_1'(x) +c_2(x) y_2'(x)
\]

\[
    \bar{y} ''(x) = c_1'(x) y_1'(x) + c_1(x) y_1''(x) + c_2'(x) y_2'(x) + c_2(x) y_2''(x)
\]

sapendo che $a \bar{y} ''+ b \bar{y} ' + c \bar{y}  = f$ sostituisco quello che ho trovato sopra a questa espressione:

\[
    \textbf{a}[c_1'(x) y_1'(x) + c_1(x) y_1''(x) + c_2'(x) y_2'(x) +c_2(x) y_2''(x)]+
\]
\[
     + \textbf{b}[c_1(x) y_1'(x) + c_2(x) y_2'(x)]+ \textbf{c}[c_1(x) y_1(x) + c_2(x) y_2(x)] = f(x)
\]

Adesso raccolgo a fattore comune le $c_i$:

\[
    c_1(x) [ \underbrace{a y_1''(x) +b y_1'(x) + c y_1(x)}_\text{=0}] + c_2(x) [\underbrace{a y_2''(x) + b y_2'(x) + c y_2(x)}_\text{=0}]+ a [c_1'(x) y_1'(x)+ c_2'(x) y_2'(x)] = f(x)
\]

quindi mi rimane:

\[
    c_1'(x) y_1'(x) + c_2'(x) y_2'(x) = \frac{f(x)}{a}
\]

Ottengo il sistema di 2 equazioni nelle due incognite ($c_1'(x),c_2'(x)$) non omogeneo:

    \begin{equation}
        \begin{cases}
            c_1'(x)y_1(x) + c_2'(x) y_2(x) = 0\\
            c_1'(x) y_1'(x) + c_2'(x) y_2'(x)  = \frac{f(x)}{a}
        \end{cases}\,.
    \end{equation}

La matrice dei coefficienti del sistema è:

\[
\begin{bmatrix}
y_1(x) & y_2(x) \\
y_1'(x) & y_2'(x) \\
\end{bmatrix}
\neq 0
\]

È la matrice Wronskiana.

Uso il metodo di Cramer per risolvere il sistema:

\[
    A = \begin{pmatrix}
        a_{11} & a_{12}  \\
        a_{21} & a_{22}  \\
\end{pmatrix}
\]

$det A = a_{11} a_{22} - a_{12} a_{21}$

Il metodo:

\[
c_1'(x) = 
    \frac{
\begin{vmatrix}
0 & y_2(x)  \\
\frac{f(x)}{a} & y_2'(x)  \\
\end{vmatrix}
    }{
\begin{vmatrix}
y_1(x) & y_2(x)  \\
y_1'(x) & y_2'(x)  \\
\end{vmatrix}
    }
\]

\[
    = \frac{- y_2(x) \frac{f(x)}{a}}{y_1(x) y_2'(x) - y_2(x) y_1'(x)}
\]

\[
c_2'(x) = 
    \frac{
\begin{vmatrix}
y_1(x) & 0  \\
y_1'(x) & \frac{f(x)}{a}  \\
\end{vmatrix}
    }{
\begin{vmatrix}
y_1(x) & y_2(x)  \\
y_1'(x) & y_2'(x)  \\
\end{vmatrix}
    }
\]

\[
    = \frac{- y_1(x) \frac{f(x)}{a}}{y_1(x) y_2'(x) - y_2(x) y_1'(x)}
\]

Ora dobbiamo integrare

\[
    c_1(x) = \int_{}^{} {c_1'(x)} \: dx 
\]

e

\[
    c_2(x) = \int_{}^{} {c_2'(x)} \: dx 
\]

Si ha che l'insieme delle soluzioni di (1):

\[
    y(t)  = \underbrace{c_1 y_1(x) + c_2 y_2(x)}_\text{generale} + \underbrace{c_1(x) y_1(x) + c_2 y_2(x)}_\text{particolare}
\]

Assegnando le condizioni iniziali: 

\[
    y(x_0) = y_0
\]

\[
    y'(x_0) = y_0'
\]

Quindi il problema di Cauchy mi viene:

    \begin{equation}
        \begin{cases}
            ay''+ by'+cy= f(x)\\
            y(x_0)=y_0\\
            y'(x_0) = y_0'
        \end{cases}\,.
    \end{equation}

Trovare le soluzioni dei seguenti problemi di Cauchy

\textbf{Esempio 1} 

    \begin{equation}
        \begin{cases}
            3y'' + 5y' + 2y=3e ^{2x}\\
            y(0) = 0\\
            y'(0) = 1
        \end{cases}\,.
    \end{equation}

Scriviamo (1) e (2):

1)
\[
    3y''+5y'+2y = 3 e^{2x}
\]

2)

\[
    3y''+5y'+2y = 0
\]

Risolvo:

\[
    3 \lambda ^{2} + 5 \lambda + 2 = 0
\]

$p = 6$ $s=5= 3+2$ 

\[
    3 \lambda^{2} + 3 \lambda + 2 \lambda + 2 =0
\]

\[
    3 \lambda ( \lambda+1) + 2( \lambda + 1) =0
\]

\[
    (3 \lambda +2 ) ( \lambda +1 ) =0
\]

\[
    \lambda= -\frac{2}{3}, \lambda=-1
\]

quindi ho soluzioni linearmente indipendenti di (2):

\[
    y_1(x) = e ^{-\frac{2}{3}x}, y_2(x) = e ^{-x}
\]

Integrale generale di (1):

\[
    y_0(x) = c_1 e^{-\frac{2}{3}x} + c_2 e ^{-x},c_1,c_2 \in \mathbb{R}
\]


Una soluzione particolare di (1) è dunque:

\[
    \bar{y} (x) = A e ^{2x}
\]

Ora derivo due volte:

\[
    \bar{y} '(x) = 2A e ^{2x}
\]

\[
    \bar{y} ''(x) = 4A e ^{2x}
\]

sostituendo poi in (1):

\[
    12Ae ^{2x} + 10 A e ^{2x} + 2 A e ^{2x} = 3 e^{2x}
\]

\[
    24A e ^{2x} = 3 e ^{2x}
\]

\[
    A = \frac{1}{8}
\]

Adesso devo imporre le condizioni iniziali a queste due:

\[
    y(x) = c_1 e ^{-\frac{2}{3}x} + c_2 e ^{-x}+ \frac{1}{8}e ^{2x}
\]

\[
    y'(x) = -\frac{2}{3}c_1 e ^{-\frac{2}{3}x} - c_2 e ^{-x}+ \frac{1}{4} e ^{2x}
\]


    \begin{equation}
        \begin{cases}
            c_1+c_2+\frac{1}{8}=0\\
            -\frac{2}{3}c_1-c_2+\frac{1}{4}=1
        \end{cases}\,.
    \end{equation}

    \begin{equation}
        \begin{cases}
            c_1=c_2-\frac{1}{8}\\
            \frac{2}{3}c_2+\frac{1}{12}-c_2+\frac{1}{4}=1
        \end{cases}\,.
    \end{equation}


    \begin{equation}
        \begin{cases}
            c_1=c_2-\frac{1}{8}\\
            -\frac{1}{3}c_2= 1- \frac{1}{3}
        \end{cases}\,.
    \end{equation}

    \begin{equation}
        \begin{cases}
            c_1=c_2-\frac{1}{8}\\
            -\frac{1}{3}c_2=\frac{2}{3}
        \end{cases}\,.
    \end{equation}

    \begin{equation}
        \begin{cases}
            c_1=\frac{15}{8}\\
            c_2=-2
        \end{cases}\,.
    \end{equation}


Infine quindi la soluzione del problema:

\[
    y(x) = \frac{15}{8}e ^{-\frac{2}{3}x}- 2 e ^{-x}+ \frac{1}{8}e ^{2x}
\]


\textbf{Esempio 2} 

    \begin{equation}
        \begin{cases}
    y''(x) -2y'(x)+ y(x) = 5sinx \\
            y(0) = 0\\
            y'(0) = 1
        \end{cases}\,.
    \end{equation}


Risolvo l'equazione associata:

\[
    \lambda^{2}-2 \lambda + 1=0
\]

\[
    (\lambda-1)^{2}=0
\]

quindi $\lambda_1=\lambda_2=1$


Devo trovare:

\[
    y_0(x) = c_1 e ^{x} + c_2 x e ^{x},c_1,c_2 \in \mathbb{R}
\]

la soluzione particolare è per essere sicuri di prendere il termine noto che è un seno:

\[
    \bar{y} (x) = A cosx + B sinx
\]

le derivate:

\[
    \bar{y} '(x)  = -A sinx + B cosx
\]

\[
    \bar{y} ''(x)  = -A cosx - B sinx
\]
    

Sostituiamo a quella iniziale:

\[
    \cancel{-A cosx} \cancel{- Bsinx} + 2A sinx - 2B cosx+ \cancel{A cosx} + \cancel{Bsinx} = 5sinx
\]

e quindi $2A = 5$ e $B=0$:

\[
    \bar{y} (x) = \frac{5}{2} cosx
\]

adesso devo trovare:

\[
    y(x) = c_1 e ^{x}+ c_2 x e ^{x} + \frac{5}{2} cosx
\]

\[
    y'(x)  = c_1 e ^{x} + c_2 e ^{x} + c_2x e ^{x} - \frac{5}{2}sinx
\]

Adesso impongo le condizioni iniziali:

    \begin{equation}
        \begin{cases}
            c_1 +\frac{5}{2}=0\\
            c_1+c_2= 1
        \end{cases}\,.
    \end{equation}

    \begin{equation}
        \begin{cases}
            c_1 = -\frac{5}{2}\\
            c_2 = \frac{7}{2}
        \end{cases}\,.
    \end{equation}

La soluzione del problema quindi:

\[
    y(x) = -\frac{5}{2}e ^{x}+ \frac{7}{2} x e ^{x} + \frac{5}{2} cosx
\]

\textbf{Esempio 3} 

    \begin{equation}
        \begin{cases}
            y''+ 2y'+ 2y= 3x^{2}\\
            y(0) = 0\\
            y'(0) = 1
        \end{cases}\,.
    \end{equation}


risolvo:

\[
    \lambda^{2}+2 \lambda+2 =0
\]

\[
    \lambda= \frac{-1 \pm \sqrt{1-2}}{1}= -1 \pm 1
\]

le soluzioni sono complesse:

\[
    \alpha \pm  i \beta
\]

con $\alpha = -1$ e $\beta = 1$ quindi sostituisco:

\[
    y_0(x) = y_0(c_1,c_2) = e ^{-x}(c_1cosx + c_2 sinx)
\]

la soluzione particolare:

\[
    \bar{y} (x) = A x^{2}+Bx+C
\]

derivo due volte:

\[
    \bar{y} '(x) = 2Ax + B
\]

\[
    \bar{y} ''(x) = 2A
\]

sostituisco in (1):

\[
    2A + 4Ax + 2B + 2Ax^{2}+2Bx + 2C = 3x^{2}
\]

\[
    2Ax^{2} + 2(2A+B) x + 2(A+B+C) = 3x^{2}
\]

risolvo un sistema per le incognite $A,B,C$ e quindi trovo che:

    \begin{equation}
        \begin{cases}
            2A = 3\\
            2A + B = 0\\
            A+B+C= 0
        \end{cases}\,.
    \end{equation}

    \begin{equation}
        \begin{cases}
            A=\frac{3}{2}\\
            B=-3\\
            \frac{3}{2}-3+C=0
        \end{cases}\,.
    \end{equation}

    \begin{equation}
        \begin{cases}
            A=\frac{3}{2}\\
            B=-3\\
            C=\frac{3}{2}
        \end{cases}\,.
    \end{equation}


Quindi la soluzione particolare:

\[
    \bar{y} (x) = \frac{3}{2}x^{2}-3x+\frac{3}{2}
\]


l'espressione quindi è:

\[
    y(x) = e ^{-x}(c_1 cosx + c_2 sinx ) + \frac{3}{2}x^{2}-3x+\frac{3}{2}
\]

derivo:

\[
    y'(x) = e ^{-x}(c_1 cosx+ c_2 sinx) + e ^{-x}(-c_1sinx+c_2 cosx) +3x -3
\]

quindi impongo le condizioni:

    \begin{equation}
        \begin{cases}
            c_1 +\frac{3}{2}=0\\
            -c_1+c_2 -3 = 1
        \end{cases}\,.
    \end{equation}

    \begin{equation}
        \begin{cases}
            c_1=-\frac{3}{2}\\
            c_2=-\frac{5}{2}
        \end{cases}\,.
    \end{equation}

Quindi la soluzione del problema:

\[
    y(x) = e ^{-x}(-\frac{3}{2}cosx + \frac{5}{2} sinx) + \frac{3}{2}x^{2}-3x + \frac{3}{2}
\]

\textbf{Esempio 4} 

    \begin{equation}
        \begin{cases}
            y''-6y' + 9y = e ^{3x}\\
            y(0) = 1\\
            y'(0) = 0
        \end{cases}\,.
    \end{equation}

\[
    \lambda^{2} -6 \lambda + 9 = 0
\]

\[
    \lambda_1=\lambda_2=3
\]

\[
    y_0(x)  = c_1 e ^{3x}+ c_2 x e ^{3x}
\]

qua abbiamo il problema che $\lambda=3$ che è la stessa del termine noto, devo quindi modificarla:

\[
    \bar{y} (x) = A x^{2}e ^{3x}
\]

dove il due della x viene dalla molteplicità, 2 in questo caso.

Proviamo a risolverlo con il metodo di variazione delle costanti:

\[
    \bar{y} (x) = c_1(x) e ^{3x}+ c_2(x) x e^{3x}
\]

lo risolvo col sistema (porco dio):

    \begin{equation}
        \begin{cases}
    c_1'(x)e ^{3x} + c_2'(x) x e ^{3x}=0 \\
    c_1'(x) 3 e ^{3x} + c_2'(x) (e ^{3x} + 3x e ^{3x}) = e ^{3x}
        \end{cases}\,.
    \end{equation}

% \[
%     c_1'(x) = \frac{
%                 \begin{vmatrix}
% 0 & x e ^{3x}  \\
%  e ^{3x} & (3x+1) e ^{3x}  \\
% \end{vmatrix}
%     }
%         \begin{vmatrix}
%     e ^{3x} & x e ^{3x}  \\
%     3 e ^{3x} & (3x+1) e ^{3x}  \\
%         \end{vmatrix}
% \]

facendolo viene:

    \begin{equation}
        \begin{cases}
            c_1(x) = \int_{}^{} {-x} \: dx =-\frac{x^{2}}{2}\\
            c_2(x) = \int_{}^{} {} \: dx  = x
        \end{cases}\,.
    \end{equation}

\[
    \bar{y} (x) = - \frac{x^{2}}{2}e ^{3x} + x x e ^{3x} = \frac{x^{2}}{2}e ^{3x}
\]


\textbf{Risoluzione esercizio 6 appunti prof} 

    \begin{equation}
        \begin{cases}
            y'= \frac{sinx}{cosy}\\
            y( \frac{\pi}{2} = \frac{\pi}{6}
        \end{cases}\,.
    \end{equation}

questa è a variabili separabili:

\[
    y'(x) cosy(x) = sinx
\]

\[
    \int_{}^{} {y'(x) cosy(x)} \: dx = \int_{}^{} {sinx} \: dx 
\]

\[
    \int_{}^{} {cosy} \: dy = \int_{}^{} {sinx} \: dx 
\]

\[
    sin y = -cos x +c
\]

impongo adesso le condizioni:

\[
    sin y( \frac{\pi}{2}) = - cos \frac{\pi}{2} + c
\]

\[
    sin \frac{\pi}{6} = c
\]

quindi:

\[
    \frac{1}{2}=c
\]

quindi sostituisco la c:

\[
    sin y(x) = - cos x + \frac{1}{2}
\]

Devo fare l'arcoseno e vedo dove è definita la cosa:

\[
    -1 \le sin y(x) \le 1
\]

ma anche:

\[
    -1 \le cosx + \frac{1}{2} \le 1
\]

impongo il sistema:

    \begin{equation}
        \begin{cases}
            -cosx +\frac{1}{2}\ge -1\\
            -cos x + \frac{1}{2} \le 1
        \end{cases}\,.
    \end{equation}

    \begin{equation}
        \begin{cases}
            cosx \le  \frac{3}{2}\\
            cosx \ge -\frac{1}{2}
        \end{cases}\,.
    \end{equation}

% \begin{tikzpicture}
% \begin{axis}[
%     xmin = -5, xmax = 5,
%     ymin = -5, ymax = 5]
%     \addplot[
%         domain = -5:5,
%         samples = 200,
%         smooth,
%         thick,
%         blue,
%     ] {sinx};
% \end{axis}
% \end{tikzpicture}

Prendo la soluzione nell'intervallo:

\[
    -\frac{2}{3}\pi \le  x \le  \frac{2}{3}\pi
\]

la soluzione del problema: 

\[
    y(x) = arcsin(-cosx +\frac{1}{2})
\]

\end{document}
