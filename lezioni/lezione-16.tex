\documentclass[../appunti-analisi.tex]{subfiles}

\begin{document}

\section{Lezione 16}

\subsection{Differenziabilità}

Avendo una funzione $f:A\subseteq \mathbb{R}^{n} \rightarrow \mathbb{R}$ con $A$ aperto

f è derivabile in $\bar{x} =(x_1, \ldots ,x_n) \in A$ è lo stesso che dire che esiste:

\[
    \nabla f(\bar{x} ) = ( \frac{\partial f}{\partial x_1}(\bar{x} ), \ldots , \frac{\partial f}{\partial x_n}(\bar{x} ) )
\]


\defn{}{ Si dice che $f$ è differenziabile in $\bar{x} \in A $ se $f$ è derivabile in $\bar{x} $ ed inoltre vale la seguente relazione:

    \[
        \lim_{ h \to 0 } \frac{f( \bar{x} + \bar{h} ) - f( \bar{x} ) - \langle  \nabla f(\bar{x} ), \bar{h} \rangle}{|\bar{h} |} =0
    \]


    dove $\bar{h} \in \mathbb{R}^{n}$ e in particolare $|\bar{h} | = \sqrt{\sum^{n}_{i=1} h_i^{2}}$, inoltre:

    \[
    \langle  \nabla f(\bar{x} ), \bar{h} \rangle = \frac{\partial f}{\partial x_1}(\bar{x} )h_1+ \frac{\partial f}{\partial x_2}(\bar{x} ) h_2+ \ldots  + \frac{\partial f}{\partial x_n}(\bar{x} ) h_n
    \]

    Se $f$ è differenziabile in ogni punto di $A$ si dice che $f$ è differenziabile in $A$
}

Da notare che il prodotto scalare $\langle \nabla f(\bar{x} ), \bar{h}  \rangle$ rappresenta l'incremento che io faccio verso ogni direzione per poter ``vedere'' che cosa succede quando mi sposto un ``pochino'' in ogni direzione.

\newpage

\defn{}{ Definisco $L : \mathbb{R}^{n} \rightarrow \mathbb{R}$ (funzione lineare) è l'applicazione che ad $\bar{h} $ associa il prodotto scalare:

    \[
        \langle  \nabla f(\bar{x} ) , \bar{h} \rangle
    \]

    si chiama differenziale di $f$ in $\bar{x} $ e si indica con $d f(x)$ è un'applicazione lineare da $\mathbb{R}^{n}$ in $\mathbb{R}$ della variabile $\bar{h} $

    \[
        d f(\bar{x} ) (\bar{h} ) = \langle \nabla f(\bar{x} ) , \bar{h}  \rangle
    \]
}

$f$ è differenziabile in $\bar{x} \in A$ se esiste un funzionale \textbf{lineare} $L: \mathbb{R}^{n} \rightarrow \mathbb{R}$ t.c.:

\[
    \lim_{ h \to 0 } \frac{f(x+h) -f(x) - L(h)}{|h|} = 0
\]

In $\mathbb{R}^{n}$ ogni funzionale lineare si rappresenta come un opportuno prodotto scalare.


Se $L: \mathbb{R}^{n}\rightarrow \mathbb{R}$ è lineare allora esiste $\bar{l} \in \mathbb{R}^{n}$ t.c.:

\[
    L(\bar{h} ) = \langle \bar{h} , \bar{l}  \rangle
\]

$\forall \bar{h} \in \mathbb{R}^{n}$

Se dunque $f: A \subset \mathbb{R}^{n}\rightarrow \mathbb{R}$ è differenziabile in $\bar{x} \in A$ se è derivabile in $\bar{x} $ ($\exists \nabla f(\bar{x} ) \in \mathbb{R}^{n}$) e se:

\[
    f(\bar{x} + \bar{h} ) = f(\bar{x} ) + \langle \nabla f(\bar{x} ), \bar{h}  \rangle + o(|h|)
\]

per $h \rightarrow 0$.


\textbf{Geometricamente} 

La differenziabilità è legata all'esistenza del piano tangente al grafico della $f$ nel punto.

Supponiamo che $f$ sia differenziabile in $\bar{x_0} \in A $ allora tornando alla relazione (che ci definisce la differenziabilità):

\[
    f(x_0+h) = f(x_0) +\langle \nabla f(x_0) , \bar{h} \rangle  + o(\bar{h} )
\]

riscrivendolo per $x_0$:

\[
    f(x) = \underbrace{f(x_0)}_\text{punto di partenza} + \underbrace{\langle \nabla f(x_0), x-x_0 \rangle}_\text{il mio spostamento} + \underbrace{o(|x-x_0|)}_\text{l'errore che commetto spostandomi}
\]

e dunque la funzione lineare $\bar{x} \rightarrow  f(\bar{x_0} ) + \langle \nabla f(\bar{x_0} ), x-x_0 \rangle$ ci fornisce un'approssimazione lineare della $f(x)$ nel punto $x_0$ a meno di infinitesimi di ordine superiore alla distanza tra $x$ e $x_0$:

\[
    \underbrace{d(x,x_0)}_\text{distanza euclidea} = |x- x_0|
\]

Il grafico di questa funzione lineare rappresenta il piano tangente al grafico di $f$ nel punto $x_0$ in cui la $f$ è differenziabile. 

(Questo è chiaro in due variabili perché io effettivamente mi sto spostando lungo sia le $x$ che le $y$ e quindi sto disegnando un piano)

\subsection{Differenziale per le funzioni in due variabili} 

Consideriamo:

\[
    P_0 = (x_0,y_0)
\]

con $f$ differenziabile in $P_0$.

L'equazione del piano tangente al grafico di $f$ nel punto:

\[
    z= f(P_0) + \langle \nabla f(P_0), \underbrace{P-P_0}_\text{vettore in $\mathbb{R}^{2}$} \rangle
\]

ovvero, sapendo che $\nabla f(P_0) = ( \frac{\partial f}{\partial x}(x_0,y_0), \frac{\partial f}{\partial y}(x_0,y_0))$:

\[
    \underbrace{z = f(x_0,y_0) + \frac{\partial f}{\partial x}(x_0,y_0) (x-x_0) + \frac{\partial f}{\partial y}(x_0,y_0) (y-y_0)}_\text{equazione del piano tangente nel punto ($x_0,y_0,f(x_0,y_0)$)}
\]


Scriviamo la definizione di differenziale in $\bar{h} $ nel nostro caso in due variabili ($\bar{h} =(h,k)$):

\[
    f(x_0+h,y_0+k) = f(x_0,y_0) + \frac{\partial f}{\partial x}(x_0,y_0) h + \frac{\partial f}{\partial y}(x_0,y_0) k + \underbrace{o(\sqrt{h^{2}+k^{2}})}_\text{$|h|$}
\]

per $(h,k) \rightarrow  (0,0)$.

Se pongo:

\begin{equation}
    \begin{cases}
           x_0+h = x\\
           y_0+ k = y
    \end{cases}\,.
\end{equation}

allora:

\[
    f(x,y) = f(x_0,y_0) + \frac{\partial f}{\partial x}(x_0,y_0) (x-x_0) + \frac{\partial f}{\partial y}(x_0,y_0) (y-y_0) + \underbrace{o(\sqrt{(x-x_0)^{2}+(y-y_0)^{2}})}_\text{$d(P,P_0)$}
\]

E dunque per le funzioni differenziabili si ha che il piano di equazione $z= f(x_0,y_0) + \frac{\partial f}{\partial x}(x_0,y_0)(x-x_0) + \frac{\partial f}{\partial y}(x_0,y_0) (y-y_0)$ dista (ha un errore) dal grafico di $f$ per una quantità che va a zero più rapidamente di quanto $P$ si avvicina a $P_0$.

\[
    f(x,y) - [f(x_0,y_0) + \frac{\partial f}{\partial x}(x_0,y_0) (x-x_0) + \frac{\partial f}{\partial y}(x_0,y_0) (y-y_0)] = \underbrace{o(\sqrt{(x-x_0)^{2}+(y-y_0)^{2}})}_\text{$d(P,P_0)$}
\]

quindi:

\[
    \lim_{ (x,y) \to (x_0,y_0) } \frac{f(x,y) - [f(x_0,y_0) + \frac{\partial f}{\partial x}(x_0,y_0) (x-x_0) + \frac{\partial f}{\partial y}(x_0,y_0) (y-y_0)] }{\sqrt{(x-x_0)^{2}+(y-y_0)^{2}}}= 0
\]

La funzione $L: \mathbb{R}^{2} \rightarrow \mathbb{R}$:

\[
    L(x,y) = f(x_0,y_0) + \frac{\partial f}{\partial x}(x_0,y_0) (x-x_0) + \frac{\partial f}{\partial y}(x_0,y_0) (y-y_0)
\]

il cui grafico è il grafico del piano tangente al grafico $f$ si chiama \textbf{linearizzazione} di $f(x,y)$ in $(x_0,y_0)$


e l'applicazione delle $f$ mediante tale piano viene detta approssimazione lineare della $f$ in $(x_0,y_0)$

\textbf{Esercizio per casa} 

Data la funzione:

\[
    f(x,y) = \begin{cases}
        \frac{x^{3}+2y^{4}}{x^{2}+y^{2}} & \text{se $(x,y) \neq (0,0)$} \\
        0 & \text{se $(x,y) = (0,0)$}
    \end{cases}
\]

\begin{enumerate}
    \item Stabilire in quali punti del piano è derivabile, calcolando esplicitamente le derivate in tale caso
    \item Stabilire in quali punti del piano è differenziabile
    \item Stabilire il più grande aperto $A \subseteq \mathbb{R}^{2}$ su cui $f$ è $\mathbb{C}^{1}$
\end{enumerate}

\textbf{Soluzione} 

\begin{enumerate}
    \item Osserviamo che se $(x,y) \neq (0,0)$ non presenta problemi è derivabile; le sue derivate parziali:

        \[
            \frac{\partial f}{\partial x}(x,y) = \frac{3x^{2}(x^{2}+y^{2})-2x(x^{3}+2y^{4})}{(x^{2}+y^{2})^{2}}= \frac{x^{4}+3x^{2}y^{2}-4xy^{4}}{(x^{2}+y^{2})^{2}}
        \]

        \[
            \frac{\partial f}{\partial y}(x,y) = \frac{8y^{3}(x^{2}+y^{2}) - 2y(x^{3}+2y^{4})}{(x^{2}+y^{2})^{2}}= \frac{8x^{2}y^{3}+8y^{5}-2x^{3}y-4y^{5}}{(x^{2}+y^{2})^{2}} = \frac{4y^{5}+8x^{2}y^{3}-2x^{3}y}{(x^{2}+y^{2})^{2}}
        \]

        derivabilità in $(0,0)$:

        \[
            f(0,y) = \frac{2y^{4}}{y^{2}} = 2y^{2}
        \]

        \[
            \frac{\partial f}{\partial y}(0,0) = 0
        \]

        \[
            f(x,0) = \frac{x^{3}}{x^{2}} = x
        \]

        \[
            \frac{\partial f}{\partial x}(x,0) = 1 = f_x(0,0)
        \]

    \item 
        essendo $f \in \mathbb{C}^{1}(A)$ dove $A = \mathbb{R}^{2} \setminus \{(0,0)\} \rightarrow f \text{ è differenziabile}$, bisogna vedere che cosa succede in $(0,0)$

        Per essere differenziabile in $(0,0)$ deve essere:

        \[
            f(x,y) - f(0,0) - \frac{\partial f}{\partial x}(0,0) (x-0) - \frac{\partial f}{\partial y}(0,0) (y-0) = o(\sqrt{x^{2}+y^{2}})
        \]
        
        e quindi devo verificare:

        \[
            \lim_{ (x,y) \to (0,0) } \frac{ \frac{x^{3}+2y^{4}}{x^{2}+y^{2}}-0 -1(x-0) -0(y-0)}{\sqrt{x^{2}+y^{2}}} = \lim_{ (x,y) \to (0,0) } \frac{ \frac{x^{3}+2y^{4}- x^{3}-xy^{2}}{x^{2}+y^{2}}}{\sqrt{x^{2}+y^{2}}} = 0
        \]

        % \[
        %
        % % \lim_{ (x,y) \to (0,0) } \frac{2y^{4}-xy^{2}}{(x^{2}+y^{2})^{ \frac{3}{2}}}}
        %     
        % \]

        provo a risolverlo ponendo $y=x$:

        \[
            \lim_{ x \to 0 } \frac{2x^{4}-x^{3}}{(2x^{2})^{ \frac{3}{2}}} = \frac{1}{2 \sqrt{2}}
        \]

        proviamo a risolverlo $y=x^{2}$:

        \[
           \lim_{ (x,y) \to (0,0) }  \frac{2y^{4}-xy^{2}}{(x^{2}+y^{2})^{ \frac{3}{2}}} = \lim_{ x \to 0 } \frac{x^{5}(2x^{3}-1)}{x^{3}(1+x^{2})^{ \frac{3}{2}}} = 0
        \]

\end{enumerate}


\end{document}
