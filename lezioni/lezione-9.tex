\documentclass[../appunti-analisi.tex]{subfiles}

\begin{document}

\section{Lezione 9}

\subsection{Limiti di Funzioni in più variabili}

\defn{Punto di accumulazione}{

    Un punto $x_0 \in R^{n}$ di accumulazione per $A \subseteq R^{n}$ si dice punto di accumulazione se in ogni intorno circolare di $x_0$ c'è almeno un punto di $A$ diverso da $x_0$

}


\textbf{Esempi} 

\begin{itemize}
    \item I punti che costituiscono l'insieme dei punti interni di $A: \dot A$  sono punti di accumulazione
    \item I punti di frontiera, ovvero i punti di $\delta A$ possono essere punti di accumulazione di $A$ oppure non esserlo in quest'ultimo caso si dice che è un punto isolato
\end{itemize}


\defn{Convergenza in $\mathbb{R}^{n}$}{Data una successione $ \{x_n\}\in \mathbb{R}^{n}$ questa si dice che converge a $x_0 \in  \mathbb{R}^{n}$ se:

\[
    \lim_{ n \to +\infty } d(x_n,x_0)=0
\]

questo equivale a dire $\forall \varepsilon >0, \exists \bar{N} \in \mathbb{N}$ e $\forall n \ge \bar{N}$ si ha:

\[
    d(x_n,x_0) < \varepsilon
\]

}

\defn{Punto di accumulazione con limiti}{

    $x_0$ è di accumulazione per $A$ $\Leftrightarrow x_0$ è il limite di una successione di elementi di $A$ tutti diversi da $x_0$
}


\textbf{Esempio di punti di accumulazione} 

\[
    A \{\bar{x}  \in R^{2};1<\underbrace{|x|}_\text{d(x,0)}<2\}
\]


\begin{figure}[ht]
    \centering
    \incfig{disegno-punto-di-accumulazione-esempio-1}
    \caption{disegno punto di accumulazione esempio 1}
    \label{fig:disegno-punto-di-accumulazione-esempio-1}
\end{figure}


Tutti i punti di $A \in \mathbb{R}^{2}$ sono punti di accumulazione

I punti del disegno sono sia punti di frontiera che di accumulazione


Nel caso di $\emptyset, \mathbb{R}^{n}$ gli insiemi sono contemporaneamente sia aperti che chiusi.

\newpage

\defn{Chiusura di un insieme $A \subset \mathbb{R}^{n}$}{

    Si indica con $\bar{A} $ è un sottoinsieme di $\mathbb{R}^{n}$ dato dall'unione di $A$ e dei suoi punti di accumulazione ($DA$)

    $\bar{A} $ è un insieme chiuso. Lo si può pensare come l'intersezione dei chiusi contenenti $A$.

    Si può inoltre dimostrare che:

    \[
        \bar{A} = A \cup \delta A
    \]

}

\defn{Dominio}{ Un dominio $D$ in $\mathbb{R}^{n}$ è la chiusura di un insieme aperto:

    \[
        D = \bar{A}  = A \cup \delta A
    \]
}

\newpage 

Consideriamo ora le funzioni in più variabili $f: A \subset \mathbb{R}^{n} \rightarrow \mathbb{R}$

\defn{Limite di funzioni in più variabili}{

Sia $x_0 \in \mathbb{R}^{n}$ un punto di accumulazione per $A$

Si dice che $f(\bar{x})$ tende (ha limite) a $l$ per $\bar{x} $ che tende a $x_0$:

\[
    \lim_{ \bar{x}  \to x_0 } f(\bar{x} ) = l
\]

scrivendolo tramite gli intorni: se $\forall $ intorno $U \subset \mathbb{R}$ di $l$ esiste un intorno di $x_0$ (sferico) $I(x_0,r)$ con $r>0$ 

tale che $f(\bar{x}) \in U$ $\forall x \in \underbrace{I(x_0,r)}_{B(x_0,r)}\cap (A\{x_0\})$ 

L'altra definizione con i delta:

\[
    \forall \varepsilon>0\  \exists \delta >0
\]

tale che 

\[
    \underbrace{|f(x) - l|}_{d(f(x),l) \in \mathbb{R}} <\varepsilon
\]

$\forall \bar{x} \in A \setminus \{x_0\}$ con $|x-x_0| < \delta$
}

\subsection{Proprietà dei limiti di funzioni in più variabili}

Adesso parliamo di un po' di proprietà:

\begin{itemize}
    \item Il limite quando esiste è \textbf{unico}
    \item I limiti di \textbf{somme} e di \textbf{prodotti} di funzioni sono dati dalla somma e dal prodotto dei limiti (se definito)
    \item Il limite del \textbf{quoziente} di due funzioni è il quoziente dei limiti (se definito)
\end{itemize}

\textbf{Esempi} 

Sia 

\[
    f(x,y) = \frac{x^{2}}{\sqrt{x^{2}+y^{2}}}
\]

La mia $f: \underbrace{\mathbb{R}^{2}\setminus \{(0,0)\}}_\text{A aperto}\rightarrow \mathbb{R}$

Il punto $(0,0)$ è punto di accumulazione per $A$ 

Vogliamo vedere che succede quando la funzione tende a questo punto di accumulazione:

\[
    \lim_{ (x,y) \to (0,0) } f(x,y) = 0
\]

questo significa che $\forall \varepsilon >0 , \exists \delta >0$ tale che $|f(x,y) -0| = |f(x,y)| <\varepsilon$ $\forall (x,y) \in A = \mathbb{R}^{2}\setminus \{(0,0)\}$ con $0< \sqrt{x^{2}+y^{2}}<\delta$:

\[
    \underbrace{0\le}_\text{sempre positiva} f(x,y) = \frac{x^{2}}{\sqrt{x^{2}+y^{2}}} \le \frac{x^{2}+y^{2}}{\sqrt{x^{2}+y^{2}}} \overset{\text{razionalizzo}}{=} \sqrt{x^{2}+y^{2}}
\]

E dunque  $\forall \varepsilon$ si ha:

\[
    0 \le f(x,y) < \varepsilon
\]

per ogni $(x,y) \neq (0,0)$ t.c. $\sqrt{x^{2}+y^{2}} < \varepsilon$


\textbf{Esercizio per casa}

Mostrare che 

\[
    \lim_{ (x,y) \to (0,0) } \frac{x}{\sqrt{x^{2}+y^{2}}} \text{ non esiste}
\] 

\textbf{Soluzione} 

\begin{figure}[ht]
    \centering
    \incfig{esercizio-limite-casa}
    \caption{esercizio limite casa}
    \label{fig:esercizio-limite-casa}
\end{figure}

Se calcolo la funzione:

\[
    f(x,0) = \frac{x}{\sqrt{x^{2}}}= \frac{x}{|x|}
\]

questa fa:

\begin{equation}
    \begin{cases}
           1,x>0\\
           -1,x<0
    \end{cases}\,.
\end{equation}

Il limite quindi non esiste perché ha valori diversi a seconda del caso e non va bene

Invece:

\[
    f(0,y) = 0
\]

\end{document}
