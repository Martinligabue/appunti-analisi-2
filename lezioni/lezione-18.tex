\documentclass[../appunti-analisi.tex]{subfiles}

\begin{document}

\section{Lezione 18}

\defn{}{Sia $A \subseteq \mathbb{R}^{n}$ con $A$ aperto, $f:A \rightarrow R$ una funzione e sia $x_0 \in A$. Si dice che $f$ è differenziabile in $x_0$ se $\exists L:\mathbb{R}^{n}\rightarrow \mathbb{R}$ funzione lineare t.c.:

    \[
        \lim_{ h \to 0 } \frac{f(x_0+h) - f(x_0) -L(h)}{|h|} = 0
    \]

    dove $L(h) = \langle \nabla f(x_0), h \rangle$
}

\proposizione{}{Se $f: A \subseteq \mathbb{R}^{n}\rightarrow \mathbb{R}$, $A$ aperto, $x_0 \in A$.
Se $f$ è differenziabile in $x_0$

    \begin{itemize}
        \item $f$ è continua in $x_0$
        \item $f$ ha tutte le derivate direzionali in $x_0$ (secondo tutte le direzioni $\bar{v} $) e si ha $\frac{\partial f}{\partial v}(x_0) = D_{ \bar{v} } f(x_0) = L(\overrightarrow{v} ), \forall v \in \mathbb{R}^{n} $

            In particolare, $\overrightarrow{v} \mapsto \frac{\partial f}{\partial v}(x_0)$ è un'applicazione lineare e il differenziale è unico (se esiste).
    \end{itemize}
}

Vogliamo ora interpretare $\frac{\partial f}{\partial h}(x_0)= \langle \nabla f(x_0), h \rangle$:

\[
    \nabla f(x_0) = (\frac{\partial f}{\partial x}(x_0),\ldots, \frac{\partial f}{\partial x_n}(x_0))^{T}
\]

\[
    |\frac{\partial f}{\partial h}(x_0)| = |\langle \nabla f(x_0),h \rangle | \overset{\text{disuguaglianza di Cauchy-Swartz}}{\le } \underbrace{| \nabla f(x_0)|}_\text{lunghezza in $\mathbb{R}^{n}$} \underbrace{|\bar{h} |}_\text{lunghezza in $\mathbb{R}^{n}$}
\]

Se $|h| = 1$ è un vettore di norma 1 (direzione) allora:

\[
    |\frac{\partial f}{\partial h}(x_0) | \le \underbrace{|\nabla f(x_0) |}_\text{valore massimo che può assumere la derivata direzionale} 
\]

Si ha l'uguaglianza se $\nabla f(x_0) = 0$ oppure se $\bar{h} $ è parallela al vettore $\nabla f(x_0)$ (cioè $h = \lambda \nabla f(x_0), \lambda \ge 0$)

Quindi ricapitolando:

\begin{enumerate}
    \item Se $\nabla f(x_0) = 0$ allora \textbf{tutte} le derivate direzionali della $f$ in $x_0$ sono zero
    \item Se $\nabla f(x_0) \neq  0$ allora al variare di $\bar{h} $ nell'insieme di vettori di $\mathbb{R}^{n}$ di modulo (lunghezza) 1, si ha che la $\frac{\partial f}{\partial h}(x_0)$ è \textbf{massima} (raggiunge il suo massimo) quando:

        \[
            \bar{h} = \frac{\nabla f(x_0)}{\underbrace{|\nabla f(x_0) |}_\text{lunghezza}}
        \]
\end{enumerate}


\teorema{Regola della catena}{

consideriamo una curva continua $\gamma : [-1,1] \subset \mathbb{R} \rightarrow A\subseteq \mathbb{R}^{n}\ curva$ e supponiamo $\gamma(t)$ vettore (di $n$ componenti) sia derivabile, cioè:

\[
    \gamma(t) = (\gamma_1(t),\ldots,\gamma_n(t) )
\]

esiste:

\[
    \gamma'(t) = (\gamma_1'(t),\ldots,\gamma_n'(t))
\]

e supponiamo che $\gamma(0) = x_0 \in A$ e $\gamma'(0) = \bar{v} \in \mathbb{R}^{n}$, 


allora se $f: A \rightarrow \mathbb{R}$ è differenziabile in $x_0$, la funzione composta $F \rightarrow f(\gamma(t))$ da $[-1,1] \rightarrow \mathbb{R}$:

\[
    F= f \circ g : [-1,1] \rightarrow \mathbb{R}
\]

\[
    F(t) = (f \circ g) (t) = f(\gamma(t))
\]

è differenziabile in $0$:

\[
    F'(0) = \frac{\partial F}{\partial t} (0) = \frac{\partial (f \circ g)}{\partial t}(0) = \langle \nabla f(x_0), \underbrace{\gamma'(0)}_\text{direzione $\bar{v} $} \rangle
\]

Questo si chiama \textbf{teorema delle derivate delle funzioni composte} o regola della catena.

}

Dunque la crescita di $f$ in $x_0$ lungo una qualsiasi curva $\gamma$ regolare ($\gamma \in \mathbb{C}^{(1)}$) uscente da $x_0$ con velocità $\gamma'(0) = v$ dipende dalla curva solamente attraverso il vettore (tangente) velocità $\bar{v} $.


Ha dunque senso interpretare la derivata direzionale:

\[
    \frac{\partial f}{\partial \bar{v} } (x_0) = D_{\bar{v}}f(x_0)
\]

come la pendenza di $f$ in $x_0$ nella direzione $\bar{v} $ con $|v|=1$

\proposizione{}{Se $\nabla f(x_0) \neq 0$, allora il vettore $\nabla f(x_0) \in \mathbb{R}^{n}$ punta nel verso di massima pendenza.}

\begin{proof}
       Dimostrato dalle considerazioni fatte fino a ora.    
\end{proof}

\textbf{Proprietà} 

Se non è nulla, la lunghezza del gradiente di una funzione differenziabile in un punto, indica la direzione e il verso di \textbf{massima pendenza}  del grafico di $f$ nel punto. E dunque se considero la derivata direzionale di $f$ (nel punto) nella direzione del vettore gradiente trovo il valore della massima pendenza, che è appunto la lunghezza del vettore gradiente $|\nabla f(x_0)|$:

\[
    v_{max} 
\]

è la direzione del $\nabla f(x_0)$


\textbf{Esempio} 

Cerchiamo la \textbf{direzione di massima crescita}  per la funzione:

\[
    f(x,y) = xe^{y}
\]

nel punto $P=(2,0)$

Per quando detto, la direzione giusta è quella che risulta parallela al vettore gradiente:

\[
    \nabla f(P) = \nabla f(2,0) \in \mathbb{R}^{2}
\]

\[
    \frac{\partial f}{\partial x}(x,y)  = e ^{y}
\]

\[
    \frac{\partial f}{\partial y}(x,y) = xe^{y}
\]

quindi:

\[
    \nabla(P) = \nabla f(2,0) = (1,2)
\]


Calcoliamo la norma (lunghezza):
\[
    |\nabla f(P) | = \sqrt{1^{2}+2^{2}} = \sqrt{5}
\]

quindi:

\[
    \bar{v}_{max} = \frac{(1,2)}{\sqrt{5}} = (\frac{1}{\sqrt{5}}, \frac{2}{\sqrt{5}})
\]

e quindi abbiamo trovato la direzione di massima crescita.


Vediamo effettivamente la derivata direzionale della $f$ secondo questa direzione (la $\bar{v}$) (secondo la direzione del versore $\bar{v} =( \frac{1}{\sqrt{5}}, \frac{2}{\sqrt{5}})$ ha il valore massimo possibile che è la lunghezza di $\nabla f(P)$):

\[
    D_{\bar{v} } f(P) = D_{\bar{v} } f(2,0) = \langle \nabla f(2,0), \bar{v}  \rangle = (1,2) \cdot (\frac{1}{\sqrt{5}}\ \frac{2}{\sqrt{5}})^{T} = \frac{1}{\sqrt{5}}+ \frac{4}{\sqrt{5}} = \frac{5}{\sqrt{5}} \overset{\text{razionalizzo}}{=} \sqrt{5}
\]

\end{document}
