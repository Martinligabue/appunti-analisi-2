\documentclass[../appunti-analisi.tex]{subfiles}

\begin{document}

\section{Lezione 10}

\proposizione{}{Se 

    \[
        \lim_{ (x,y) \to (x_0,y_0) } f(x,y)=l
    \]

    allora per ogni sottoinsieme $C$ di $A$ (si sottintende che $P_0=(x_0,y_0)$ sia punto di accumulazione per $C$)

    Si deve avere:

    \[
        \lim_{ \underbrace{(x,y) \to (x_0,y_0)}_{(x,y) \in C} } f(x,y)=l
    \]

}

\textbf{Esercizi} 

\textbf{1} 

Mostriamo che:

\[
    \lim_{ (x,y) \to (0,0) } \frac{xy}{x^{2}+y^{2}}
\]

non esiste.

Restringiamo lo studio di funzione lungo l'asse x ($y=0$):

\[
    f(x,0) = 0
\]

\[
    \lim_{ \underbrace{(x,y) \to (0,0)}_{y=0} } f(x,y)
\]

Stessa cosa lungo l'asse y ($x=0$):

\[
    \lim_{ \underbrace{(x,y) \to (0,0)}_{x=0} } f(x,y)
\]

Candidato limite è a 0 


Adesso ci spostiamo con altri parametri tipo la bisettrice del primo e del terzo quadrante $y=x$:

\[
    \lim_{ \underbrace{(x,y) \to (0,0)}_{y=x} } f(x,y) = \lim_{ x \to 0 } f(x,x)=
\]


\[
    = \lim_{ x \to 0 } \frac{x^{2}}{x^{2}+y^{2}} = \lim_{ x \to 0 } \frac{x^{2}}{2x^{2}}= \frac{1}{2}
\]

\textbf{2} 

\[
    \lim_{ (x,y) \to (0,0) } \frac{xy^{2}}{x^{2}+y^{4}}
\]

$f:\mathbb{R}^{2}\setminus (0,0) \rightarrow \mathbb{R}$

per $y=0$ viene a 0 e anche per $x=0$

Considero quindi qualunque retta passante per l'origine:

\[
    y= mx
\]

con $m \neq 0$ e $x \neq 0$:

\[
    \lim_{ \underbrace{(x,y) \to (0,0)}_{y=mx} } f(x,y) = \lim_{ x \to 0 } f(x,mx) = \lim_{ x \to 0 } \frac{m^{2}x^{3}}{x^{2}+m^{4}x^{4}}= 0
\]

ma questo non basta, devo controllare anche il caso della parabola $y^{2}=x$:


\[
    \lim_{ \underbrace{(x,y) \to (0,0)}_{x=y^{2}} } f(x,y) = \lim_{ y \to 0 } f(y^{2},y) = \lim_{ y \to 0 } \frac{y^{4}}{y^{4}+y^{4}} = \lim_{ y \to 0 } \frac{y^{4}}{2y^{4}} = \frac{1}{2} \neq 0
\]

\defn{Funzione continua in più variabili}{ Sia una funzione e sia $P_0$ un punto di accumulazione per a, si dice che la funzione è continua in $P_0$ se:

    \[
        \lim_{ P \to P_0 } f(P) = f(P_0)
    \]

    se $P_0$ è un punto isolato per $A$ per convenzione $f$ è continua

}

\textbf{Esempi} 

Avendo queste due funzioni 

\[
    f(x,y)=x
\]

\[
    g(x,y) = y
\]

devo mostrare che $f$ e $g$ sono continue in ogni punto:

Consideriamo la $f$
 
Sia dunque $(x_0,y_0) \in  \mathbb{R}^{2}$ e $\varepsilon >0$ dobbiamo mostrare che $\exists \delta= \delta(\varepsilon) >0$:

\[
     d(f(x,y)- f(x_0,y_0) ) < \varepsilon
\]

se $d(P,P_0) < \delta$ 

scritto meglio

\[
    d(P,P_0) = \sqrt{(x-x_0)^{2}+(y-y_0)^{2}}
\]

mostriamo che $\sqrt{(x-x_0)^{2}+(y-y_0)^{2}} < \delta $ si ha $|x-x_0|< \varepsilon$:


\[
|x-x_0| = \sqrt{(x-x_0)^{2}}\le \sqrt{(x - x_0) ^{2} + (y- y_0) ^{2}} < \delta
\]

dobbiamo prendere quindi $ \delta =\varepsilon$


\teorema{}{Siano $f$ e $g$ continue (sugli opportuni domini) allora:

    \begin{itemize}
        \item $f+g , f\cdot g$ sono continue 
        \item se $g \neq 0$ allora $\frac{f}{g}$ è continua
        \item se $g>0$ allora $f^{g}$ è continua
        \item la funzione comporta $g \circ f$ è continua (dove è definita)
    \end{itemize}
}

Sono dunque funzioni continue:

\begin{itemize}
    \item I polinomi in due variabili 
    \item Le funzioni razionali (rapporti, quoziente di polinomi)
    \item Le funzioni elementari 
\end{itemize}


Condizione necessaria affinché $f(x,y)$ ammetta limite $l$ quando $(x,y) \rightarrow  (x_0,y_0)$ è che per ogni curva regolare di equazione:

\begin{equation}
    \begin{cases}
           x=x(t)\\
           y=y(t)
    \end{cases}\,.
\end{equation}

questa è una curva  passante per $P_0= (x_0,y_0)$:

\[
    \lim_{ t \to t_0 } f(x(t), y(t)) 
\]

si arriva alla stessa conclusione di non esistenza del limite se la restrizione di $f(x,y)$ ad una curva (come sopra) non ha limite. Ovviamente non è vero il viceversa

\subsection{Coordinate polari}

Abbiamo $(\rho,\theta)$ dove:

\[
    \rho = \bar{ OP} = d(P,O) = \sqrt{x^{2}+y^{2}}
\]

\[
    \theta = arctan \frac{y}{x}
\]

\begin{equation}
    \begin{cases}
           x= x_0+ \rho cos \theta\\
           y = y_0+ \rho sin \theta
    \end{cases}\,.
\end{equation}

Scriviamo i limiti con le coordinate polari:

\[
    \lim_{ (x,y) \to (x_0,y_0) } f(x,y)
\]

\begin{figure}[ht]
    \centering
    \incfig{disegno-polari}
    \caption{disegno polari}
    \label{fig:disegno-polari}
\end{figure}

\teorema{}{Sia $f: D \subset \mathbb{R}^{2} \rightarrow  \mathbb{R}$ e sia $P_0=(x_0,y_0) \in D$ allora:

    \[
        \lim_{ (x,y) \to (x_0,y_0) } f(x,y)  = l \Leftrightarrow \lim_{ \rho \to 0^{+} } f(x_0+\rho cos \theta, y_0+\rho sin \theta) = l
    \]

    uniformemente rispetto a $\theta$

}


\end{document}
